\section{Diskussion}
\label{sec:Diskussion}
Die Kurve des Magnetfelds in \autoref{fig:plot1.pdf} weißt ein relativistischen Verlauf auf, wodurch das Maximum gut abgeschätzt werden kann.\\
Als Literaturwert für die effektive Masse wird $m*_{\text{lit}}= 0.067 m_e$ (\cite{effektiveMasseGaAs}) benutzt. Diw Abweichung zu der den gemessenen
Werten ergibt sich bei der Dotierung von $N = \qty{1.2}{\centi\meter^(-3)}$ zu $\qty{61.19}{\percent}$ und bei der Dotierung von $N = \qty{2.8}{\centi\meter^(-3)}$
zu $\qty{23.88}{\percent}$. Die erhöhten Abweichungen lassen sich größtenteils aus den menschlichen Ablesefehlern herleiten. 
Eine Fehlerquelle ist das Einstellen des Winkels, was per Hand eingestelllt werden musst, da die minimale Signalamplitude am Oszilloskop relativ ungenau
abzulesen war. Aufgrund des Rauschens musste teilweise die minimalste Signalamplitude abgeschätzt werden, was zusätzlich zu Winkelunsicherheiten 
führt. Um diese Fehlerquelle zu verringern kann die Anzahl der verwendeten Interferenzfilter erhöht werden. Eine weitere Fehlerquelle ist, das die Magnetspulen
sich erwärmt haben.\\
Zusammenzufassend lässt sich sagen, dass diese Methode, sobald sie automatisiert wird, eine gute Methode ist um den Faraday-Effekt nachzuweisen.


