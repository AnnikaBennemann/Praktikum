\section{Diskussion}
\label{sec:Diskussion}
Die Kurve des Magnetfelds in \autoref{fig:plot1} weist einen eindeutigen Verlauf auf, wodurch das Maximum gut abgeschätzt werden kann.\\
Als Literaturwert für die effektive Masse wird $m*_{\text{lit}}= 0.067 m_e$ \cite{effektiveMasseGaAs} herangezogen. Die Abweichung von den gemessenen
Werten ergibt sich bei der Dotierung von $N_{1,2}= \num{1.2 e24} \unit[per-mode=reciprocal]{\per\cubic\meter}$ zu $\qty{22.39}{\percent}$ und bei der Dotierung 
von $N_{2,8}= \num{2.8 e24} \unit[per-mode=reciprocal]{\per\cubic\meter}$zu $\qty{7.46}{\percent}$.
Die Abweichungen lassen sich größtenteils aus menschlichen Ablesefehlern herleiten. 
Eine Fehlerquelle ist das Einstellen des Winkels, was per Hand durchgeführt wird, da die minimale Signalamplitude am Oszilloskop relativ ungenau
abzulesen war. Aufgrund des Rauschens musste die minimale Signalamplitude teilweise abgeschätzt werden, was zusätzlich zu Winkelunsicherheiten 
führt. Um diese Fehlerquelle zu verringern kann die Anzahl der verwendeten Interferenzfilter erhöht werden. Eine weitere Fehlerquelle ist, dass die Magnetspule
sich erwärmt hat. Dies könnte durch Wartezeiten zwischen den Messreihen verringert werden um die Messungen bei jeweils derselben Temperatur
der Spule durchführen zu können.
Weitere systematische Fehler können aufgrund unzureichender Justage des Versuchsaufbaus entstehen.\\
Zusammenzufassend lässt sich sagen, dass diese Methode, sobald sie automatisiert wird, eine gute Methode ist um den Faraday-Effekt nachzuweisen.
\newpage

