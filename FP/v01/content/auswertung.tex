\section{Auswertung}
\label{sec:Auswertung}

Um den Vielkanalanalysator zu kalibrieren werden mit Hilfe eines Doppelimpuls-Generators Impulse
mit unterschiedlichem zeitlichem Abstand eingelesen. Die belegten Kanäle sind in dargestellt
und in gegen die Impulsabstände aufgetragen.



Zeugs für Halbwertsbreite: 

Der Graph zur Justierung der Verzörgerungsleitungen in \autoref{fig:Verzoergerung} weist ein leichtes Plateau auf.
Dieses Plateau ist, wie in der Durchführung beschrieben, auf die normierte Pulsdauer der Ausgangssignale der Diskriminatoren zurückzuführen und befindet sich im erwarteten Bereich von \SI{10}{\nano \second}. \\
Da das Plateau nicht sehr stark ausgeprägt ist, wurde eine Gaußfunktion höherer Potenz, anstatt einer Rechteckfunktion als Anpassungsfunktion gewählt.
Dementsprechend ist die Halbwertsbreite der angepassten Funktion mit $T_{\text{FWHM}} = \SI{16,4}{\nano \second}$ größer als das eigentliche Plateau.


Die Auflösungszeit~$\symup{\Delta}t_K$ der Koinzidenzapparatur ist gleichbedeutend
mit der Differnez zwischen der Summe der Diskriminatorbreiten und der in grün
eingezeichneten Halbwertsbreite. Zur Bestimmung von dieser wird
ein linearer Fit für das Plateau und die beiden Flanken durchgeführt. Durch die halbe
Höhe des Plateau ist anschließend die Halbwertsbreite gegeben. Durch die abgelesenen Schnittpunkte
mit den beiden Flanken lässt sich nun die Verzögerungszeit zu
\begin{equation}
  \symup{\Delta}t_K = 40 - \left[ |-25 \pm 0.5| + |18 \pm 0.5| \right] \si{\nano\second} = \- 3 \pm 0.71\; \si{\nano\second}
\end{equation}
bestimmen. Dabei ergeben sich die \SI{40}{\nano\second} aus der Summation der Breiten der beiden Diskriminatoren.
