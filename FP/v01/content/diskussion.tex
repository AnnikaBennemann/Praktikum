\section{Diskussion}
\label{sec:Diskussion}

Die im Versuch bestimmte mittlere Lebensdauer der Myonen
\begin{align*}
    \tau &= (2.24 \pm 0.04)\si{\micro\second}
\end{align*}
weicht
um $\qty{1.81 \pm 1.81}{\percent}$ vom Literaturwert\cite{PDG} ab.
Die Genauigkeit des Versuchs ist also sehr hoch, was daran liegen kann, dass viele Anpassungen zur Rauschunterdrückung
vorgenommen wurden um Fehler zu minimieren.
Allerdings gab es auch mehrere Fehlerquellen.
Das Oszilloskop zum Beispiel war nicht sehr gut ablesbar, zudem kommt es durch menschliches Ablesen zu Fehlern.
Bei der Justage bzw Kalibration wurde außerdem die Messung per Hand gestartet und gestoppt, sodass auch hier Abweichungen zu erwarten sind.
Außerdem arbeitete der Impulszähler nicht immer zuverlässig und musste zwischen den Messungen mehrmals an- und ausgeschaltet werden.
Eine genauere Bestimmung der Lebensdauer der kosmischen Myonen kann durch eine längere Messung über mehr Kanäle erreicht werden.
