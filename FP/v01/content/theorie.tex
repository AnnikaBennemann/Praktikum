\section{Zielsetzung}
\label{sec:Zielsetzung}
Ziel dieses Versuchs ist die Bestimmung der Lebensdauer von kosmischen Myonen.


\section{Theorie}
\label{sec:Theorie}

Myonen ($\mu^-$) sind Elementarteilchen, die zur zweiten der drei Leptonenfamilien gehören.
Sie sind Fermionen und besitzen somit einen Spin von $\frac{1}{2}$.
Außerdem sind sie wie Elektronen einfach negativ geladen (die Antimyonen einfach positiv), besitzen jedoch mit $\qty{105.66}{\mega\electronvolt}$
eine ungefähr 207-mal höhere Ruhemasse als Elektronen.
Die mittlere Lebensdauer von Myonen beträgt $\qty{2.2}{\micro\second}$. \cite{PDG}
Myonen unterliegen der schwachen und der elektromagnetischen Wechselwirkung.
Sie sind der Hauptbestandteil von sekundärer kosmischer Strahlung.
Diese entsteht, wenn primäre kosmische Strahlung (hochenergetische Protonen aus der Sonne uns anderen Sternen) mit Atomkernen
der oberen Atmosphäre wechselwirkt. Dabei werden in circa $15-\qty{20}{\kilo\meter}$ Höhe Pionen und Kaonen erzeugt,
welche wiederum in circa $\qty{15}{\kilo\meter}$ Höhe durch die schwache Wechselwirkung hauptsächlich zu Myonen zerfallen.
\begin{align*}
    \pi^+ \longrightarrow \mu^+ + \nu_{\mu} \\
    \pi^- \longrightarrow \mu^- + \overline{\nu}_{\mu}
\end{align*}
Der Zerfall von Kaonen erfolgt analog.

 

\begin{align*}
    \mu^- \longrightarrow \nu_{\mu} + \text{e}^- + \overline{\nu}_{\text{e}} \\
    \mu^+ \longrightarrow \overline{\nu}_{\mu} + \text{e}^+ + \nu_{\text{e}} 
\end{align*}


Wenn das Myon eine Energie von $E_{\mu}$=\SI{10}{\giga\electronvolt} besitzt, kann die Geschwindigkeit des Myons $v_{\mu}$ klassisch durch
Umstellen von 
\begin{align*}
    E_{\mu}^2 &= m_{\mu}^2c^4 + p_{\mu}^2c^2
    \shortintertext{mit}
    p_{\mu}&=\gamma_{\mu} \text{m}_{\mu}v_{\mu}
    \shortintertext{zu}
    v_{\mu} &\approx 0.998c
\end{align*}
berechnet werden.
Daraus ergibt sich nach dem Weg-Zeit-Gesetz die Reichweite
\begin{align*}
    s = v_{\mu} \cdot \tau_{\mu} \approx \SI{658.22}{\meter}.
    \label{eqn:s}
\end{align*}
Da die Myonen, wie oben erwähnt auf einer Höhe von circa $\qty{15}{\kilo\meter}$ entstehen, würden sie nach dem klassischen Ansatz den Erdboden nicht erreichen.
Dies widerspricht allerdings den experimentellen Erkentnissen.
Erklären lässt sich dies mithilfe der Zeitdilatation. So bezieht sich die Eigenzeit $\tau_{\mu}$ auf das Ruhesystem des Myons. Im Ruhesystem
des Beobachters auf der Erde muss \autoref{eqn:s} also noch durch den Lorentzfaktor ergänzt werden und lautet somit
\begin{equation*}
    s = v_{\mu} \cdot \gamma_{\mu} \tau_{\mu} \approx \SI{0}{\meter}.
    \label{eqn:s2}
\end{equation*}

