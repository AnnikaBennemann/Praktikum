\section{Zielsetzung}
\label{sec:Zielsetzung}
Ziel dieses Versuchs ist die Bestimmung der Lebensdauer von kosmischen Myonen.


\section{Theorie}
\label{sec:Theorie}

Myonen ($\mu^-$) sind Elementarteilchen, die zur zweiten der drei Leptonenfamilien gehören.
Sie sind Fermionen und besitzen somit einen Spin von $\frac{1}{2}$.
Außerdem sind sie wie Elektronen einfach negativ geladen (die Antimyonen einfach positiv), besitzen jedoch mit $\qty{105.66}{\mega\electronvolt}$
eine ungefähr 207-mal höhere Ruhemasse als Elektronen.
Die mittlere Lebensdauer von Myonen beträgt $\qty{2.2}{\micro\second}$. \cite{PDG}
Myonen unterliegen der schwachen und der elektromagnetischen Wechselwirkung.
Sie sind der Hauptbestandteil von sekundärer kosmischer Strahlung.
Diese entsteht, wenn primäre kosmische Strahlung (hochenergetische Protonen aus der Sonne uns anderen Sternen) mit Atomkernen
der oberen Atmosphäre wechselwirkt. Dabei werden in circa $10-\qty{15}{\kilo\meter}$ Höhe Pionen und Kaonen erzeugt,
welche wiederum in circa $\qty{10}{\kilo\meter}$ Höhe durch die schwache Wechselwirkung hauptsächlich zu Myonen zerfallen.
\begin{align*}
    \pi^+ \longrightarrow \mu^+ + \nu_{\mu} \\
    \pi^- \longrightarrow \mu^- + \overline{\nu}_{\mu}
\end{align*}
Der Zerfall von Kaonen erfolgt analog.

Die entstandenen kosmischen Myonen zerfallen spontan durch schwache Wechselwirkung.

\begin{align*}
    \mu^- \longrightarrow \nu_{\mu} + \text{e}^- + \overline{\nu}_{\text{e}} \\
    \mu^+ \longrightarrow \overline{\nu}_{\mu} + \text{e}^+ + \nu_{\text{e}} 
\end{align*}
\\

Teilchenzerfälle sind statistische Prozesse. Daher wird dem Zerfall des Myons
eine Wahrscheinlichkeit 
\begin{align}
    \text{d}W &= \lambda \text{d}t.
    \label{eqn:dW}
\end{align}
zugeordnet, mit der dieses in einer bestimmten Zeit $\text{d}t$ zerfällt.
Die Anzahl an Teilchen $\text{d}N$ die in einem infinitessimalen Zeitraum $\text{d}t$ mit der Wahrscheinliichkeit $W$ zerfallen ergibt sich
aus \eqref{eqn:dW} zu 
\begin{align*}
    \text{d}N &= -\lambda N\text{d}t.
\end{align*}
Hieraus lässt sich das Zerfallsgesetz 
\begin{align}
    N(t) &= N_0 \cdot e^{- \lambda \cdot t}\label{eqn:Zerfallsgesetz}
\end{align}
durch Umstellen und Integrieren herleiten. $N_0$ ist dabei die Anzahl der zum Startzeitpunkt vorhanden Teilchen. 

Die Zerfallskonstante ist hierbei der Proportionalitätsfaktor $\lambda$, wobei
\begin{align*}
    \lambda &= \frac{1}{\tau}
\end{align*}
gilt. Die mittlere Lebensdauer $\tau$ beschreibt hierbei die Zeit nach der die Anzahl der Myonen auf den Bruchteil $\frac{1}{e}$ abgefallen ist.
\\
Wenn das Myon eine Energie von $E_{\mu}$=\SI{10}{\giga\electronvolt} besitzt, kann die Geschwindigkeit des Myons $v_{\mu}$ durch
Umstellen von 
\begin{align*}
    E_{\mu}^2 &= m_{\mu}^2c^4 + p_{\mu}^2c^2
    \shortintertext{mit}
    p_{\mu}&=\gamma_{\mu} \text{m}_{\mu}v_{\mu}
    \shortintertext{zu}
    v_{\mu} &\approx 0.994c
\end{align*}
berechnet werden.
Daraus ergibt sich klassisch nach dem Weg-Zeit-Gesetz die Reichweite
\begin{align}
    s = v_{\mu} \cdot \tau_{\mu} \approx \SI{655.6}{\meter}.
    \label{eqn:s}
\end{align}
Da die Myonen, wie oben erwähnt auf einer Höhe von circa $\qty{10}{\kilo\meter}$ entstehen, würden sie nach dem klassischen Ansatz den Erdboden nicht erreichen.
Dies widerspricht allerdings den experimentellen Erkentnissen.
Erklären lässt sich dies mithilfe der Zeitdilatation. So bezieht sich die Eigenzeit $\tau_{\mu}$ auf das Ruhesystem des Myons. Im Ruhesystem
des Beobachters auf der Erde muss \autoref{eqn:s} also noch durch den Lorentzfaktor ergänzt werden und lautet somit
\begin{align}
    s = v_{\mu} \cdot \gamma_{\mu} \tau_{\mu} \approx \SI{6206.0}{\meter}.
    \label{eqn:s2}
\end{align}
Für energiereichere Myonen mit einer Energie von $E_{\mu}$=\SI{20}{\giga\electronvolt} beträgt die relativistische Reichweite sogar $\qty{12451}{\meter}$.
Der Fluss der Myonen beträgt circa ein Myon pro \si{\square\centi\metre} pro Minute auf Meereshöhe.\cite{Grupen}
Die horizontale Querschnittsfläche eines liegenden Zylinders mit $\qty{50}{\liter}$ Volumen und einer Höhe von 2r beträgt
$\qty{1594.36}{\square\centi\metre}$, somit lässt sich die Ereignisrate im Zylinderquerschnitt zu 1594 Myonen pro Minute bestimmen.




\subsection{Grundlagen zum Szintillator}
\label{subsec:Szintillator}
In einem Szintillator trifft einfallende Strahlung auf Atome des Szintillatormediums, welche dadurch entweder ionsiert oder angeregt werden.
Anschließend wird die Anregungsenergie in Form von Photonen wieder abgegeben, wenn die angeregten Atome wieder relaxieren.
Wenn das Myon innerhalb des Szintillators zerfällt regt das freigewordene Elektron auch ein Atom des Szintillatormediums an und
erzeugt somit ein weiteres Signal.

Ein Szintillator wird dazu benutzt, die Energie und die Intensität ionisierender Strahlung zu messen. Dabei ergibt sich die Energie 
jedes einzelnen Stoßvorgangs aus der abgegeben Lichtmenge.
Die Lichtpulse können in elektrische Signale umgewandelt und weiterverarbeitet werden.
Es gibt zwei Arten von Szintillatoren.\\


Organische Szintillatoren können Kristalle, Flüssigkeiten und polymere Festkörper sein. 
Der Mechanismus der Szintillation ist vor allem von der Molekülstruktur des Kohlenstoffs bestimmt.
Durch Energieabsorption im Szintillator wird ein organisches Molekül vom Grundzustand in ein höheres Energieniveau angeregt.
Bei der Relaxation dieses Molküls in den Grundzustand zurück, kommt es zur Aussendung von einem Photon.
Organische Szintillatoren haben eine Ansprechzeit von nur wenigen Nanosekunden und eignen sich daher sehr gut für die Messung der Lebensdauer
von kosmischen Myonen.

Anorganische Szintillatoren sind Kristalle, welche durch ihre Regelmäßigkeit eine ausgeprägte Bandstruktur aufweisen.
Die Szintillation findet nicht aufgrund molekularer Geschehnisse, sondern im Rahmen des Bandschemas statt.
Aufgrund von Störstelle im Kristall und durch die Dotierung mit fremdatomen enststehen sogenannte Aktivator-Zentren.
Ionisierende Strahlung erzeugt beim Auftreffen auf das Szintillatormedium freie 
Elektronen, freie Löcher oder Elektron-Loch-Paare. Diese Anregzustände wandern umher, bis sie auf einen Aktivatorzustand treffen und diesen
anregen. Unter Emission von sichtbarem Licht fällt dieser wieder in den Grundzustand zurück.

Anorganische Szintillatoren haben im Vergleich zu organischen Szintillatoren eher lange Ansprechzeiten von mehreren hundert Nanosekunden,
allerdings haben sie eine relativ hohe Strahlenfestigkeit und es gibt weniger Alterungseffekte. Zudem sind sie häufig hygroskopisch.

Durch Messung der Zeit zwischen dem Eintritt des Myons und des
Zerfalls ist eine Bestimmung der mittleren Lebensdauer des Myons möglich.
Es wird also innerhalb eines bestimmten Zeitrahmens nach Eintritt eines Myons ein weiteres Signal
erwartet.
Sollte innerhalb dieses Zeitrahmens ein weiteres Myon den Detektor erreichen oder das
Myon nicht innerhalb des Detektors zerfallen, wird die Messung verfälscht.
Diese Verfälschungen der Messung erzeugen einen Untergrund
\begin{equation}
  U = P_{\lambda} (k) = \frac{\lambda^{k}}{k!} \exp{(-\lambda)},
  \label{eqn:Untergrundrate}
\end{equation}
welcher mit einer Poissonverteilung mit Erwartungswert $\lambda$ der Myonen pro Sekunde und der Ereignisanzahl $k$ angenähert wird.

