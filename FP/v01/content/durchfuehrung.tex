\section{Aufbau}
\label{sec:Aufbau}
Ein Schaltbild des Versuchsaufbaus ist in \autoref{fig:Schaltbild} dargestellt. 
\begin{figure}[H]
    \centering
    \includegraphics[width=0.5\textwidth]{Abbildungen/Schaltung.png}
    \caption {Schematische Dastellung des Versuchsaufbaus.\cite{V01}}
    \label{fig:Schaltbild}
\end{figure}
Dabei besteht der Versuchsaufbau aus einem Szintillatortank, welcher ein Volumen von ungefähr $\qty{50}{\liter}$ fasst.
An beiden Ausgängen des Szintillatortanks ist je ein Photomultiplier (PMT) gekoppelt, welcher schwache Lichtsignale durch Erzeugung und Verstärkung eines elektrischen Signals detektiert.
Dabei werden an einer Photokathode durch den Photoeffekt Elektronen ausgelöst, welche über mehrere positiv geladene Dyonen beschleunigt werden.
An jeder Dynode vervielfacht sich die Zahl der Elektronen. Die Elektronen treffen am Ende des PMTs auf eine Anode, wo das ursprüngliche
Signal stark verstärkt in ein elektronisches Signal umgewandelt wird.
Die Ausgänge der PMTs werden über eine Verzögerungsleitung auf den den Eingang jeweils eines Diskriminators gegeben, welcher eine variable Schwelle besitzt. 
Anschließend gelangen die Signale beider PMT in eine Koinzidenzschaltung, welche nur dann ein Ausgangssignal erzeugt, wenn beide Signale der PMT zeitgleich an den Eingangen der Koinzidenzschaltung
eintreffen.\\
Danach folgt das elektronische Äquivalent einer Stoppuhr.
Das Ausgangsignal der Koinzidenzschaltung gelangt zu zwei AND-Gattern und über eine weitere Verzögerungsleitung ($\qty{30}{\nano\second}$) zu einem Monoflop. Dieser gibt die Suchzeit $T_S$ vor.
Aufgrund der Verbindung der AND-Gitter mit dem Time-Amplitude-Converter (TAC), startet die Zeitessung durch das Signal des ersten AND-Gatters,
wenn ein Myon das aktive Volumen betritt. Die Zeitmessung stoppt aufgrund eines Signal des anderen AND-Gatters, wenn das Myon zerfällt. 
Außerdem wird die Zeitmessung bzw. der Monoflop zurückgesetzt wenn die Suchzeit $T_S$ abgelaufen ist, ohne dass ein Stoppsignal ausgelöst wurde.
Die Start- und Stoppimpulse werden mithilfe von Impulszählern gezählt.
Das Signal des TAC gelangt zu einem Vielkanalanalysator (bzw. Multi-Channel-Analyser) (MCA). Dieser ordnet jedem Signal entsprechend
seiner Größe einen Kanal innerhalb eines Datensatzes zu und erhöht den Wert des entsprechenden Kanals um Eins. Die Anzahl der in einem Kanal
gespeicherten Signale entspricht also der Häufigkeit des Auftretens eines Signals mit einer bestimmten Lebensdauer.


\subsection{Rauschunterdrückung}
\label{subsec:Rauschunterdrückung}
Bei endlichen Temperaturen neigen die Photokathoden in den PMTs zu spontaner Elektronemission, wobei kleine Impulse entstehen.
Diese Impulse sind kleiner als \dq echte \dq Impulse und können so mit einem Diskriminator herausgefiltert werden.
Dieser lässt nur Signale durch, die eine Mindestspannung überschreiten. Die Schwelle wird
dabei so gesetzt, dass möglichst nur die ungewollten Impulse rausgefiltert werden. Da die Elektronenemission an beiden PMTs unabhängig
voneiander geschieht, wird eine Koinzidenzschaltung verwendet, sodass die Signale der PMT
nur weitergeleitet werden wenn ein Myon eintrifft und zerfällt und somit beide PMT gleichzeitig ein Signal aufnehmen und zur Koinzidenz
weiterleiten. Die Wahrscheinlichkeit, dass beide PMTs gleichzeitig ein Signal weiterleiten das nicht von einem Myon oder dessen Zerfall
stammt ist sehr gering. Diese Signale können allerdings nicht weiter von den echten Signalen unterschieden werden.


\subsection{Stoppuhr-Methode}
\label{subsec:Messmethode}
Es kann sein, dass Myonen in den Szintillator gelangen und dann entweder vom Szintillatormedium absorbiert werden oder sie genug 
Energie haben um komplett durch das Szintillatormaterial und den Detektor zu gelangen ohne zu zerfallen.
In beiden Fällen kann die Lebenszeit von Myonen nicht bestimmt werden.
Es wird somit von diesen Myonen nur ein Impuls gesendet und der Impuls des nächsten eintreffenden Myons würde als Zerfallsimpuls des ersten
registriert werden. Um disesm entgegen zu wirken
wird die Stoppuhr-Methode benutzt. Dabei beginnt eine Suchzeit $T_S$ nach dem Startimpuls des Myons. Zerfällt das Myon nicht innerhalb der Suchzeit,
wird die Apparatur wieder in den Anfangszustand zurückgebracht.\\
Dieses grundlegende Messprinzip wird schaltungstechnisch durch den Monoflop und die zwei AND-Gitter geregelt.
Der Monoflop hat einen stabilen und einen instabilen Zustand.
Die Zeit, in der der Monoflop im instabilem Zustand ausgelenkt ist, sollte ungefähr der Suchzeit entsprechen, denn im stabilen Zustand sendet
der Monoflop ein Signal an das zweite AND-Gitter, welches die Messung stoppt.
Die Koinzidenzschaltung lenkt dabei den Monoflop in den instabilen Zustand aus und das TAC misst die Zeit. Kommt es dann in der Suchzeit zu keinem
Stromimpuls, der über das zweite AND-Gatter an das TAC weitergeleitet wird, wird die Messung verworfen und das TAC gibt kein Signal ab.
Um die Anzahl der verworfenen Messungen zu bestimmen, wird ein weiterer Impulszähler angeschlossen.\\
Treffen zwei Myonen gleichzeitig innnerhalb der Suchzeit im Tank ein, kommt es zu einer Fehlermessung, die auch Untergrund genannt wird.
Dabei ist der Zeitabstand in dem zwei Myonen eintreffen statistisch verteilt und füllt daher die Kanäle des Vielkanalanalysator gleichmäßig auf.

\section{Durchführung}
\label{sec:Durchführung}
Für die Durchführung des Versuchs wird die Schaltung wie in \autoref{sec:Aufbau} aufgebaut und anschließend kalibriert.
Dabei wird zur Überprüfung der Schaltung ein Oszilloskop benutzt.\\
Zur Einstellung der Schwellspannung der Diskriminatoren, werden zunächst die Spannungsimpulse der Photomultiplier mit dem Oszilloskop überprüft.
Die Spannungsimpulse sollten vor den Diskriminatoren von unterschiedlicher Höhe und Länge sein und die Diskriminatorschwelle wird so eingestellt,
dass an beiden Ausgängen ungefähr $30$ Impulse pro Sekunde gemessen werden, welche nun die gleiche Höhe besitzen.
Die Pulsdauer wird auf $\Delta t = \qty{10}{\nano\second}$ eingestellt, was klein gegenüber der Lebensdauer der Myonen ist, damit sich zwei Impulse
möglichst nicht überlagern.\\
Anschließend wird die Koinzidenzschaltung eingestellt, indem die Verzögerungsleitungen so gewählt werden, dass die Signale gleichzeitig eintreffen. Dabei wird der Messbereich so gewählt,
dass sich die Halbwertsbreite der Verteilung bestimmen lässt. Die Ereignisrate soll im Bereich von $\qty{20}{\second^{-1}}$ liegen.\\
Danach wird der restliche Teil der Schaltung verkabelt, eine Suchzeit $T_S$ eingestellt und der Messbereich des TAC entsprechend angepasst.
Der MCA kann über einen Doppelimpulsgenerator kalibriert werden. Dieser Doppelimpulsgenerator generiert dabei Doppelimpulse mit einem
variablen Zeitabstand bei einer Frequenz von $\qty{1}{\kilo\Hz}$. Die Kalibration wird für neun Messwerte im Bereich von
$\qty{1}{\micro\second}$ bis $\qty{9}{\micro\second}$ durchgeführt. Dabei wird bei allen Messpunkten die gleiche Messzeit benutzt
und anschließend die absolute Zählrate verglichen.
Um die verfügbaren Kanäle des Vielkanalanalysators optimal ausnutzen zu können wird die Spannungsausgabe des
TAC so eingestellt, dass bei Impulsabständen von $\qty{9}{\micro\second}$ ungefähr der mittlere Kanal
aufgefüllt wird.\\
Nach erfolreicher Kalibration werden die Photomultiplier angeschlossen und die eigentlich Messung beginnt.
Dabei werden die individuellen Lebensdauern der Myonen für $47.18$ Stunden aufgenommen.
Aus den Daten des Histogramms wird dann die mittlere Lebensdauer der Myonen bestimmt.


