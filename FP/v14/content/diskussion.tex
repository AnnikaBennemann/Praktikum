\section{Diskussion}
\label{sec:Diskussion}

Der Verlauf des aufgenommenen Gamma-Spektrums von $\ce{^{137}Cs}$ ist wie erwartet.
Die charakteristischen Merkmale, also der Rückstreupeak, das Compton-Plateau, die Compton-Kante und der Photo-Peak sind eindeutig zu erkennen.

Der Absorptionskoeffizient vom ersten Würfel berechnet sich zu
\begin{align*}
    \mu_{\text{Alu}} &= \qty{0.192 \pm 0.527}{\per\centi\meter}.
\end{align*}
Die Aussage, dass es sich um Aluminium handelt, kann mit großer Sicherheit bestätigt werden.
Die Abweichung zum Literaturwert beträgt
\begin{align*}
   \frac{|\mu_\mathrm{Alu, gemessen}-\mu_\mathrm{Alu, Literatur}|}{\mu_\mathrm{Alu, Literatur}} \cdot 100 = \qty{4.95}{\percent}.
\end{align*}

Beim zweiten Würfel beträgt der Absorptionskoeffizient
\begin{align*}
    \mu_{\text{Würfel 2}} &= \qty{0.055 \pm 0.004}{\per\centi\meter}.
\end{align*}
Die Abweichung vom Literaturwert für Delrin beträgt
\begin{align*}
    \frac{|\mu_\mathrm{Delrin, gemessen}-\mu_\mathrm{Delrin, Literatur}|}{\mu_\mathrm{Delrin, Literatur}} \cdot 100 = \qty{53.39}{\percent}.
\end{align*}

Beim vierten Würfel betragen die Absorptionskoeffizienten
\begin{align*}
    \mu_{\mathrm{pink}} &= \qty{0.149 \pm 0.006}{\per\centi\meter} \\
    \shortintertext{und}
    \mu_{\mathrm{blau}} &= \qty{0.838 \pm 0.006}{\per\centi\meter}.
\end{align*}
Die Abweichung von $\mu_{\mathrm{pink}}$ zu $\mu_{\mathrm{Delrin, Literatur}}$ beträgt 
\begin{align*}
    \frac{|\mu_{\mathrm{pink}}-\mu_\mathrm{Delrin, Literatur}|}{\mu_\mathrm{Delrin, Literatur}} \cdot 100 = \qty{26.27}{\percent}
\end{align*}
und die Abweichung von $\mu_{\mathrm{blau}}$ zu $\mu_{\mathrm{Messing, Literatur}}$
\begin{align*}
    \frac{|\mu_{\mathrm{blau}}-\mu_\mathrm{Messing, Literatur}|}{\mu_\mathrm{Messing, Literatur}} \cdot 100 = \qty{36.93}{\percent}.
\end{align*}

Die teilweise hohen Abweichungen können aus mehreren Fehlerquellen resultieren.

Die Strahlbreite von $\qty{0.5}{\centi\meter}$ führt, gerade bei den diagonalen Projektionen, zu sehr starken Abweichungen,
da nicht nur die Diagonale durchstrahlt wird, sondern auch die daneben liegenden Elementarwürfel.
Auch bei Projektionsrichtungen die senkrecht zu der Würfeloberfläche verlaufen, kann es durch geringe Verschiebungen und das händische Ausrichten
der Würfel zu Fehlern kommen.
Zusammenfassend ist festzustellen, dass die Bestimmung der Materialien der Elementarwürfel mithilfe des gegebenen Versuchsaufbaus nur mit Einschränkungen
beziehungsweise bei einer vorgegebenen Auswahl an Materialien durchgeführt werden kann.
