
In diesem Versuch sind die beobachtbaren Effekte der Compton und Photoeffekt.
Da die Energie der Gamma-Strahlung bei $\SI{662}{\kilo\electronvolt}$ \cite{LEIFI} liegt, kommt es zu keiner Paarbildung.
Die dafür benötigte Energie muss bei $2 \cdot \SI{511}{\kilo\electronvolt}$ liegen.
In \autoref{fig:LEIFI_CS_Spektrum} ist das Spektrum von $\ce{^{137}Cs}$ zu sehen.
Der Photopeak bzw. der Full Energy Peak ist deutlich bei einer Energie von $\SI{662}{\kilo\electronvolt}$ zu erkennen. 
Die Comptonkante befindet sich bei einer Energie von ungefähr $\SI{460}{\kilo\electronvolt}$. Dies 
entspricht einem Streuwinkel von $\theta = \SI{180}{\degree}$. Der Rückstreupeak, der auch gekennzeichnet ist, 
entspricht Comptonphotonen, die nach Comptonstreuung mit z.B. der Rückwand der Präparathalterung, in den Szintillationsdetektor kommen. 