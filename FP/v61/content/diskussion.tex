\section{Diskussion}
\label{sec:Diskussion}

Die Abweichungen der experimentell bestimmten maximalen Resonatorlänge von den theoretischen Werten ist mit $\qty{11.14}{\percent}$ bei der Kombination
aus planem und konkavem Spiegel und $\qty{25.0}{\percent}$ bei der Kombination aus zwei konkaven Spiegeln relativ hoch.
Dies liegt daran, dass aufbaubedingt keine längeren Resonatorlängen einstellbar sind. Die Intensität fiel somit bei den experimentell bestimmten
Werten nicht auf Null ab und es kann keine gute Aussage über die tatsächlichen maximalen Resonatorlängen im Versuch getroffen werden.\\



\begin{align*}
    \bar{\lambda} &= \qty{590.80 \pm 48.73}{\nano\meter}.
\end{align*}
Die Abweichung zur theoretischen Wellenlänge von $\lambda_{\text{theo}}= \qty{632.80}{\nano\meter}$ beträgt

%abw_lambda = 6.64+/-7.70
%abw_M1 = 0.99+/-0.94
%abw_M2 = 0.32+/-0.93
%abw_M3 = 0.20+/-1.27
%abw_M4 = 200.19+/-0.46


Modendifferenzen bie Länge von 160 nicht realistisch, wo liegt der Fehler?