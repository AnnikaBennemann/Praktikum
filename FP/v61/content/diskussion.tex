\section{Diskussion}
\label{sec:Diskussion}

Die Abweichungen der experimentell bestimmten maximalen Resonatorlänge von den theoretischen Werten ist mit $\qty{11.14}{\percent}$ bei der Kombination
aus planem und konkavem Spiegel und $\qty{25.0}{\percent}$ bei der Kombination aus zwei konkaven Spiegeln relativ hoch.
Dies liegt daran, dass aufbaubedingt keine längeren Resonatorlängen einstellbar sind. Die Intensität fiel somit bei den experimentell bestimmten
Werten nicht auf Null ab und es kann keine gute Aussage über die tatsächlichen maximalen Resonatorlängen im Versuch getroffen werden.\\

Die gemessenen Moden in \autoref{subsec:Moden_aus} besitzen den gewünschten Verlauf der theoretischen Regressionsfunktion. Bei der TEM$_{20}$-Mode folgen die Messwerte teilweise dem gewünschten 
Verlauf, jedoch wurden nicht bis zum Modenrand gemessen, weshalb nur der zantrale Bereich der Mode abgebildet ist.\\


Die Intensitätkurve in Abhängigkeit von der Polarisation besitzte den zu erwartenden Verlauf der theoretischen Kurve.

Die gemittelte Modendifferenz zu den theoretischen Werten aus dem \autoref{sub:Longitudinale Moden} beitzen die Abweichungen
\begin{align}
    M_{1 abw} = (0.99 \pm 0.94) \si{\percent}\\
    M_{2 abw} = (0.32 \pm 0.93) \si{\percent}\\
    M_{3 abw} = (0.20 \pm 1.27) \si{\percent}\\
    M_{4 abw} = (200.19 \pm 0.46) \si{\percent}.
\end{align}
Die ersten drei Abweichungen sind sehr gering. Die letzte Abweichung ist sehr hoch und lässt auf einen Messfehler hinweisen.

Die berechnete Wellenlänge ist 
\begin{align*}
    \bar{\lambda} &= \qty{590.80 \pm 48.73}{\nano\meter}.
\end{align*}
Die theoretischen Wellenlänge von $\lambda_{\text{theo}}= \qty{632.80}{\nano\meter}$ liegt im Fehlerbreich der berechneten Wellenlänge. Die Abweichung ist recht gering mit 
$\qty{6.64 \pm 7.70}{\percent}$.

