\section{Auswertung}
\label{sec:Auswertung}

Im Folgenden werden die Mittelwerte, sowie die Fehler und weitere Berechnungen mithilfe der Pythonmodule Matplotlib\cite{matplotlib},
Uncertainties\cite{uncertainties} und Numpy\cite{numpy} durchgeführt.

\subsection{Stabilitätsbedingung des Lasers}
\label{subsec:Stabilitätsbedingung}
Die maximal mögliche Resonatorlänge $L$ lässt sich für verschiedene Spiegelkonfigurationen aus der Stabilitätsbedingung
\begin{align*}
  0 \leq g_1 g_2 &< 1 & \text{mit}& & g_i = 1 - \frac{L}{r_i}
\end{align*}
bestimmen. In diesem Versuch wird eine Kombination aus zwei konkaven Spiegeln und einem planen Spiegel und einem konkaven Spiegel.
Der Krümmungsradius der konkaven Spiegeln ist bei beiden $r_k = \qty{1400}{\milli\meter}$ und bei dem planen Spiegel gilt $r_p \to \infty$


\begin{figure}[H]
  \centering
  \includegraphics[scale=1]{build/Stabi.pdf}
  \caption {Stabilitätsparameter verschiedener Spiegelkonfigurationen in Abhängigkeit von der Resonatorlänge.}
  \label{fig:Stabi}
\end{figure}

\begin{figure}[H]
  \centering
  \includegraphics[scale=1]{build/kkpk.pdf}
  \caption {Intensitätsverteilung der Spiegelkonfigurationen in Abhängigkeit von der Resonatorlänge.}
  \label{fig:kkpk}
\end{figure}

TEM$_{00}$-Mode
%I01 &= 28.09± 0.18
%r01 &= -0.78 ± 0.03
%w01 &= 4.43 ± 0.03

\begin{figure}[H]
  \centering
  \includegraphics[scale=1]{build/TEM00.pdf}
  \caption {Intensitätsverteilung der TEM$_{00}$-Mode in Abhängigkeit vom Abstand zur Modenmitte.}
  \label{fig:TEM00}
\end{figure}

TEM$_{10}$-Mode
%I02 &= 0.92 ± 0.01
%r02 &= -0.01 ± 0.05
%w02 &= 4.58 ± 0.04

\begin{figure}[H]
  \centering
  \includegraphics[scale=1]{build/TEM10.pdf}
  \caption {Intensitätsverteilung der TEM$_{10}$-Mode in Abhängigkeit vom Abstand zur Modenmitte.}
  \label{fig:TEM10}
\end{figure}

TEM$_{20}$-Mode
%I03 &= 0.00082 ± 0.00
%r03 &= 1.03 ± 0.45
%w03 &= 0.67 ± 0.21

\begin{figure}[H]
  \centering
  \includegraphics[scale=1]{build/TEM20.pdf}
  \caption {Intensitätsverteilung der TEM$_{20}$-Mode in Abhängigkeit vom Abstand zur Modenmitte.}
  \label{fig:TEM20}
\end{figure}


Polarisation
%I0p &= 2.90 ± 0.07
%phi0p &= 88.73 ± 0.12

\begin{figure}[H]
  \centering
  \includegraphics[scale=1]{build/Pol.pdf}
  \caption {Intensitätsverteilung in Abhängigkeit vom Polarisationswinkel.}
  \label{fig:Pol}
\end{figure}