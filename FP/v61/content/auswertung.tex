\section{Auswertung}
\label{sec:Auswertung}

Im Folgenden werden die Berechnungen der Mittelwerte, sowie die Fehler und weitere Berechnungen mithilfe der Pythonmodule Matplotlib\cite{matplotlib},
Uncertainties\cite{uncertainties} und Numpy\cite{numpy} durchgeführt.

\subsection{Stabilitätsbedingung des Lasers}
\label{subsec:Stabilitätsbedingung}
Die maximal mögliche Resonatorlänge $L$ lässt sich für verschiedene Spiegelkonfigurationen aus der Stabilitätsbedingung
\begin{align*}
  0 \leq g_1 g_2 &< 1 & \text{mit}& & g_i = 1 - \frac{L}{r_i}
\end{align*}
bestimmen. In diesem Versuch wird eine Kombination aus zwei konkaven Spiegeln und eine Kombination aus einem planen Spiegel und einem konkaven
Spiegel untersucht.
Der Krümmungsradius der konkaven Spiegeln ist bei beiden $r_k = \qty{1400}{\milli\meter}$ und bei dem planen Spiegel gilt $r_p \to \infty$.

\begin{figure}[H]
  \centering
  \includegraphics[scale=1]{build/Stabi.pdf}
  \caption {Stabilitätsparameter verschiedener Spiegelkonfigurationen in Abhängigkeit von der Resonatorlänge.}
  \label{fig:Stabi}
\end{figure}

In \autoref{fig:Stabi} ist der Verlauf des Stabilitätsparameters $g_1 \cdot g_2$ in Abhängigkeit von der Resonatorlänge $L$ aufgetragen. Dabei beschreibt die
Kombination aus dem konkaven und dem planen Spiegel eine Gerade, welche die Stabilitätsbedingung $0 \leq L \leq r_k$ erfüllt. Die Kombination aus zwei konkaven Spiegeln
beschreibt eine Parabel mit der Stabilitätsbedingung $0 \leq L \leq 2 r_k$.
Die theoretischen maximalen Resonatorlängen bestimmen sich zu
\begin{align*}
  \text{plan und konkav:} && L_{pk,theo} &= 1.4 \si{\meter},\\
  \text{konkav und konkav:}&& L_{kk,theo} & = 2.8 \si{\meter}.
  \label{eqn:Länge_theo}
\end{align*}
Mit den gemessenen Werten der Intensität in Abhängigkeit von der Resonatorlänge aus \autoref{fig:kkpk} ergibt sich die experimentell
bestimmte maximale Länge zu
\begin{align}
  \text{plan und konkav:} && L_{pk,theo} &= 1.244 \si{\meter},\\
  \text{konkav und konkav:}&& L_{kk,theo} & = 2.105 \si{\meter}.
  \label{eqn:Länge_gem}
\end{align}


\begin{figure}[H]
  \centering
  \includegraphics[scale=1]{build/kkpk.pdf}
  \caption {Intensitätsverteilung der Spiegelkonfigurationen in Abhängigkeit von der Resonatorlänge.}
  \label{fig:kkpk}
\end{figure}

\subsection{Beobachtung der TEM-Moden}
\label{subsec:Moden_aus}
Im Folgenden werden die $\text{Tem}_{00}$-, die $\text{Tem}_{10}$- und die $\text{Tem}_{20}$-Mode untersucht.

\subsubsection{TEM$_{00}$-Mode}
\label{subsubsec:00Mode}
Die Messdaten der, senkrecht zur optischen Achse gemessenen, Intensitätsverteilung der Grundmode sind in \autoref{fig:TEM00} mit einer Regressionsgrade dargestellt.
Dabei hat die Regressionsgrade die Form
\begin{align*}
  I_{00}(r) = I_0 \text{exp}\Biggl(\frac{-(r-r_0)^2}{2 w^2}\Biggr).
\end{align*}

\begin{figure}[H]
  \centering
  \includegraphics[scale=1]{build/TEM00.pdf}
  \caption {Intensitätsverteilung der TEM$_{00}$-Mode in Abhängigkeit vom Abstand zur Modenmitte.}
  \label{fig:TEM00}
\end{figure}

Es ergeben sich die Parameter zu
\begin{align*}
  I_{0} &= (28.09 \pm 0.18) \si{\micro\W},\\
  r_0 &= (-0.78 \pm 0.03) \si{\milli\meter},\\
  w_0 &= (4.43 \pm 0.03) \si{\milli\meter}.
\end{align*}


\subsubsection{TEM$_{10}$-Mode}
\label{subsubsec:10Mode}

TEM$_{10}$-Mode
%I02 &= 0.92 ± 0.01
%r02 &= -0.01 ± 0.05
%w02 &= 4.58 ± 0.04
\begin{align*}
  I_{10}(r) = I_0 \cdot \frac{8(r-r_0)^2}{w^2} \cdot \text{exp}\Biggl(\frac{-(r-r_0)^2}{2 w^2}\Biggr).
\end{align*}

\begin{figure}[H]
  \centering
  \includegraphics[scale=1]{build/TEM10.pdf}
  \caption {Intensitätsverteilung der TEM$_{10}$-Mode in Abhängigkeit vom Abstand zur Modenmitte.}
  \label{fig:TEM10}
\end{figure}

\subsubsection{TEM$_{20}$-Mode}
\label{subsubsec:20Mode}

TEM$_{20}$-Mode
%I03 &= 0.00082 ± 0.00
%r03 &= 1.03 ± 0.45
%w03 &= 0.67 ± 0.21
\begin{align*}
  I_{20}(r) = I_0 \cdot \Biggl(\frac{64(r-r_0)^4}{w^4}-\frac{32(r-r_0)^2}{w^2}+4\Biggr) \cdot \text{exp}\Biggl(\frac{-(x-r_0)^2}{2 w^2}\Biggr).
\end{align*}


\begin{figure}[H]
  \centering
  \includegraphics[scale=1]{build/TEM20.pdf}
  \caption {Intensitätsverteilung der TEM$_{20}$-Mode in Abhängigkeit vom Abstand zur Modenmitte.}
  \label{fig:TEM20}
\end{figure}


Polarisation
%I0p &= 2.90 ± 0.07
%phi0p &= 88.73 ± 0.12

\begin{figure}[H]
  \centering
  \includegraphics[scale=1]{build/Pol.pdf}
  \caption {Intensitätsverteilung in Abhängigkeit vom Polarisationswinkel.}
  \label{fig:Pol}
\end{figure}






\subsection{Longitudinale Moden}
\label{sub:Longitudinale Moden}

Die Modendifferenzen zwischen den jeweils ersten vier Peakfrequenzen für vier verschiedene Resonatorlängen sind in \autoref{tab:Modendifferenzen}
aufgetragen. Zusätzlich wird der Mittelwert berechnet und auch in die Tabelle eingetragen.

\begin{table}[H]
  \centering
  \caption{Modendifferenzen zwischen den jeweils ersten vier Peakfrequenzen der Messung.}
  \label{tab:Modendifferenzen}
  \sisetup{table-format=3.0}
  \begin{tabular}{S[table-format=2.1] S S S S S@{${}\pm{}$} S[table-format=3.2]}
    \toprule
    {$L \mathbin{/} \si{\centi\meter}$} &{$\Delta f_{0,1} \mathbin{/} \si{\mega\hertz}$} &
    {$\Delta f_{1,2} \mathbin{/} \si{\mega\hertz}$} &{$\Delta f_{2,3} \mathbin{/} \si{\mega\hertz}$} &
    {$\Delta f_{3,4} \mathbin{/} \si{\mega\hertz}$} & \multicolumn{2}{c}{$\bar{\Delta f} \mathbin{/} \si{\mega\hertz}$}\\
    \midrule
       61.6  & 248 & 243 & 248 & 244 & 245.75 & 2.28\\
       72.8  & 206 & 207 & 202 & 206 & 205.25 & 1.92\\
      110.0  & 139 & 135 & 135 & 135 & 136.00 & 1.73\\
      160.0  & 281 & 282 & 281 & 281 & 281.25 & 0.43\\
    \bottomrule
  \end{tabular}
\end{table}

Die theoretischen Modendifferenzen ergeben sich nach

\begin{align*}
  \Delta f_{\text{theo}} &= \frac{c}{2 \cdot L}
\end{align*}
zu 
\begin{align*}
  L &= \SI{ 61.6}{\centi\meter}: & \Delta f_\text{theo} &= \SI{243.33}{\mega\hertz}\, , \\  
  L &= \SI{ 72.8}{\centi\meter}: & \Delta f_\text{theo} &= \SI{205.91}{\mega\hertz}\, , \\  
  L &= \SI{110.0}{\centi\meter}: & \Delta f_\text{theo} &= \SI{136.27}{\mega\hertz}\, , \\
  L &= \SI{160.0}{\centi\meter}: & \Delta f_\text{theo} &= \SI{93.69}{\mega\hertz}\, . 
\end{align*}
wobei $c$ die Lichtgeschwindigkeit ist.

Die Dopplerverbreiterung des Neon-Übergangs lässt sich mithilfe von 
\begin{align*}
  \partial f_D &= \frac{2f_0}{c} \sqrt{\frac{2RT\: \text{ln}(2)}{M}} \\
  &= \num{7.16e-7}\sqrt{\frac{\si{\gram\per\mol}}{\si{\kelvin}}} \cdot f_0 \cdot \sqrt{\frac{T}{M}}
\end{align*}
zu 
\begin{align*}
  \partial f_D &= \SI{1.314}{\giga\hertz} = \SI{1314}{\mega\hertz}
\end{align*}
berechnen, wobei $R$ die allgemeine Gaskonstante ist, die Molmasse
$M = \SI{20}{\gram\per\mol}$ beträgt und $T=\qty{300}{\kelvin}$ als Raumtemperatur angenommen wird.
$f_0$ ist die Laserfrequenz und beträgt $f_0= \frac{\lambda}{c} = \SI{4.74e5}{\giga\hertz}$.


