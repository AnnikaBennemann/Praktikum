\section{Zielsetzung}
\label{sec:Zielsetzung}
Ziel des Versuches ist es, die Funktionsweise des He-Ne-Lasers kennenzulernen.

\section{Theorie}
\label{sec:Theorie}

Laser steht für \textbf{l}ight \textbf{a}mplification by \textbf{s}timulated \textbf{e}mission of \textbf{r}adiation, also
Verstärkung des Lichts durch angeregte Emission von Strahlung.
Laser sind demnach Quellen elektromagnetischer Strahlung, die kohärentes, monochromatisches Licht mit einer hohen Intensität aussenden.

Laser bestehen grundsätzlich aus drei verschiedenen Komponenten, dem aktiven Medium, der Pumpquelle und dem Resonator.
Das aktive Medium bestimmt das Strahlungsspektrum und wird besteht beim He-Ne-Laser aus Neon.
Bei genügend Energiezufuhr durch die Energiepumpe wird im Medium eine Besetzungsinversion der Zustände und eine Lichtverstärkung erreicht.
Der Resonator dient dazu, den Lichtstrahl mehrfach durch das aktive Medium zu schicken und somit verstärkt eine induzierte Emission der Photonen
zu erzwingen.
Wenn die Verstärkung im Resonator gegenüber den Verlusten überwiegt, tritt Lasertätigkeit auf.

