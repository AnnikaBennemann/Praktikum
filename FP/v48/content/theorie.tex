\section{Zielsetzung}
\label{sec:Zielsetzung}
In diesem Versuch wird die Dipolrelaxation in Ionenkristallen untersucht. Besonderes Augenmerk liegt hierbei auf der Relaxationszeit der Dipole
$\tau(T)=\tau_0 \exp(W/k_BT)$ einer strontiumdotierten Kaliumbromidprobe in Abhängigkeit von der Temperatur.
Im Versuch wird daher die Aktivierungsenergie $W$ und die charakeristische Relaxationszeit $\tau_0$ bestimmt.

\section{Theorie}
\label{sec:Theorie}
 
Im Folgenden werden zunächst die Ionenkristalle und die enthaltenden Dipole erklärt. Anschließend wird der 
Dipolarisationsstrom mit zwei verschiedenen Ansätzen hergeleitet.

\subsection{Kristallstruktur und Dipole}
\label{subsec:KriDi}
Ionenkristalle sind Festkörper aus Ionenbindungen, wobei eine Wanderung von Elektronen nicht vorhanden
ist. Bei höheren Temperaturen kommt es jedoch zu Gitterfehlern und somit Ionenwanderung.\\
In diesem Versuch können zwei Proben verwendet werden. Die eine Probe ist Kaliumbromid (KBr), was aus einem kubischen Gitter
besteht mit Kaliumkationen und Bromidanionen. Die andere Probe ist Cäsiumiodid (CsJ), was aus einem kubischen
Gitter mit Cäsiumkationen und Jodanionen besteht. Beide Proben wurden mit Strotonium (Sr) dotiert, sodass sich
Leerstellen bildet, wie in \autoref{fig:KrisDi.png} dargestellt.
\begin{figure}[H]
    \centering
    \includegraphics[width=0.5\textwidth]{Abbildungen/KrisDi.png}
    \caption {Schematische 2D-Darstellung eines strontiumdotierten Cäsiumiodid}
    \label{fig:KrisDi.png}
\end{figure}
Dipole sind physikalische Anordnungen zweier zueinander entgegengesetzter Ladungen, wobei sich die beiden Ladungen
kompensieren und der Dipol ingesamt neutral geladen ist.
Bei einem dotierten Ionenkristall bilden ein doppelt geladenes Ion und eine Leerstelle ein Dipol und geben so die 
Richtung des Dipolmoments an.\\
Bei Raumtemperatur ist das Gesamtdipolmoment neutral, weil die verschiedenen Dipole, die durch die Dotierung 
entstehen, sich ausgleichen.\\
Dipole besitzen eine bestimmte Aktivierungsenergie $W$. Das ist eine materialabhängige Energiedifferenz, die 
überwunden werden muss, damit Leerstellendiffusion ensteht.
Außerdem besitzen Dipole eine Relaxationszeit
\begin{equation}
    \tau(T) = \tau_0 exp(\frac{W}{k_\text{B} T}),
    \label{eqn:Relaxationszeit}
\end{equation}
welche ein Maß dafür ist, wie schnell sich wieder ein elektrisches Gleichgewicht einstellt.\\
\\ 
Als Vorgriff auf die Durchführung (\ref{sec:Durchführung}) sei hier erwähnt, dass die Dipole in der Probe
mit einem E-Feld vorher ausgrichtet und anschließend in diesem Zustand eingefroren werden. Anschließend
wird die Probe wieder erwärmt und ein Strom gemessen. Dieser enstehende Strom ensteht aus der Reorientierung
der Dipole, die wieder in ihre energetisch günstigere Ausgangslage relaxieren. Dadurch dass es keine idealen Ionenkristalle
gibt, ist die Dipolrelaxation nicht der einzige Strom, der gemessen wird. Aufgrund von Gitterfehlern kommt es zu weiteren
Ladungsträgerdiffusionen, der Untergrund. \\
Der Depolarisationsstrom kann mithilfe der folgenden zwei Ansätzen hergeleitet werden.

\subsection{Herleitung des Depolarisationsstroms über die Stromdichte}
\label{Subsec:HerStromdichte}



\subsection{Polarisationsansatz}
\label{subsec:polarisation}

Da die Dipole sich bei genügend hoher Temperatur wieder statistisch ausrichten und dies mit Ladungsträgerdiffusion verbunden ist,
entsteht ein elektrisch messbarer Strom.
Der Strom wird Depolarisationsstrom genannt und ist gleich der Änderungsrate der 
Polarisation $P(t)$, also dem Gesamtdipolmoment pro Volumeneinheit.
\begin{equation}
    I(T) = - \frac{\text{d}P(t)}{\text{d}t}.
\end{equation}
Dies lässt sich auch mit der Relaxationszeit darstellen und ergibt somit die Differentialgleichung
\begin{equation}
    \frac{\text{d} P(t)}{\text{d} t} = \frac{P(t)}{\tau(T)}.
    \label{eqn:diff}
\end{equation}
Die Lösung der Gleichung mithilfe der Separation der Variablen ergibt
\begin{equation}
    P(t) = P_0 \text{exp}\left(-\int_0^t\frac{\text{d}t}{\tau(T)}\right).
\end{equation}
Ableiten dieser Lösung nach der Zeit ergibt wieder die Änderung der Polarisation und somit den 
Depolarisationsstrom.
\begin{equation}
    I(T) = \frac{P_0}{\tau(T)} \text{exp}\left(-\int_0^t\frac{\text{d}t}{\tau(T)}\right) \quad.
    \end{equation}
Mit der konstanten Heizrate
\begin{equation}
    b := \frac{\text{d}T}{\text{d}t} = const
\end{equation}
lässt sich der Depolarisationsstrom nun als 
\begin{equation}
    I(T) = \frac{P_0}{\tau(T)} \text{exp}\left(-\frac{1}{b\tau_0}\int_{T_0}^T\frac{\text{d}T'}{\tau(T')}\right)
\end{equation}
ausdrücken, da $T$ eine lineare Funktion in Abhängigkeit von $t$ ist.

\subsection{Stromdichtenansatz}

Die Zeit in der das elektrische Feld eingeschaltet ist muss groß gegen die Relaxationszeit $\tau(T)$ sein, 
damit sich ein Bruchteil $y(T)$ der Dipole in Feldrichtung einstellt.
Zudem muss $pE << k_\text{B}T$ gelten.
$y(T)$ ist dann gegeben durch
\begin{align*}
    y(T)= \frac{pE}{3 k_\text{B}T}
\end{align*}

Die Depolarisationsstromdichte $j(T)$ setzt sich zusammen aus $y(T)$, sowie dem Dipolmoment $p$ und der Zahl der pro Zeit- und Volumeneinheit 
relaxierenden Dipole $\frac{\text{d}N}{\text{d}t}$:
\begin{align}
    j(T) = y(T) \cdot p \cdot \frac{\text{d}N}{\text{d}t}.
    \label{eqn:stromdichte}
\end{align}

Analog zu \autoref{subsec:polarisation} lässt sich auch hier eine Differentialgleichung aufstellen und lösen.
Die Änderung der Dipole entspricht
\begin{align}
    \frac{\text{d} N}{\text{d} t} = -\frac{N}{\tau(T)}
\end{align}
und die Lösung dieser Differentialgleichung lautet somit 
\begin{align}
    N = N_\text{P} \text{exp}\left(-\frac{1}{b}\int_{T_0}^T\frac{\text{d}T'}{\tau(T')}\right)
\end{align}

Einsetzen der gesamten Gleichungen in \autoref{eqn:stromdichte} liefert mit Nutzung von \autoref{eqn:Relaxationszeit} die Stromdichte
\begin{equation}
  j(T) = \frac{p^2E}{3k_\text{B}T}\frac{N_\text{P}}{\tau_0}\text{exp}\left(-\frac{1}{b\tau_0}\int_{T_0}^T\text{exp}(-W/ k_ \text{B}T')\text{d}T'\right)\text{exp}\left(-\frac{W}{k_\text{B}T}\right).
 \label{eqn:lang}
\end{equation}

Für kleine Temperaturen $T$ ist die Aktivierungsenergie $W$ groß gegenüber der Energie $k_\text{B}T$ und der Temperaturdifferenz $T-T_0$ und das Integral in \autoref{eqn:lang} wird zu
\begin{align}
    \int_{T_0}^T\frac{\text{d}T'}{\tau(T')} \approx 0.
\end{align}
Somit ergibt sich der Strom zu
\begin{equation}
    j(T) = \frac{p^2E}{3k_\text{B}T}\frac{N_\text{P}}{\tau_0} \text{exp}\left(-\frac{W}{k_\text{B}T}\right).
    \label{eqn:kurz}
\end{equation}


\subsection{Berechnung der Aktivierungsenergie W}

\subsubsection{Bestimmung mit Hilfe des Maximums}

        Trotz konstanter Heizrate gibt es bei einer gewissen Temperatur $T_\text{max}$ aufgrund der endlichen Anzahl der Dipole im Kristall ein Maximum des Depolarisationsstroms.
        Dies kann genutzt werden um charakteristische Eigenschaften des Kristalles zu bestimmen.        
       	
        Aus \autoref{eqn:kurz} kann eine Geradengleichung der Form
        \begin{equation}
            \text{ln}(j(T)) = \left(\frac{p^2EN_\text{P}}{3k_\text{B}T\tau_0}\right) - \frac{W}{k_\text{B}} \frac{1}{T} \quad 
        \end{equation}
        erstellt werden.
        Die Steigung $m$ dieser Geraden ist also $\frac{W}{k_\text{B}}$ beziehungsweise entspricht die Aktivierungsenergie $W$
        \begin{equation}
            W = m \cdot k_\text{B}.
            \label{eqn:W}
        \end{equation}
        Diese Gleichung gilt am besten für den Bereich vor dem Maximum und wird daher in diesem Bereich zur Auswertung angewandt.			

\subsubsection{Verwendung des gesamten Kurvenverlaufs}

        Ein weiterer Ausdruck entsteht wenn der Verlauf der Gesamtpolarisation $P$ betrachtet wird:
        \begin{equation}
            \frac{dP}{dt} = \frac{P(t)}{\tau(T(t))}.
        \end{equation}
        Umstellen und mit $\frac{dT}{dT}$ erweitern liefert:
        \begin{equation}
            \tau(T) = P(t) \cdot \frac{\text{d}T}{\frac{dP}{dt}\text{d}T}
        \end{equation}
        Auch hier ist die Änderung der Temperatur gleich der Heizrate $b$:
        \begin{equation}
            \tau(T) = \frac{P(t)}{b}\frac{dT}{dP}.
        \end{equation}
        Durch Erweitern mit $\frac{dt}{dt}$ ergibt sich
        \begin{equation}
            \frac{P(t)}{b} \frac{\frac{dT}{dt}}{\frac{dP}{dt}}
        \end{equation}
        Die Änderung Polarisation $\frac{dP}{dt}$ entspricht dem Strom $I(T)$ und mit $P = \int \text{d}P$ ist dann
        \begin{equation}
            \tau(T) = \frac{\int \frac{dP}{dt}\text{d}T}{I(T)b} \quad .
        \end{equation}
        Der Dipolarisationsstrom kann hier erneut gegen die Änderung der Polarisation vertauscht werden
        \begin{equation}
            \tau(T) = \frac{\int_T^\infty I(T')\text{d}T'}{I(T)b}
        \end{equation}
        und somit ergibt sich für die Aktivierungsenergie $W$ nach Einsetzen von \autoref{eqn:Relaxationszeit}
        \begin{equation}
            W = k_\text{B}T\;\text{ln}\left( \frac{\int_T^\infty I(T')\text{d}T'}{I(T)b\tau_0}\right).
            \label{eqn:int}
        \end{equation}
        Die obere Grenze verschwindet in der Praxis, da bei hohen Temperaturen keine Dipole mehr vorhanden sind die noch relaxieren können.

\subsection{Berechnung der charakteristischen Relaxationszeit}

Die charakteristische Relaxationszeit $\tau_0$ kann aus dem Maximum der Temperaturkurve indirekt gewonnen werden.
Dazu wird Gleichung \ref{eqn:long} nach der Temperatur abgeleitet um eine Beziehnung zwischen $T_text{max}$ und $\tau(T_\text{max})$ herzustellen.

\begin{equation}
    \frac{dj(T)}{dT} \approx \frac{1}{\tau_0} \left( - \frac{1}{b\tau_0}
    \int_{T_0}^T \text{exp}\left(\frac{W}{k_\text{B}T}\right)\text{d}T' - 
    \frac{W}{k_\text{B}T}\right) \cdot \left( \frac{W}{k_\text{B}T^2} - 
    \frac{1}{b\tau_0} 
    \text{exp}\left(-\frac{W}{k_\text{B}T}\right)\right).
\end{equation}
Die abgeleitete Gleichung lässt sich nun am Maximum nach der charakteristischen Relaxationszeit umstellen:
\begin{equation}
    \tau_0 = \tau(T_\text{max})\text{exp}\left(- \frac{W}{k_\text{B}T_\text{max}}\right) = \frac{k_\text{B}T^2_\text{max}}{Wb}\text{exp}\left(-\frac{W}{k_\text{B}T_\text{max}}\right)
    \label{eqn:taumaxsource}
\end{equation}
