\section{Zielsetzung}
\label{sec:Zielsetzung}
In diesem Versuch wird die Dipolrelaxation in Ionenkristallen untersucht. Besonderes Augenmerk liegt hierbei auf der Relaxationszeit der Dipole
$\tau(T)=\tau_0 \exp(W/k_BT)$ einer strontiumdotierten Kaliumbromidprobe in Abhängigkeit von der Temperatur.
Im Versuch wird daher die Aktivierungsenergie $W$ und die charakeristische Relaxationszeit $\tau_0$ bestimmt.

\section{Theorie}
\label{sec:Theorie}
 
Im Folgendem werden zunächst die Ionenkristalle erklärt und die enthaltenden Dipole. Anschließend wird der 
Dipolarisationsstrom mit zwei verschiedenen Ansätzen hergeleitet.

\subsection{Kristallstruktur und Dipole}
\label{subsec:KriDi}
Ionenkristalle sind Festkörper aus Ionenbindungen, wobei eine Wanderung von Elektktronen nicht vorhanden
ist. Bei höheren Temperaturen kommt es jedoch zu Gitterfehlern und somit Ionenwanderung.\\
In diesem Versuch können zwei Proben verwendet werden. Die eine Probe ist Kaliumbromid (KBr), was aus einem kubischen Gitter
besteht mit Kaliumkationen und Bromidanionen. Die andere Probe ist Cäsiumiodid (CsJ), was aus einem kubischen
Gitter mit Cäsiumkationen und Jodanionen besteht. Beide Proben wurden mit Strotonium (Sr) dotiert, sodass sich
Leerstellen bildet, wie in \autoref{fig:KrisDi.png} dargestellt.
\begin{figure}[H]
    \centering
    \includegraphics[width=0.5\textwidth]{Abbildungen/KrisDi.png}
    \caption {Schematische 2D-Darstellung eines strontiumdotierten Cäsiumiodid}
    \label{fig:KrisDi.png}
\end{figure}
Dipole sind physikalische Anordnungen zweier zueinander entgegengesetzter Ladungen, wobei sich die beiden Ladungen
kompensieren und der Dipol ingesamt neutral geladen ist.
Bei einem dotierten Ionenkristall bilden ein doppelt geladenes Ion und eine Leerstelle ein Dipol und geben so die 
Richtung des Dipolmoments an.\\
Bei Raumtemperatur ist das Gesamtdipolmoment neutral, weil die verschiedenen Dipole, die durch die Dotierung 
entstehen, sich ausgleichen.\\
Dipole besitzen eine bestimmte Aktivierungsenergie $W$. Das ist eine materialabhängige Energiedifferenz, die 
überwunden werden muss, damit Leerstellendiffusion ensteht.
Außerdem besitzen Dipole eine Relaxationszeit
\begin{equation}
    \tau(T) = \tau_0 exp(\frac{W}{k T}),
    \label{eqn:Relaxationszeit}
\end{equation}
welche ein Maß dafür ist, wie schnell sich wieder ein elektrisches Gleichgewicht einstellt.\\
\\ 
Als Vorgriff auf die Durchführung (\ref{sec:Durchführung}) sei hier erwähnt, dass die Dipole in der Probe
mit einem E-Feld vorher ausgrichtet und anschließend in diesem Zustand eingefroren werden. Anschließend
wird die Probe wieder erwärmt und ein Strom gemessen. Dieser enstehende Strom ensteht aus der Reorientierung
der Dipole, die wieder in ihre energetisch günstigere Ausgangslage relaxieren. Dadurch dass es keine idealen Ionenkristalle
gibt, ist die Dipolrelaxation nicht der einzige Strom, der gemessen wird. Aufgrund von Gitterfehlern kommt es zu weiteren
Ladungsträgerdiffusionen, der Untergrund. \\
Der Depolarisationsstrom kann mit den folgenden zwei Ansätzen hergeleitet werden.

\subsection{Herleitung des Depolarisationsstroms über die Stromdichte}
\label{Subsec:HerStromdichte}