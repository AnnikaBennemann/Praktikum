\subsection{Polarisationsansatz}

Die Dipole sind in ihrem angeregtem Zustand alle in dieselbe Richtung ausgerichtet.
Dies bedeutet auch, dass durch die Relaxation im Mittel alle positiven Ladungen in 
die gleiche Richtung verschoben werden und somit einen Strom erzeugen.
Der Strom wird Depolarisationsstrom genannt und ist gleich der Änderungsrate der 
gesamten Polarisation $P(t)$:
\begin{equation}
    I(T) = - \frac{\text{d}P(t)}{\text{d}t}.
\end{equation}
Die Änderungsrate lässt sich auch mittels der Relaxationszeit darstellen und ergibt somit 
folgende fracferentialgleichung:
\begin{equation}
    \frac{\text{d} P(t)}{\text{d} t} = \frac{P(t)}{\tau(T)}.
    \label{eqn:fracf}
\end{equation}
Das Lösen der fracferentialgleichung mittels Seperation der Variabeln ergibt dann:
\begin{equation}
    P(t) = P_0 \text{exp}\left(- \frac{t}{\tau(T)} \right).
\end{equation}
Ableiten dieser Lösung ergibt wieder die Änderung der Polarisation und somit den 
Depolarisationsstrom.
\begin{equation}
    I(T) = \frac{P_0}{\tau(T)} \text{exp}\left(-\frac{t}{\tau(T)}\right)
\end{equation}
Hier gibt $t$ die Zeit an, die benötigt wurde um $T$ zu erreichen.
Sie lässt sich auch als Integral schreiben:
\begin{equation}
    I(T) = \frac{P_0}{\tau(T)} \text{exp}\left(-\int_0^t\frac{\text{d}t}{\tau(T)}\right) \quad.
    \end{equation}
Mittels einer konstanten Heizrate
\begin{equation}
    b := \frac{T}{t} = const
\end{equation}
lässt sich der Depolarisationsstrom nun als 
\begin{equation}
    I(T) = \frac{P_0}{\tau(T)} \text{exp}\left(\frac{-1}{b\tau_0}\int_{T_0}^T\frac{\text{d}T'}{\tau(T')}\right)
\end{equation}
ausdrücken.

\subsection{Stromdichtenansatz}

Ein weiterer Ansatz für den Depolarisationsstrom ergibt, sich mittels der Debeye-Polarisation
\begin{equation}
    \bar{P}(T) = \frac{N}{N_V}\frac{p^2E}{3k_\text{B}T} \, ,
\end{equation}	
mit dem Dipolmoment $p$, der elektrischen Feldstärke $E$, der Temperatur $T$ und der Dipoldichte $N_V$.
Die Änderung der Anzahl der Dipole lässt sich auch hier wieder über die Relaxationszeit ausdrücken:
\begin{equation}
    \frac{N(T)}{t} =- \frac{N}{\tau(T)}.
\end{equation}
Analog zum vorherigen Kapitel ergibt sich die Lösung der fracferentialgleichung zu:
\begin{equation}
    N = N_\text{P} \text{exp}\left( \frac{-1}{b}\int_{T_0}^T \frac{\text{d}T'}{\tau(T')}\right) \quad .
\end{equation}
Weiterhin gilt
\begin{align}
    I(T) = \bar{P}(T)\frac{N}{t}& &\text{und} & & I(T) = -\bar{P}(T) \frac{N}{\tau(T)}
\end{align}
Zusammensetzten aller dieser Terme ergibt dann einen Ausdruck für den Depolarisationsstrom
\begin{equation}
    I(T) =\frac{p^2E}{3k_\text{B}T}\frac{N_\text{P}}{\tau_0}\text{exp}\left(\frac{-1}{b\tau_0}\int_{T_0}^T\frac{\text{d}T'}{\tau(T')}\right)\text{exp}\left(-\frac{W}{k_\text{B}T}\right).
    \label{eqn:long}
\end{equation}		

\subsection{Berechnung der Aktivierungsenergie W}

\subsubsection{Bestimmung mit Hilfe des Maximums}

        Aufgrund der endlichen Anzahl an Dipolen entsteht trotz konstanter Heizrate, in der Theorie und im Experiment, bei einer gewissen Temperatur $T_\text{max}$ ein Maximum des Depolarisationsstroms.
        Dies kann dazu genutzt werden um charakteristische Eigenschaften des Kristalles zu bestimmen.
        Wird angenommen, dass die Aktivierungsenergie $W$ groß gegenüber der Energie $k_\text{B}T$ und der Temperaturfracferenz $T-T_0$ ist, so wird das Integral in Gleichung \ref{eqn:long} zu
        \begin{equation}
            \int_{T_0}^T\frac{\text{d}T'}{\tau(T')} \approx 0.
        \end{equation}
        Somit ergibt sich dich der Strom dann zu
        \begin{equation}
            I(T) = \frac{P^2E}{3k_\text{B}T}\frac{N_\text{P}}{\tau_0} \text{exp}\left(-\frac{W}{k_\text{B}T}\right).
        \end{equation}	
        Mittes des Logarithmus entsteht hieraus eine Geradengleichung der Form
        \begin{equation}
            \text{ln}(I(T)) = \left(\frac{P^2EN_\text{P}}{3k_\text{B}T\tau_0}\right) - \frac{W}{k_\text{B}} \frac{1}{T} \quad .
        \end{equation}
        Die Steigung $m$ dieser Geraden ist also $\frac{W}{k_\text{B}}$ oder
        \begin{equation}
            W = m \cdot k_\text{B}.
            \label{eqn:W}
        \end{equation}
        Dieses analytische Ergebnis gilt am besten für den Bereich vor dem Maximum und wird daher in diesem Bereich ausgewertet.			

\subsubsection{Verwendung des gesamten Kurvenverlaufs}

        Ein weiterer Ausdruck entsteht wenn der Verlauf der Gesamtpolarisation $P(T)$ betrachtet wird:
        \begin{equation}
            \frac{P}{t} = \frac{P(t)}{\tau(T(t))}.
        \end{equation}
        Umstellen und mit $\frac{T}{T}$ erweitern liefert:
        \begin{equation}
            \tau(T) = P(T) \cdot \frac{\text{d}T}{\frac{P}{t}\text{d}T}
        \end{equation}
        Auch hier ist die Änderung der Temperatur gleich der Heizrate $b$:
        \begin{equation}
            \tau(T) = \frac{P(t)}{b}\frac{T}{P}.
        \end{equation}
        Durch Erweitern mit $\frac{t}{t}$ ergibt sich
        \begin{equation}
            \frac{P(t)}{b} \frac{\frac{T}{t}}{\frac{P}{t}}
        \end{equation}
        Die Änderung Polarisation entspricht dem Strom und mit $P = \int \text{d}P$ ist dann
        \begin{equation}
            \tau(T) = \frac{\int \frac{P}{t}\text{d}T}{I(T)b} \quad .
        \end{equation}
        Hier kann nun erneut der Depolarisationsstrom identifiziert und eingesetzt werden:
        \begin{equation}
            \tau(T) = \frac{\int_T^\infty I(T')\text{d}T'}{I(T)b}
        \end{equation}
        und somit ergibt sich für die Aktivierungsenergie $W$:
        \begin{equation}
            W = k_\text{B}T\;\text{ln}\left( \frac{\int_T^\infty I(T')\text{d}T'}{I(T)b\tau_0}\right)
            \label{eqn:int}
        \end{equation}
        Die obere Grenze verschwindet in der Praxis, da bei hohen Temperaturen keine Dipole mehr vorhanden sind die noch relaxieren können.

\subsection{Berechnung der charakteristischen Relaxationszeit}
        
An dem Maximum der Stomstärke, bei der Temperatur $T_\text{max}$, ist dessen Ableitung verschwindend. Dies lässt sich nutzen um die charakteristische Relaxationszeit zu bestimmen.
Dazu wird Gleichung \ref{eqn:long} nach der Temperatur abgeleitet:
\begin{equation}
    \frac{I(T)}{T} \approx \frac{1}{\tau_0} \left( - \frac{1}{b\tau_0}
    \int_{T_0}^T \text{exp}\left(\frac{W}{k_\text{B}T}\right)\text{d}T' - 
    \frac{W}{k_\text{B}T}\right) \cdot \left( \frac{W}{k_\text{B}T^2} - 
    \frac{1}{b\tau_0} 
    \text{exp}\left(-\frac{W}{k_\text{B}T}\right)\right).
\end{equation}
Dies lässt sich nun am Maximum nach der charakteristischen Relaxationszeit umstellen:
\begin{equation}
    \tau_0 = \tau(T_\text{max})\text{exp}\left(- \frac{W}{k_\text{B}T_\text{max}}\right) = \frac{k_\text{B}T^2_\text{max}}{Wb}\text{exp}\left(-\frac{W}{k_\text{B}T_\text{max}}\right)
    \label{eqn:taumaxsource}
\end{equation}