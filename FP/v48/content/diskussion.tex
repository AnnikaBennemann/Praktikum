\section{Diskussion}
\label{sec:Diskussion}
Die berechneten Werte für die Aktivierungsenergie aus der Anlaufkurve sind
\begin{align*}
    W_{1,\text{Anlauf}} &= (0.50 \pm 0.06) \unit{\electronvolt}\\
    W_{2,\text{Anlauf}} &= (0.75 \pm 0.04) \unit{\electronvolt}\\
\end{align*}
und die berechnet Aktivierungsenergie aus dem Integral ergibt sich zu
\begin{align*}
    W_{1,\text{Integral}} &= (0.47 \pm 0.04) \unit{\electronvolt}\\
    W_{2,\text{Integral}} &= (0.51 \pm 0.04)\unit{\electronvolt}.\\
\end{align*} 
Die Abweichungen zu den Theoriewerten (\cite{RMucillo}) 
\begin{align*}
    W_\text{lit}&= \qty{0.66}{\electronvolt}\\
    \tau_{0,\text{lit}}&= \qty{4e-14}{\second}\\ 
\end{align*}
lässt sich mit 
\begin{align*}
    abw=\frac{|x_\mathrm{gemessen}-x_\mathrm{theorie}|}{x_\mathrm{theorie}}\cdot 100 \label{eqn:abweich}
\end{align*}
bestimmen.
Der Theoriewert der Aktivierungsenergie liegt nicht im Fehlerbereich der experimentell
bestimmten Werte. Die größte Abweichung liegt bei $\qty{28.78}{\percent}$ für den Wert
von der ersten Messung, die über das Integral berechnet wurde. Die kleinste Abweichung
ist bei der zweiten Messung, die über die Anlaufkurve bestimmt wurde, mit $\qty{13.63}{\percent}$.
Sowohl die höchste als auch die kleinste Abweichung sind nicht so prägnant, sodass beide
Berechnungsmethoden als relativ genau angenommen werden können. 
Die berechneten Relaxationszeiten sind
\begin{align*}
    \tau_{01,\text{Anlauf}} &= \qty{1.26 (15)e-18}{\second}\\ 
    \tau_{02,\text{Anlauf}} &= \qty{7.80 (50)e-19}{\second}\\
    \tau_{01,\text{Integral}} &=\qty{1.36 (11)e-18}{\second}\\
    \tau_{02,\text{Integral}} &=\qty{1.15 (0.10)e-18}{\second}.\\
\end{align*}
Auch hier lässt sich die Abweichungen bestimmen. Dabei weichen diese Werte um mehrere
Größenordnungen ab. Dies lässt sich unteranderem daher erklären, dass die Relaxationszeit abhängig von
der Aktivierungsenergie und der Heizrate ist und es somit zu einer größeren Fehlerfortpflanzung kommen kann.
Weitere Fehlerquellen für die Abweichungen sind zum Einen menschlich bedingte Messunsicherheiten, zum Beispiel
das erschwerte Ablesen und das Einstellen einer konstanten Heizrate. Zum Anderen ist das Picoamperemeter
sehr empfindlich gegenüber Erschütterungen und Bewegungen.
Eine weitere Fehlerquelle sind starke Schwankungen bei dem Thermometer und eine ungewollte Veränderung 
des Vakkuums im Versuchsverlauf. Zudem weisen die Werte der Kurve des erstens Maximums einen nicht zu erwartenden Verlauf auf.
Es gibt einen Spike in der Anlaufkurve des Maximums, der für die Berechnung der Aktivierungsenergie und der
charakteristischen Relaxationszeit nicht berücksichtigt wurde, jedoch zu verfälschten Messwerten führen kann.
Bei beiden Messung tritt diese Unregelmäßigkeit im selben Temperaturbereich auf, was darauf schließen lässt,
dass es sich um einen systematischen Fehler handelt.