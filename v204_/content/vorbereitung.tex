\section{Vorbereitung}
\label{sec:Vorbereitung}

Vor der Durchführung des Versuchs sollten die Literaturwerte von Messing, Aluminium und Edelstahl herausgesucht werden.
In der folgenden Tabelle werden somit die Literaturwerte von der Dichte $\rho$, der spezifischen Wärme $c$ und der Wärmeleitfähigkeit $\kappa$ aufgelistet.
Außerdem wird die Wellenlänge mit den vorhandenen Literaturwerten nach \autoref{eqn:welle} bestimmt.

\begin{table}
    \centering
    \caption{Literaturwerte von Messing, Aluminium und Edelstahl.}
    \label{tab:Vorbereitung} 
    \begin{tabular}{
        l
        c
        c
        c
        c
        }
    \toprule
    & {$\rho (10^3\si{kg}\mathbin{/}\si{m^3})$} &{$c (\si{J}\mathbin{/}\si{kg}\si{K})$} & {$\kappa (\si{W}\mathbin{/}\si{mK}$)} & {$\lambda (\si{m})$}\\
    \midrule
        Aluminium & 2,7 & 895 & 237 & 0,186\\
        Messing & 8,4 & 380 & 113 & 0,320\\
        Edelstahl & 7,89 & 450 & 20 & 0,125 \\
        Wasser & 0,998 & 4,182 & 0,6 \\
    \bottomrule
    \cite[378,381]{PhyPrak} \cite{EdelWäLeit} 
    \end{tabular}
\end{table}