\section{Durchführung}
\label{sec:Durchführung}


\subsection{Aufbau}
Der Versuchsaufbau besteht aus einer Grundplatte mit vier rechteckigen Stäben, die an einer Seite von einem Peltier-Element simultan geheizt oder gekühlt werden.
Die Stäbe sind aus drei verschiedenen Materialien:  Aluminium, Edelstahl und zweimal Messing, mit verschiedenen Durchmessern.
Zusätzlich sind an jedem Stab zwei Thermoelemente, welche die Temperatur an verschiedenen Stellen der Stäbe messen \autoref{fig:Versuchsaufbau}.
Die Thermoelemente sind verbunden mit einem GLX Datenlogger \autoref{fig:GLX}, welcher die Temperaturen aufnimmt und eine Tabelle überführt.
Zuletzt gibt es auch eine Spannungsquelle, welche bei der statischen Mode eine Betriebspannung von $5\si{V}$ auf das Heizelement überträgt. 
Bei der der dynamischen Mode wird sie auf $8\si{V}$ eingestellt. Bei beidem wird der Strom auf Maximal gestellt.

\begin{figure}[H]
    \centering
    \includegraphics{content/Abb_1.pdf}
    \caption{Grundplatte mit Aluminium, Edelstahl und zweimal Messing\cite[3]{V204}}
    \label{fig:Versuchsaufbau}
\end{figure}

\subsection{Statische Methode}
An allen acht Thermoelementen wird der Temperaturverlauf in Abhängigkeit des der Zeit gemessen.
Dafür wird die Abtastrate beim GLX auf $\Delta t_{GLX} = 5\si{s}$ Sekunden gestellt.
Es wird solange gemessen bis das Thermoelement T7 $45°$ anzeigt.
Während des Heizvorgangs werden über die Stäbe eine Isolierung gezogen, damit der Wärmeaustausch mit der Umgebung verringert wird.
Nach der Messung müssen die Stäbe wieder gekühlt werden, sodass deren Temperatur maximal $30$ betragen.

\subsection{Dynamische Methode}
Ein andere Name für dieses Methode ist die Angström-Messverfahren.
Dabei werden die Probenstäbe periodisch geheizt.
Die Abtastrate wird vorher auf $\Delta t_{GLX} = 2\si{s}$ geändert.\\
Die erste Messung ist über eine Periode von $80\si{s}$, wobei die ersten $40\si{s}$ geheizt und die letzten $40\si{s}$ gekühlt werden.
Während gekühlt wird muss das Peltier-Element auf "COOL" gestellt werden und die Wärmeisolatoren müssen abgenommen werden.
Diese Messung geht über $10$ Perioden.\\
Die zweite Messung wird analog durchgeführt. 
Die Periode beträgt jedoch nun $200 \si{s}$ und die Messung endet, wenn eines der Thermoelement $80°$ erreicht.

\begin{figure}[H]
    \centering
    \includegraphics{content/Abb_2.pdf}
    \caption{Xplore GLX\cite[5]{V204}}
    \label{fig:GLX}
\end{figure}

T1 Messing dick fern
T2 Messing dick nah
T3 Messing dünn nah 
T4 Messing dünn fern
T5 Aluminium fern
T6 Aluminium nah
T7 Edelstahl nah
T8 Edelstahl fern