\section{Diskussion}
\label{sec:Diskussion}

\subsection{statische Messung}
Temperaturverläufe ferne Thermoelemente vergleichen
Gemeinsamkeiten, Unterschiede

Beste Wärmeleitung Temperatur nach 650s

Wärmeströme vergleichen mit Literatur

Temperaturdifferenzen vergleichen 

\subsection{dynamische Messung}

\begin{table}[H]
    \centering
	\caption{Vergleich der berechneten Wärmeleitfähigkeiten mit Literaturwerten}
	\label{tab:Vlit}
    \sisetup{table-format=3.2}
    \begin{tabular}{l
        S@{${}\pm{}$}
        S
        S[table-format=3.0]
        S@{${}\pm{}$}
        S[table-format=2.2]}
		\toprule
		\multicolumn{1}{l}{Material}&\multicolumn{2}{p{3cm}}{berechneter Wert$\,[\si{\watt\per\meter\per\kelvin}$]} &\multicolumn{1}{p{3cm}}{ Literaturwert $[\si{\watt\per\meter\per\kelvin}$]} & \multicolumn{2}{p{3cm}}{Abweichung [$\si{\percent}$]}\\
		\midrule
		Messing &133.95 &63.92&113&18.54 &56.57\\
        Aluminium&250.72 & 140.93&113&5.79 & 59.46\\
        Edelstahl&10.97&1.52&113&45.17 & 7.60\\
		\bottomrule
	\end{tabular}
\end{table}	

In \autoref{tab:Vlit} sind die aus den Messwerten berechneten Wärmeleitfähigkeiten und die zugehörigen Literaturwerte, sowie die Abweichung davon aufgelistet.
Die Werte von Messing und Aluminium wurden der Messung nach Angström-Methode mit einer Periode von $\qty{80}{\second}$ entnommen, bei Edelstahl wurden die Werte aus der Messung
mit einer Periode von $\qty{200}{\second}$ genommen.

kappa edelstahl literaturwert nicht in Fehlerbereich, allerdings edelstahl verschiedene Zusammensetzungen mit unterschiedlichen Eigenschaften,
daher kann man nicht unbedingt davon ausgehen, dass der Literaturwert dieselbe Stahlzusammensetzung beschreibt.


\begin{table}
    \centering
    \caption{Wellenlängen der Temperaturwellen.}
    \label{tab:die}
    \sisetup{table-format=1.3}
    \begin{tabular}{l
        S@{${}\pm{}$}
        S
        S
        S[table-format=2.2]@{${}\pm{}$}
        S[table-format=2.2]}
        \toprule
        \multicolumn{1}{l}{Material}&\multicolumn{2}{p{3cm}}{berechneter Wert$\, [\si{\m}]$} &\multicolumn{1}{p{3cm}}{ Literaturwert $ [\si{\m}]$} & \multicolumn{2}{p{3cm}}{Abweichung [$\si{\percent}$]}\\
        \midrule
        Messing & 0.203 & 0.048 & 0.186 & 8.88 & 25.98\\
        Aluminium & 0.329 & 0.093 & 0.320 & 2.85 & 28.91\\
        Edelstahl & 0.093 & 0.006 & 0.125 & 25.95 & 5.13\\
        \bottomrule
    \end{tabular}
\end{table}

Die Frequenz der Welle ist der Kehrwert der Periodendauer. 




