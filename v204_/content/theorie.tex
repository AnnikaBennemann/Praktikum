\section{Theorie}
\label{sec:Theorie}
 
Ziel dieses Versuches ist es die Wärmeleitung von Aluminium, Messing und Edelstahl zu untersuchen.

\subsection{Allgemein}

Existiert ein Temperaturungleichgewicht, entsteht ein Wärmetransport entlang des Temperaturgefälles. 
Dies kann entweder durch Konvektion, Wärmeleitung oder Wärmestrahlung passieren.
In diesem Versuch wird sich auf die Wärmeleitung beschränkt.\\
In festen Körpern erfolgt der Wärmetransport über Phononen und frei bewegliche Elektronen, wobei der Gitterbeitrag vernachlässigbar ist \cite[1]{V204}.
Die Wärmemenge lässt sich bestimmen durch:
\begin{equation}
    dQ = -\kappa A \frac{\partial T}{\partial x} dt
    \label{eqn:Wärmemenge}
\end{equation}
wobei $\kappa$ die Wärmeleitfähigkeit ist.
Mit der Wärmestromdichte
\begin{equation}
    j_{\omega} = -\kappa\frac{\partial T}{\partial x}
    \label{eqn:Wärmestromdichte}
\end{equation}
und der Kontinnuitätsgleichung kann eine eindimensionale Wärmeleitungsgleichung gebildet werden.
\begin{equation}
    \frac{\partial T}{\partial t} = \frac{\kappa}{\rho c} \frac{\partial ^2 T}{\partial x^2}
    \label{eqn:1D-Wärmeleitungsgleichung}
\end{equation}
Diese gibt den räumlichen- und zeitlichen Verlauf der Temperaturverteilung an.
$\sigma_t = \frac{\kappa}{\rho c}$ ist die Temperaturleitfähigkeit des Materials.
Sie gibt die "Geschwindigkeit" der Ausgleichung des Temperaturgefälles an.
Die Temperatur kann als Funktion mit zeitlicher Abhängigkeit beschrieben werden.
Bei einem periodischen Temperaturwechsel bildet sich eine räumliche und zeitliche Temperaturwelle, welche eine Funktion der Form
\begin{equation}
   T(x,t) = T_{max} e^{-\sqrt{\frac{\omega \rho c}{2 \kappa}}x} cos(\omega t - \sqrt{\frac{\omega \rho c}{2 \kappa}})
   \label{eqn:periTempWel}
\end{equation} 
annimmt.
Aus der Phasengeschwindigkeit
\begin{equation}
    \upsilon = \frac{\omega}{k} = \sqrt{\frac{2\kappa \omega}{\rho c}}
    \label{eqn:upsilon}\\
\end{equation} 
lässt sich mit der Dämpfung die Formel für die Wärmeleitfähigkeit $\kappa$ aufstellen. 
Die Dämpfung lässt sich aus den Amplitudenverhältnissen $\frac{dA_{nah}}{dA_{fern}}$ berechnen, wobei diese Amplituden an zwei Thermoelementen mit dem Abstand $\Delta x$ gemessen werden.
\begin{equation}
    \kappa =\frac{\rho c (\Delta x)^2}{2\Delta t \: ln(A_{nah}/A_{fern})}\label{eqn:kappa}.
\end{equation}
$\Delta t$ steht hierbei für die Phasendifferenz zwischen den beiden Thermoelementen.
Die Wellenlänge lässt sich berechnen durch 
\begin{equation}
  \label{eqn:welle}
  \lambda = \sqrt{\frac{4 \pi \kappa T}{\rho c}},
\end{equation}
wobei $T=\frac{1}{f}$ die Periodendauer und $f$ die Frequenz der Welle beschreibt.
