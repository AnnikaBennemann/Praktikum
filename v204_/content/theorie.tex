\section{Theorie}
\label{sec:Theorie}
 
Ziel dieses Versuches ist die Wärmeleitung von Aluminium, Messing und Edelstahl zu untersuchen.

\subsection{Allgemein}

Existiert eine Temperaturungleichgewicht, entsteht ein Wärmetransport entlang des Temperaturgefälles. 
Dies kann entweder durch Konvektion, Wärmeleitung oder Wärmestrahlung passieren.
In diesem Versuch beschränkt man sich auf die Wärmeleitung.\\
In festen Körpern erfolgt der Wärmetransport über Phononen und frei bewegliche Elektronen, wobei der Gitterbeitrag vernachläsigbar ist \cite[1]{V204}.
Die Wärmemenge lässt sich bestimmen durch:
\begin{equation}
    dQ = -\kappa A \frac{\partial T}{\partial x} dt
    \label{eqn:Wärmemenge}
\end{equation}
wobei $\kappa$ die Wärmeleitfähigkeit ist, welche in \autoref{eqn:kappa} beschrieben ist.
Mit der Wärmestromdichte \autoref{eqn:Wärmestromdichte}
\begin{equation}
    j_{\omega} = -\kappa\frac{\partial T}{\partial x}
    \label{eqn:Wärmestromdichte}
\end{equation}
und der Kontinnuitätsgleichung kann eine eindimensionale Wärmeleitungsgleichung \autoref{eqn:1D-Wärmeleitungsgleichung} gebildet werden.
\begin{equation}
    \frac{\partial T}{\partial t} = \frac{\kappa}{\rho c} \frac{\partial ^2 T}{\partial x^2}
    \label{eqn:1D-Wärmeleitungsgleichung}
\end{equation}
Diese gibt an den rämlichen- und zeitlichen Verlauf der Temperaturverteilung an und $\frac{\kappa}{\rho c}$ ist die Temperaturleitfähigkeit $\sigma_t$ von dem Material.
Sie gibt die "Geschwindigkeit" der Wärmemenge an.
Die Temperatur ist kann man als Funktion mit einer zeitlichen Abhängigkeit beschreiben.
Bei einem periodischen Temperaturwechsel bildet sich eine räumliche und zeitliche Temperaturwelle, welche eine Funktion der Form
\begin{equation}
   T(x,t) = T_{max} e^{-\sqrt{\frac{\omega \rho c}{2 \kappa}}x} cos(\omega t - \sqrt{\frac{\omega \rho c}{2 \kappa}})
   \label{eqn:periTempWel}
\end{equation} 
hat.
Aus der Phasengeschwindigkeit $\upsilon$ \autoref{eqn:upsilon} lässt sich mit der Dämpfung die Wärmeleitfähigkeit $\kappa$ \autoref{eqn:kappa} formen. 
Wobei die Dämpfung sich aus der Amplitudenverhältnis $\frac{dA_{nah}}{dA_{fern}}$ berechnen lässt, wobei diese Amplituden an zwei Thermoelementen mit dem Abstand $\Delta x$ gemessen wird.
\begin{align}
    \upsilon = \frac{\omega}{k} = \sqrt{\frac{2\kappa \omega}{\rho c}}
    \label{eqn:upsilon}\\
    \kappa =\frac{\rho c (\Delta x)^2}{2\Delta t \: ln(A_{nah}/A_{fern})}\label{eqn:kappa}
\end{align}