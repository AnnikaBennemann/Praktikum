\section{Durchführung}
\label{sec:Durchführung}

Die Brückenschaltungen werden nach den Abbildungen in \autoref{sec:Theorie} aufgebaut, wobei die Brückenspannung von einem Oszilloskop abgebildet wird.
Das Oszilloskop stellt somit den Nullindikator dar. Die Amplitude, sowie die Frequenz der Speisepannung werden von einem Funktionsgenerator angelegt.

\subsection{Wheatstonesche Brücke}
\label{subsec:wheatstone_durch}
Mithilfe der Wheatstoneschen Brückenschaltung \autoref{fig:wheat} wird der unbekannte Widerstand $Wert 10$ bestimmt.
Am Funktionsgenerator wird eine Amplitude von $1V$ und eine konstanten Frequenz von 76 H eingestellt.
Das Potentiometer R3R4 wird so eingestellt, dass die am Oszilloskop angezeigte Brückenspannung
minimal wird. Nun werden die Werte für R3 R4 abgelesen und in einer Tabelle notiert und Wert 10 berechnet.
Die Messung wird drei Mal mit verschiedenen Referenzwiderständen R2 durchgeführt.

\subsection{Kapazitätsmessbrücke}
\label{subsec:kapazitäts_durch}
Als nächstes wird eine Kapazitätsmessbrücke nach \autoref{fig:kapazität} aufgebaut und es werden nacheinander zwei unbekannte Kapazitäten, Wert 9 (mit integriertem Widerstand) und Wert 1 (ohne integriertem Widerstand),
bestimmt. Zur Bestimmung der unbekannten Kapazitäten werden abwechselnd das Potentiometer R3R4 und der verstellbare Widerstand R2 so eingestellt,
dass auch hier die Brückenspannung minimal wird. Das Potentiometer dient hierbei der Grobeinstellung und der verstellbare Widerstand dient der Feineinstellung.
Die Werte der Widerstände R234 werden notiert und die unbekannten Kapazitäten mitsamt der zugehörigen Widerstände berechnet.

\subsection{Induktivitätsmessbrücke}
\label{subsec:induktivität_durch}
Nun wird die unbekannte Induktivität Lx=Wert 17 und der zugehörige Widerstand Rx berechnet. Die Schaltung wird wie in \autoref{fig:induktivität} konstruiert.
Die Frequenz beträgt nun 2076 Hz.
Zur Messung der unbekannten Induktivität werden das Potentiometer und der verstellbare Widerstand wieder so eingestellt, dass die Brückenspannung minimal wird.
Die Werte der Widerstände R234 werden notiert und die Induktivität, sowie der zugehörige Widerstand berechnet.

\subsection{Maxwell Brücke}
\label{subsec:maxwell_durch}
Mit der Maxwell-Brücke, die nach \autoref{fig:maxwell} aufgebaut wird, wird die Induktiivität Lx=Wert 17 und der zugehörige Widerstand Rx nochmal bestimmt.
Die Frequenz wird auf 4076 erhöht. 
R2 ist nun ein fester Widerstand und R3 sowie R4 sind verstellbare Widerstände, die so eingestellt werden, dass die Brückenspannung minimal wird.
Analog zu \autoref{subsec:induktivität_durch} werden die Messwerte notiert und ausgewertet.

\subsection{Wien-Robinson-Brücke}
\label{subsec:wien-robinson_durch}
Die Bauteile der Wien-Robinson-Brücke werden wie in \autoref{wien-robinson} verschaltet.
In dieser Schaltung sind die Stellglieder nicht die Widerstände, sondern die Frequenz.
Die Amplitude wird auf 10V erhöht.
Nun wird die Grenzfrequenz berechnet und die Brückenspannung bei verschiedenen Frequenzen abgelesen.
Es werden Frequenzen zwischen 20 und 20000Hz eingestellt, wobei der Abstand der Messpunkte in der Umgebung der Grenzfrequenz geringer wird.