\section{Diskussion}
\label{sec:Diskussion}


\begin{table}[H]
    \centering
    \caption{Ergebnisse der Messungen im Vergleich mit Theoriewerten.}
    \label{tab:end}
    \sisetup{table-format=3.2}
    \begin{tabular}{S S S@{${}\pm{}$} S[table-format=1.2]}
     \toprule
      \multicolumn{2}{c}{unbekannte Größe} & {Theoriewert} & \multicolumn{2}{c}{Abweichung [\si{\percent}]}\\
     \midrule
        {Wert 10} & \multicolumn{1}{c|}{R_x [\si{\ohm}]} & 239 & 0.68  & 0.03 \\
     \bottomrule
    \end{tabular}
  \end{table} 

In \autoref{tab:end} sind die Theoriewerte der zu bestimmenden unbekannten Messgrößen aufgelistet.
Außerdem wird die Abweichung zu den zuvor erfassten Messwerten (siehe \autoref{sec:auswertung}) dargestellt.
Die Berechnungen werden mithilfe der Pythonmodule Numpy \cite{numpy} und Uncertainties \cite{uncertainties} durchgeführt.

$Wheat R_x 240.62 +- 0.08
Abweichung Theoriewert 0.68 +- 0.03
Kapazität: wert 9 R_x 429.36 +- 2.31
Abweichung Theoriewert 7.64 +- 0.5$ 

%   aaaaaaaaaaaaaaaaaaa
%   R_x  [240.4137931  240.74074074 240.69478908]
%   Mittelwert  240.61644097535827
%   Standardabweichung 0.14451645518791226
%   Fehler des Mittelwertes  0.08343661430507164
%   Abw theorie 0.68+/-0.03
%   bbbbbbbbbbbbbbbbbbb
%   Wert 9
%   R2 192.0+/-0.4 R3 691.0 R4 309.0
%   Rx 429.36+/-2.31
%   Abw theorie -7.64+/-0.50
%   Cx (4.4360+/-0.0239)e-07
%   Abw theorie 2.28+/-0.55
%   Wert 1
%   R2 0 R3 605.0 R4 395.0
%   Rx 0.00+/-0
%   Cx (6.4767+/-0.0349)e-07
%   Abw theorie -1.87+/-0.53
%   cccccccccccccccccccc
%   R2 59.00+/-0.12 R3 609.0 R4 391.0
%   Rx 91.90+/-0.49
%   Abw theorie -1.87+/-0.53
%   Lx (4.2832+/-0.0231)e-02
%   Abw theorie 2.35+/-0.55
%   ddddddddddddddddddddd
%   R2 500.0+/-1.0 R3 291+/-9 R4 1000+/-30
%   Rx 145.50+/-6.18
%   Abw theorie 55.37+/-6.60
%   Lx (6.5475+/-0.1973)e-02
%   Abw theorie 56.45+/-4.71
%   eeeeeeeeeeeeeeeeeeeee
%   U2 0.1341640786499874
%   Klirrfaktor 0.01341640786499874