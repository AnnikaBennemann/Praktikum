\section{Diskussion}
\label{sec:Diskussion}


\begin{table}[H]
    \centering
    \caption{Ergebnisse der Messungen im Vergleich mit Theoriewerten.}
    \label{tab:end}
    \sisetup{table-format=3.2}
    \begin{tabular}{S S S S[table-format=2.2] @{${}\,\; \pm{}$} S[table-format=1.2]}
     \toprule
      \multicolumn{2}{c}{unbekannte Größe} & {Theoriewert} & \multicolumn{2}{c}{Abweichung [\si{\percent}]}\\
     \midrule
        {Wert 10} & $R_x$ [\si{\ohm}] &\multicolumn{1}{S|} {239.00} & 0.68 &0.03 \\
    \midrule
        {Wert 9} & $R_x$ [\si{\ohm}] &\multicolumn{1}{S|} {464.90} & 7.64 &0.50 \\
                 & $C_x$ [\si{\nano\farad}] &\multicolumn{1}{S|} {433.71} & 2.28 &0.55 \\
        {Wert 1} & $C_x$ [\si{\nano\farad}] &\multicolumn{1}{S|} {660.00} & 1.87 &0.53 \\
     \midrule
        {Wert $17_{\text{I}}$} & $R_x$ [\si{\ohm}] &\multicolumn{1}{S|} {93.65} & 1.87 &0.53 \\
                 & $L_x$ [\si{\milli\henry}] &\multicolumn{1}{S|} {41.85} & 2.35 &0.55 \\
     \midrule
        {Wert $17_{\text{M}}$} & $R_x$ [\si{\ohm}] &\multicolumn{1}{S|} {93.65} & 55.37 &6.60 \\
                    & $L_x$ [\si{\milli\henry}] &\multicolumn{1}{S|} {41.85} & 56.45 &4.71 \\
     \bottomrule
    \end{tabular}
  \end{table} 

In \autoref{tab:end} sind die Theoriewerte der zu bestimmenden unbekannten Messgrößen aufgelistet.
Außerdem wird die Abweichung zu den zuvor erfassten Messwerten (siehe \autoref{sec:Auswertung}) dargestellt.
Die Berechnungen werden mithilfe der Pythonmodule Numpy \cite{numpy} und Uncertainties \cite{uncertainties} durchgeführt.\\

\noindent Anhand \autoref{tab:end} lässt sich feststellen, dass die gemessenen Werte kaum von den Theoriewerten abweichen.
Diese Abweichungen sind durch geringe systematische Fehler zu erklären, zum Beispiel durch das manuelle Einstellen der verstellbaren Widerstände
zur Minimierung der Brückenspannung.
Die untersuchten Schaltungen eignen sich somit sehr gut zur Bestimmung der unbekannten Messgrößen.\\
Die einzige Ausnahme stellen die Werte der Maxwell-Brücke dar. Hier beträgt die Abweichung vom Referenzwert über \qty{50}{\percent}.
Dies lässt sich dadurch erklären, dass die Brückenspannung bei diesem Aufbau minimal wird, wenn $R_4$ \qty{1000}{\ohm} beträgt.
Normalerweise sollte sich $R_4$ bei minimaler Brückenspannung im Bereich zwischen \qty{200}{\ohm} und \qty{800}{\ohm} bewegen. Eine mögliche Ursache für diesen
Ausreißer sind die teilweise sehr alten Bauteile. So zeigt das Oszilloskop zum Beispiel auch eine Spannung an, wenn keine Kabel angeschlossen sind.\\

\noindent Sowohl durch \autoref{fig:wrb-plot} als auch durch die Bestimmung des Klirrfaktors wird deutlich, dass die Abweichungen von den Theoriewerten sehr gering sind.
Der geringe Klirrfaktor sagt aus, dass durch den Funktionsgenerator ein sehr reines Sinussignal erzeugt wird, welches einen geringen Anteil an Oberwellen besitzt. 
Anhand von \autoref{fig:wrb-plot} lässt sich jedoch zusätzlich erkennen, dass dies nur in der Nähe der Grenzfrequenz der Fall ist.
Im Bereich um die Grenzfrequenz sind die Messwerte am genauesten. Je größer der Abstand zur Grenzfrequenz ist, desto größer werden die Abweichungen.
Deswegen ist anzunehmen, dass der Klirrfaktor nur begrenzt als Indikator für die Güte des Funktionsgenerators herangezogen werden kann.