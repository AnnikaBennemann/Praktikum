\section{Diskussion}
\label{sec:Diskussion}

Im ersten Versuchsteil werden die mittleren Energien der Alpha-Strahlung zu
\begin{align*}
    R_{m1}=& (6.40 \pm 0.60)\si{\centi\meter}\\
    R_{m2}=& (2.43 \pm 0.23)\si{\centi\meter}
\end{align*}
und die zugehörigen Energien zu
\begin{align*}
    E_{\alpha, m1}=& (1.62 \pm 0.10)\si{\mega\eV}\\
    E_{\alpha, m2}=& (0.85 \pm 0.05)\si{\mega\eV}
\end{align*}
bestimmt.
Die Energieänderungen betragen
\begin{align*}
    \frac{dE_{\alpha, m1}}{dx} &= c = (-1.199 \pm  0.120)\si{\mega\eV\per\centi\meter}\\
    \frac{dE_{\alpha, m2}}{dx} &= c2 = (-0.717 \pm  0.044)\si{\mega\eV\per\centi\meter}.
\end{align*}
Es fällt auf, dass die Werte der beiden Messreihen stark voneinander abweichen, was so nicht zu erwarten war.
Eine Fehlerquelle in der 
Abstand zu gering daher doofe Werte.


Im zweiten Versuchsteil wird das Histogramm der Messwerte mit einer Gauß-Kurve und einer Poissonverteilung verglichen.
Die