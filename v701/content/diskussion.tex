\section{Diskussion}
\label{sec:Diskussion}

Im ersten Versuchsteil werden die mittleren Reichweiten der Alpha-Strahlung zu
\begin{align*}
    R_{m1}=& (6.40 \pm 0.60)\si{\centi\meter}\\
    R_{m2}=& (2.43 \pm 0.23)\si{\centi\meter}
\end{align*}
und die zugehörigen Energien zu
\begin{align*}
    E_{\alpha, m1}=& (1.62 \pm 0.10)\si{\mega\eV}\\
    E_{\alpha, m2}=& (0.85 \pm 0.05)\si{\mega\eV}
\end{align*}
bestimmt.
Die Energieänderungen betragen
\begin{align*}
    \frac{dE_{\alpha, m1}}{dx} &= c = (-1.199 \pm  0.120)\si{\mega\eV\per\centi\meter}\\
    \frac{dE_{\alpha, m2}}{dx} &= c2 = (-0.717 \pm  0.044)\si{\mega\eV\per\centi\meter}.
\end{align*}
Es fällt auf, dass die Werte der beiden Messreihen stark voneinander abweichen, was so nicht zu erwarten war.
Eine Fehlerquelle in der ersten Messreihe kann ein zu geringer Abstand $x_0$ sein, weshalb unter anderem der Wert $\frac{N_0}{2}$ 
von der Zählrate während der Messung nicht erreicht wird.
Daher sind die Messwerte aus Messreihe 1 eventuell nicht geeignet um die mittlere Reichweite und dementsprechend auch die zugehörige Energie
zu bestimmen.
Die Messreihe 2 entspricht eher den zu erwartenden Ergebnissen. Jedoch gibt es in \autoref{fig:plot2} wenige Werte die im Intervall der 
linearen Regression liegen, jedoch dem linearen Verlauf der Gerade nicht folgen. Diese Werte können als "Ausreißer" beziehungsweise 
fehlerhafte Werte angenommen werden.
\autoref{eqn:Energiereichweite} gilt außerdem nur für Energien unter 2,5 MeV, was im Versuch auch nicht gegeben ist.
Weitere Fehlerquellen sind der ungenau abzulesende Abstand $x_0$ und der Druck $p$.

Im zweiten Versuchsteil wird das Histogramm der Messwerte mit einer Gaußkurve und einer Poissonverteilung verglichen.
Bei dem Vergleich mit der Gaußkurve fällt auf, dass die Werte im Bereich um $N=4600$ stark von der Gaußkurve abweichen, ansonsten
aber recht nah an der Kurve liegen. 
Auch bei der Poissonverteilung lässt sich ein ähnliches Ergebnis im Bereich $k=5$ feststellen.
Der Alpha-Zerfall ist ein zufälliger Prozess, sodass eine Anzahl von $100$ Messungen zu gering ist um eine Häufigkeitsverteilung
genau abbilden zu können.