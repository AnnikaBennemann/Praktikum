\section{Diskussion}
\label{sec:Diskussion}

Abweichungen von Literauturwerten werden im Folgenden mithilfe von
\begin{align*}
    a=\frac{|x-x_{\text{lit}}|}{x_{\text{lit}}}\cdot 100 \label{eqn:abweich}
\end{align*}
berechnet. Die Angabe erfolgt in Prozent.

\subsection{Überprüfung der Bragg-Bedingung} % (fold)
\label{sub:Bragg_dis}
Der bestimmte Winkel $\theta=\qty{14.1}{\degree}$ weicht um $\Delta \theta =\qty{0.1}{\degree}$, also um $\qty{0.71}{\percent}$
vom Sollwinkel $\theta_{soll}=\qty{14.0}{\degree}$ ab.
Die Apparatur ist somit genügend genau einjustiert um weitere Messungen durchführen zu können.


\subsection{Analyse des Emissionsspektrums einer Cu-Röntgenröhre} % (fold)
\label{sub:Emission_dis}

Die für die $K_\alpha$- und die $K_\beta$- Linien bestimmten Energien lauten
\begin{align*}
    E_{K_\alpha}&= \qty{8.01}{\kilo\electronvolt}\\
    E_{K_\beta}&= \qty{8.91}{\kilo\electronvolt}.
\end{align*} 
Die Abweichung zu den in \autoref{sub:Vorbereitung} bestimmten Literaturwerten beträgt bei $E_{K_\alpha}$ $\qty{0.5}{\percent}$
und bei $E_{K_\beta}$ $\qty{0.0}{\percent}$. Die Werte sind also sehr genau bzw. stimmen im Fall von $E_{K_\beta}$ sogar mit dem
Theoriwert überein.

Das Auflösungsvermögen beträgt
\begin{align*}
    A_{K_\alpha} &= 61.62\\
    A_{K_\beta} &= 35.64,
\end{align*}
jedoch ist anzunehmen, dass es nicht sehr genau ist, da die Werte um $\qty{72.9}{\percent}$ voneinander abweichen.
Die Halbwertsbreite weist große Fehler auf, da die Kurve nicht genau genug aufgenommen wurde.
