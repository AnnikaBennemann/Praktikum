\section{Durchführung und Aufbau}
\label{sec:Durchführung}
\subsection{Versuchsaufbau} % (fold)
\label{sub:Versuchsaufbau}
Es werden zwei Stabpendel verwendet, welche eine reibungsarme Spitzenlagerung besitzen.
An den Stabpendeln sind zwei Pendelkörper mit gleicher Masse befestigt.
Diese Pendelkörper können am Stabpendel verschoben werden, sodass mehrere Pendellängen eingestellt werden können.
Zusätzlich können die Pendel mit einer Feder in der Nähe der Massen verbunden werden.

% subsection Versuchsaufbau (end)
\subsection{Versuchdurchführung} % (fold)
\label{sub:Versuchdurchführung}
Es werden die Schwingungsdauern für die gleichphasige Schwingung (\autoref{subsec:Gleich}) und gegenphasige Schwingung (\autoref{subsec:Gegen}) gemessen.
Die Messung wird über fünf Periodenlängen durchgeführt und zehnmal wiederholt, damit Fehler verringert werden.
Die Schwingungsdauern werden notiert.
Danach wird die gekoppelte Schwingung (\autoref{subsec:Gekoppelt}) erzeugt.
Die Schwebungsdauer wird gemessen, indem die Zeit gestoppt wird bis das zweite Pendel zur Ruhe kommt. 
Dies wird fünf mal wiederholt, damit Fehler verringert werden. \\
Die drei verschiedenen Messungen werden analog für eine andere Pendellänge durchgeführt.

% subsection Versuchdurchführung (end)