\section{Zielsetzung}
\label{sec:Zielsetzung}

Ziel dieses Versuches ist es das Relaxationsverhalten eines RC-Kreises zu untersuchen. 
Dies erfolgt durch Bestimmung der Zeitkonstante, Messung der frequenzabhängigen Amplitude der Kondensatorspannung und 
der frequenzabhängigen Phasenverschiebung zwischen Generator- und Kondensatorspannung.
Außerdem wird gezeigt, dass der RC-Kreis als Integrator arbeiten kann.


\section{Theorie}
\label{sec:Theorie}

\subsection{Allgemeines Relaxationsverfahren} % (fold)
\label{sub:Allgemein}
Relaxation in der Physik ist das nicht-oszillatiorische Zurückkehren in den Ausgangszustand eines Systems, nachdem es daraus entfernt wurde.

% subsection Allgemein (end)

\subsection{Entladevorgang eines Kondensators} % (fold)
\label{sub:Entladevorgang}


% subsection Entladevorgang (end)


\subsection{Aufladevorgang eines Kondensators} % (fold)
\label{sub:Aufladevorgang}


% subsection Aufladevorgang (end)

\subsection{Relaxationsphänomen bei einer periodischen Auslenkung} % (fold)
\label{sub:Rela_peri}

% subsection Relaxationsphänomen bei einer periodischen Auslenkung (end)

\subsection{Der RC-Kreis als Integrator} % (fold)
\label{sub:Integrator4}

% subsection  (end)

