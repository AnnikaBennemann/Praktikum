\section{Theorie}
\label{sec:Theorie}

\subsection{Zielsetzung}
\label{sub:Zielsetzung}

Ziel dieses Versuches ist das Relaxationsverhalten eines RC-Kreises zu untersuchen. 
Dies erfolgt durch Bestimmung der Zeitkonstante, Messung der frequenzabhängigen Amplitude der Kondensatorspannung und 
der frequenzabhängigen Phasenverschiebung zwischen Generator- und Kondensatorspannung.
Außerdem wird bestimmt unter welchen Vorrausetzungen der RC-Kreis als Integrator arbeiten kann.

\subsection{Allgemeines Relaxationsverfahren} % (fold)
\label{sub:Allgemein}

% subsection Allgemein (end)

\subsection{Entladevorgang eines Kondensators} % (fold)
\label{sub:Entladevorgang}


% subsection Entladevorgang (end)


\subsection{Aufladevorgang eines Kondensators} % (fold)
\label{sub:Aufladevorgang}


% subsection Aufladevorgang (end)

\subsection{Relaxationsphänomen bei einer periodischen Auslenkung} % (fold)
\label{sub:Rela_peri}

% subsection Relaxationsphänomen bei einer periodischen Auslenkung (end)

\subsection{Der RC-Kreis als Integrator} % (fold)
\label{sub:Integrator}

% subsection  (end)