\section{Auswertung}
\label{sec:Auswertung}

Die in den Messungen verwendeten Zylinder und Platten werden mithilfe einer Schieblehre vermessen und haben die folgenden Dicken $d$.
Dabei werden die Werte als fehlerfrei angenommen.
\begin{description}
  \item[Zylinder 1] \SI{3.10}{\centi\meter}
  \item[Zylinder 2] \SI{3.97}{\centi\meter}
  \item[Zylinder 3] \SI{8.03}{\centi\meter}
  \item[Zylinder 4] \SI{10.20}{\centi\meter}
  \item[Zylinder 5] \SI{12.05}{\centi\meter}
  \item[Platte 1] \SI{0.61}{\centi\meter}
  \item[Platte 2] \SI{1.00}{\centi\meter}
\end{description}

\subsection{Bestimmung der Schallgeschwindigkeit und der Dämpfungskonstante mit dem Impuls-Echo-Verfahren}
\label{sub:ImpEch}
Zur Bestimmung der Schallgeschwindigkeit wird Platte 1 benutzt und dann die Laufzeit $t$ gemessen und in \autoref{tab:ImpLaufzeit} eingetragen.
Die Schallgeschwindigkeit $c$ wird mit \autoref{eqn:Strecke} berechnet, indem die Formel nach $c$ umgestellt und der dadurch berechnete Wert auch in \autoref{tab:ImpLaufzeit} 
eingetragen wird.

\begin{table}[H]
  \centering
  \caption{Laufzeit und Schallgeschwindigkeit durch Platte 1.}
  \label{tab:ImpLaufzeit}
  \sisetup{table-format=2.2}
  \begin{tabular}{S[table-format=1.0] S[table-format=1.0] S[table-format=2.0] S[table-format=4.0] }
  \toprule
  {Messung} & {Laufzeit $\Delta t / \si{\micro\second}$} & {Zeit $\Delta t / \si{\micro\second}$} & {Schallgeschwindigkeit $c / \frac{\si{meter}}{\si{\second}}$}\\
  \midrule
  1: &  5   & 5 & 2440  \\
  2: &  10  & 5 & 2440  \\
  3: &  14  & 4 & 3050  \\
  4: &  19  & 5 & 2440  \\
  \bottomrule
  \end{tabular}
\end{table}
Der Mittelwert der Schallgeschwindigkeit wird mit 
\begin{equation}
  \bar{c}=\frac{1}{n} \sum_{i=1}^n c_i \label{eqn:Mittelwert}
\end{equation}
zu
\begin{align*}
  \bar{c}=2592,5 \si{\meter\per\second}.
\end{align*}
Die Abweichung lässt sich mit
\begin{equation}
  \Delta \bar{c} = \frac{\sqrt{\frac{1}{n-1}\sum_{j=1}^n (c_j-\bar{c})^2}}{\sqrt{n}} \label{eqn:standabw}
\end{equation}
berechnen, sodass sich die mittlere Schallgeschwindigkeit zu
\begin{align*}
  \bar{c}= (2592,5 \pm 305) \si{\meter\per\second}
\end{align*}
ergibt.

Die gemessene Periode $T$ wird mit der daraus nach
\begin{align*}
  f&= \frac{1}{T}
\end{align*}
berechneten Frequenz $f$ und der nach
\begin{align*}
  \lambda &= \frac{c}{f}
\end{align*} 
berechneten Wellenlänge $\lambda$ in \autoref{tab:Tfc} eingetragen.
Die Abweichung der Wellenlänge wird mit der Gaußschen Fehlerfortpflanzung
\begin{equation*}
  \Delta \lambda =\sqrt{\sum_{j=1}^n \left(\frac{1}{f} \cdot \Delta \bar{c} \right)^{2} }\label{eqn:Gauß}
\end{equation*}
bestimmt.
\begin{table}[H]
  \centering
  \caption{Periode, Frequenz und Wellenlänge.}
  \label{tab:Tfc}
  \sisetup{table-format=2.2}
  \begin{tabular}{S[table-format=1.1] S[table-format=1.1] S[table-format=4.2] @{${}\pm{}$} S[table-format=3.2] }
  \toprule
   {Periode $T / \si{\micro\second}$} & {Frequenz $f/ \si{\mega\Hz}$} & \multicolumn{2}{c}{Wellenlänge $\lambda /\si{\meter}$}\\
  \midrule
    0.4  & 2.5 & 1037.00 & 122.41 \\ 
    0.6  & 1.7 & 1525.00 & 179.41 \\
    0.6  & 1.7 & 1525.00 & 179.41 \\
    0.5  & 2.0 & 1296.25 & 152.50 \\
  \bottomrule
  \end{tabular}
\end{table}

Die Mittelwerte der Frequenz und der Wellenlänge betragen somit analog zu \autoref{eqn:Mittelwert} und \autoref{eqn:standabw} 
\begin{align*}
  \bar{f} &= 
  \bar{\lambda} &= 
\end{align*}

Der erste berechnete Wert der Schallgeschwindigkeit in Acryl $c=2440 \frac{\si{meter}}{\si{\second}}$ wird in das Darstellungsprogramm eingegeben.
Es wird eine Tiefenmessung durchgeführt. Die ersten beiden Peaks lagen bei $0.6 \si{\centi\meter}$ und $1.2 \si{\centi\meter}$, woraus abzulesen ist, dass die 
verwendete Platte eine Dicke von $d=0.6\si{\centi\meter}$ hat.\\

In der zweiten Messreihe werden die zuvor ausgemessenen Acrylzylinder mithilfe des Impuls-Echo-Verfahrens gemäß \autoref{sec:Durchführung} vermessen.
Die Ergebnisse der Messung sind in \autoref{tab:tAimpE} aufgelistet.
\begin{table}[H]
  \centering
  \caption{Laufzeit und Amplituden durch verschiedene Zylinder mit dem Impuls-Echo-Verfahren.}
  \label{tab:tAimpE}
  \sisetup{table-format=2.2}
  \begin{tabular}{S[table-format=1.0] S[table-format=2.0] S[table-format=1.2] }
  \toprule
  {Zylinder} & {Laufzeit $t / \si{\micro\second}$} &  {Amplitude $A$}\\
  1 &  24  & 1.24  \\
  2 &  30  & 1.24  \\
  3 &  59  & 0.24  \\
  4 &  76  & 0.08  \\
  \bottomrule
  \end{tabular}
\end{table}
Nach dem Umformen von \autoref{eqn:Strecke} zu
\begin{align*}
  c &= 2 \cdot \frac{d}{t}
\end{align*}
wird die Schallgeschwindigkeit zu
\begin{align*}
  \bar{c}= ( \pm ) \si{\meter\per\second}
\end{align*}
berechnet.


\subsection{Bestimmung der Schallgeschwindigkeit mit dem Durchschallverfahren}
\label{subsec:SchallDurchV}

\begin{table}[H]
  \centering
  \caption{Laufzeit und Amplitudent durch verschiedene Zylinder mit dem Durchschallungs-Verfahren.}
  \label{tab:tADurch}
  \sisetup{table-format=2.2}
  \begin{tabular}{S[table-format=1.0] S[table-format=2.0] S[table-format=1.2] }
  \toprule
  {Zylinder} & {Laufzeit $t / \si{\micro\second}$} &  {Amplitude $A$}\\
  1 &  10  & 1.24  \\
  2 &  15  & 1.32  \\
  3 &  30  & 1.18  \\
  4 &  38  & 0.60  \\
  \bottomrule
  \end{tabular}
\end{table}

\subsection{Biometrische Untersuchung eines Augenmodells}
\label{subsec:Augew}

\begin{table}[H]
  \centering
  \caption{Laufzeiten im Auge.}
  \label{tab:Auge}
  \sisetup{table-format=2.2}
  \begin{tabular}{S[table-format=1.0] S[table-format=2.0] }
  \toprule
  {Peak} & {Laufzeit $t / \si{\micro\second}$} \\
  1 &  7 \\
  2 &  11 \\
  3 &  17 \\
  4 &  25 \\
  \bottomrule
  \end{tabular}
\end{table}