\section{Diskussion}
\label{sec:Diskussion}

Die durch Messungen bestimmten Schallgeschwindigkeiten in Acryl werden mit dem Theoriewert verglichen.
Die Abweichung vom Theoriewert $c_{\text{theo}}=\qty{2750}{\meter\per\second}$ \cite{Schall} wird durch
\begin{align*}
    a=\frac{|c-c_{\text{theo}}|}{c_{\text{theo}}}\cdot 100 \label{eqn:abweich}
\end{align*}
berechnet. Die Ergebnisse sind in \autoref{tab:abweich} dargestellt.
\begin{table}[H]
    \centering
    \caption{Abweichungen der ermittelten Schallgeschwindigkeiten vom Literaturwert\cite{Schall}.}
    \label{tab:abweich}
    \sisetup{table-format=2.2}
    \begin{tabular}{S S[table-format=4.2] @{${}\pm{}$} S[table-format=3.2] S}
    \toprule
    &\multicolumn{2}{{p{3cm}}}{Schallgeschwindigkeit $c / \si{\meter\per\second}$}& {Abweichung $/ \si{\percent}$} \\
    {$c_{\text{IE-Platte}}$} & 2592.50 & 241.12 & 5.72\\
    {$c_{\text{IE-Zylinder}}$} & 2740.76 & 40.50  & 0.34\\
    {$c_{\text{D}}$} & 2791.47 & 76.34  & 1.51\\
    \bottomrule
    \end{tabular}
\end{table}
Anhand \autoref{tab:abweich} ist zu bemerken, dass die Abweichungen bei beiden Messmethoden sehr gering ausfallen, weshalb
anzunehmen ist, dass der Versuch gut geeignet ist um eine Bestimmung der Schallgeschwindigkeit in Acryl durchzuführen.
Die geringe Abweichung, sowie die Tatsache, dass die Messergebnisse auch voneinander geringfügig abweichen, kann verschiedene Ursachen haben.
Zum einen ist auch der Literaturwert nur eine ungefähre Angabe und variiert je nach genauer Zusammensetzung des Materials.
Außerdem hat die Messung nicht unter optimalen Bedingungen stattgefunden und die Probestücke können kleine Beschädigungen aufweisen,
welche das Messergebnis beeinflussen. 
Zudem kann es zu Ablesefehlern gekommen sein, da die Skala nicht sehr genau ist.\\

Die experimentell bestimmten Werte für die Frequenz und Wellenlänge des Ultraschalls
\begin{align*}
    \bar{f} &= (1.98 \pm 0.19) \si{\mega\hertz},\\
    \bar{\lambda} &= (1.346 \pm 0.158) \si{\milli\meter}
\end{align*}
liegen sehr nah an dem von der Sonde erzeugten $\qty{2}{\mega\hertz}$-Signal.
Auch bei dieser Messung kann es durch oben genannte Ursachen zu Messunsicherheiten kommen.
Bei der Tiefenmessung um die Dicke der Platte zu bestimmen, gibt es eine absolute Abweichung von $\qty{0.1}{\milli\meter}$ zur 
zuvor mit der Schieblehre bestimmten Dicke. Diese Unsicherheit ist auch durch menschliche Ablesefehler zu erklären.
Außerdem wurde in das Programm der erste berechnete Wert für die Schallgeschwindigkeit aus der vorangegangenen Messung eingesetzt.
Dieser Wert entspricht nicht dem Mittelwert, also kann es auch deswegen zu weiteren Ungenauigkeiten kommen.\\
%Der Absorptionskoeffizient ergibt sich zu
%\begin{align*}
%  \alpha &=( 39.63\pm 3.61) \si{\per\meter}.
%\end{align*}

Die berechnete Dicke der Ausgleichsschicht beträgt
\begin{align*}
 b &= (3.75\pm 2.00) \si{\milli\meter}.   
\end{align*}
Die genaue Dicke der Ausgleichsschicht am Ultraschallkopf ist nicht angegeben und kann somit nicht verglichen werden. Der berechnete 
Wert scheint jedoch relaistisch zu sein.\\

Die Messung der Dicke der Platte durch die Spektrale Analyse bringt ein Ergebnis identisch zur vorherigen Tiefenmessung hervor.
Es kann von den gleichen Ursachen für Messunsicherheiten ausgegangen werden, allerdings kommt hier erschwerend hinzu, dass die Peaks 
nicht deutlich ablesbar waren.\\

Bei der Vermessung des Augenmodells werden folgende Werte für die Abstände am Augenmodell ermittelt.
\begin{table}[H]
    \centering
    \caption{Abstände im Auge.}
    \label{tab:Auge}
    \sisetup{table-format=2.2}
    \begin{tabular}{c S[table-format=2.2] }
    \toprule
    {Augenbestandteil} & {Standort $s / \si{\milli\meter}$} \\
    Iris            &  7.76\\
    Linseneintritt  &  11.99\\
    Linsenaustritt  &  21.99\\
    Retina          &  60.77\\
    \bottomrule
    \end{tabular}
 \end{table}

Die Werte scheinen realistisch im Bezug auf das Augenmodell bzw. die schematische Darstellung eines Auges (\autoref{fig:Abb_2}) zu sein.
Ungenauigkeiten entstehen hier dadurch, dass es sehr schwierig ist den richtigen Winkel mit der Ultraschallsonde zu treffen um alle 
Bestandteile des Auges als deutliche Peaks abzubilden. Außerdem muss bei der Berechnung der Abstände mit verschiedenen Schallgeschwindigkeiten
gerechnet werden. Auch hier kann es zu Abweichungen von den realen Schallgeschwindigkeiten kommen.

