\section{Zielsetzung}
\label{sec:Zielsetzung}

In diesem Versuch werden die Welleneigenschaften von Licht untersucht. Besonderes Augenmerk liegt hierbei auf der Reflexion, 
Brechung und Beugung eines Lichtstrahls und somit der Strahlen- und Wellenoptik.

\section{Theorie}
\label{sec:Theorie}
Licht und ihre Ausbreitung wird in der Strahlenoptik mit Lichtstrahlen dargestellt, wobei diese die Wellennormalen sind.
Die Ausbreitungsgeschwindigkeit einer Lichtwelle ist abhängig von den Materialeigenschaften, weshalb es zu dem Phänomen der Brechung kommt 
beim Übergang von einem Medium ins andere.
Das Snellius Brechungsgesetz
\begin{align}
    \frac{sin(\alpha)}{sin(\beta)}  &= \frac{v_1}{v_2} = \frac{n_2}{n_1}
    \label{eqn:Brechung}
\end{align}
gibt die Beziehung zwischen den Ausbreitungsgeschwindigkeiten $v_1$ und $v_2$, den Brechungsindizes $n_1$ und $n_2$ und den Winkeln $\alpha$ und $\beta$.
Dabei ist $\alpha$ der Einfallswinkel und $\beta$ der Brechungswinkel.
\begin{figure}[H]
    \centering
    \includegraphics[width=0.5\textwidth]{build/Abb_1.pdf}
    \caption {Schematische Darstellung der Brechung einer Lichtwelle an einer Grenzfläche\cite[1]{V400}.}
    \label{fig:Abb_1}
\end{figure}
In diesem Versuch wird das zweite Medium Luft sein. 
Licht hat in der Luft eine Ausbreitungsgeschwindigkeit von $v_1 = c = 2,9979 \cdot 10^8 \si{\meter\per\second}$ und einen Brechungsindex von $n_1 = 1,000292$.
Somit ist $n_2$ der absolute Brechungsindex.
Ist die Ausbreitungsgeschwindigkeit vom Licht in einem Medium höher als in dem anderen Medium, 
wird es als optisch dichter und umgekehrt als optisch dünner bezeichnet.
In einem homogenen Medium breiten sich die Lichtstrahlen gradlinig aus und beeinflussen sich nicht gegenseitig, selbst wenn sich zwei oder mehrere kreuzen.
Der Weg des Lichtstrahls ist umkehrbar.


\subsection{Reflexion}
\label{subsec:Reflexion}

\begin{minipage}[t]{0.5\textwidth}
    Wenn ein Lichtstrahl auf eine Grenzfläche trifft, wird  dieser reflektiert. 
    Es gilt das Reflexiongesetz.
    Der Einfallswinkel $\alpha_1$ ist gleich der Reflexionswinkel $\alpha_2$.
    \begin{equation}
        \alpha_1 = \alpha_2
        \label{eqn:Reflexion}
    \end{equation}
\end{minipage}
\begin{minipage}[t]{0.5\textwidth}
    \begin{figure}[H]
        \centering
        \includegraphics[width=0.6\textwidth]{build/Abb_2a.pdf}
        \caption {Reflexion eines \\Lichtstrahls an einer Grenzfläche\cite[2]{V400}.}
        \label{fig:Abb_1}
    \end{figure}
\end{minipage}

\subsection{Brechung}
\label{subsec:Brechung}
\begin{minipage}[t]{0,5\textwidth}
    Wenn ein Lichtstrahl auf ein Medium meit einem Brechungsindex $n$ trifft, breitet er sich in diesem Medium aus. 
    Die Ausbreitungsgeschwindigkeiten in den beiden Medium sind verschieden und der Lichtstrahl erfährt an der Grenzfläche eine Richtungsänderung.
    Es gilt das Gesetz von Snellius(\ref{eqn:Brechung}).\\
    Falls ein Lichtstrahl durch einer planparallele Platte hindurch geht und an den beiden Grenflächen gebrochen wird , ändert diese seine Richtung nicht.
    Er erfährt ein Strahlversatz $s$
    \begin{equation}
        s = d\frac{sin(\alpha-\beta)}{cos(\beta)},
        \label{eqn:Strahlversatz}
    \end{equation}
    welche durch den Einfallswinkel $\alpha$, der Geometrie und dem Material der Platte bestimmt wird.

\end{minipage}
\begin{minipage}[t]{0,5\textwidth}
    \begin{figure}[H]
        \centering
        \includegraphics[width=0.6\textwidth]{build/Abb_2b.pdf}
        \caption {Brechung eines \\Lichtstrahls an einer Grenzfläche\cite[2]{V400}.}
        \label{fig:Abb_1}
    \end{figure}

\end{minipage}
    \begin{figure}[H]
        \centering
        \includegraphics[width=0.2\textwidth]{build/Abb_5.pdf}
        \caption {Brechung eines Lichtstrahls an einer planparallelen Platte\cite[2]{V400}.}
        \label{fig:Abb_1}
    \end{figure}

\subsection{Reflexion und Transmission}
\label{subsec:R+T}
\begin{minipage}[t]{0,5\textwidth}
    Lichtstraheln, die auf eine Grenzfläche treffen, werden in der Regel nicht vollständig reflektiert bzw. gebrochen.
    Ein Teil der Intensität wird reflektiert ($R$) und der andere Teil wird transmittiert ($T$) und gebrochen.
    Der Anteil der Reflexion und Transmission ist materialabhängig, es muss jedoch immer $T + R = 1$ gelten
\end{minipage} 
\begin{minipage}[t]{0,5\textwidth}
    \begin{figure}[H]
        \centering
        \includegraphics[width=0.6\textwidth]{build/Abb_2c.pdf}
        \caption {Brechung und Reflexion\\eines Lichtstrahls an\\einer Grenzfläche\cite[2]{V400}.}
        \label{fig:Abb_1}
    \end{figure}

\end{minipage}   
\subsection{Beugung}
\label{subsec:Beugung}
Wenn Licht auf ein Hinderniss trifft, kann beobachtet werden, dass das Licht sich auch im Schattenbereich ausbreitet.
Die Strahlenoptik kann das Phänomen der Beugung nicht erklären, weswegen die Wellenoptik dafür benutzt werden muss.
Wichtig für die elektromagnetischen Wellen ist die Frequenz $\nu$ bzw. die Wellenlänge $\lambda$ und die Ausbreitungsgeschwindigkeit $\v$.
Außerdem gilt das Prinzip der Superposition und das Huygensche Prinzip.\\
Licht besteht aus mehreren Wellenzügen.
Unter der Bedingung der Kohärenz, also dass die Wellenzüge die gleiche Frequenz und eine feste Phasenbeziehung besitzen, erzeugen sie ein Inteferenzbild.
Bei der Überlagerung von Wellen kann es zu einer konstruktiven Inteferenz kommen, also es zu einer  Verstärkung oder zu einer destruktiven
Inteferenz, also einer Abschwächung.\\
Wenn eine ebene Wellenfront auf ein Spalt mit einer Spaltbreite $a$ trifft, werden alle Punkte an der Spaltöffnung gebeugt.
Dabei hat die gebeugte Welle die gleiche Frequenz und eine feste Phasenbeziehung. 
In der Entfernung $L$ ist ein Schirm, auf dem ein Muster aus dunklen und hellen Inteferenzstreifen zu sehen ist.

