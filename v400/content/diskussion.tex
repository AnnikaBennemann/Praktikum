\section{Diskussion}
\label{sec:Diskussion}

Beim Versuch zum Reflexionsgesetz fällt auf, dass die Einfalls- und Reflexionswinkel übereinstimmen.
Durch die Messung der Einfalls- und Reflexionswinkel lässt sich also das Reflexionsgesetz (siehe \autoref{eqn:Reflexion}) verifizieren.
Die Genauigkeit der Winkel ist jedoch durch menschliche Ablesefehler eingeschränkt. Außerdem ist die Skala in 1 Grad Schritte aufgeteilt,
sodass eine genauere Bestimmung der Winkel darüber hinaus sehr schwierig ist.\\


Im zweiten Teil des Versuchs wird das Brechungsgesetz nachvollzogen. Mithilfe der Ergebnisse aus der Messung wird der Brechungsindex
für Plexiglas berechnet. 
Die Abweichung des Brechungsindexes vom Literaturwert (siehe \autoref{sec:Vorbereitung}) wird zu
\begin{align*}
  a=(0,67 \pm 0,70) \si{\percent}
\end{align*}
berechnet. Somit liegt der Literaturwert im Fehlerbereich des berechneten Wertes. Die Messung kann also trotz menschlicher Ablesefehler
und eventuellen Fehlern beim Einstellen der Eintrittswinkel als sehr genau eingestuft werden.\\

Beim Vergleich der beiden Methoden zur Bestimmung des Strahlversatzes lässt sich feststellen, dass die Werte des
Strahlversatzes nicht stark voneinander abweichen. Beide Methoden sind offensichtlich ähnlich gut dazu geeignet den Strahlversatz 
an einer planparallelen Platte zu bestimmen.\\


Beim Prisma ist die Bestimmung der Ablenkung auf zwei verschiedene Methoden durchführbar.
In diesem Versuch wurde sich dazu entschieden die Methode zu verwenden,
in der die Ablenkung über den brechenden Winkel des Prismas berechnet wird (siehe \autoref{eqn:gamma}).
So wird die Fehlerfortpflanzung verringert, da die beiden Brechungswinkel $\beta_1$ und $\beta_2$ nicht
explizit ausgerechnet werden müssen.\\

Im letzten Versuchsteil werden die Wellenlängen der Laser mithilfe von verschiedenen Beugungsgittern bestimmt.
Die berechneten Werte belaufen sich auf
\begin{align*}
    \lambda_{\text{grün}} &= (541,91 \pm 9,80) \si{\nano\meter},\\
    \lambda_{\text{rot}} &= (646,78 \pm 7,33) \si{\nano\meter}.
\end{align*}
Die Abweichung zu den Literaturwerten (siehe \autoref{sec:Durchführung}) ergibt sich zu
\begin{align*}
    a_{\text{grün}}=& (1.86\pm 1.80) \si{\percent},\\
    a_{\text{rot}}=& (1.86 \pm 1.20) \si{\percent}.
\end{align*}
Die theoretischen Wellenlängen liegen gerade nicht im Fehlerbereich der berechneten Werte. Die Abweichung ist jedoch sehr gering,
was auf Ungenauigkeiten beim Aufbau zurückzuführen ist.
\pagebreak




