\section{Vorbereitung}
\label{sec:Vorbereitung}

Im Vorlauf zum Versuch wird der Brechungsindex von verschiedenen Materialien herausgesucht und in \autoref{tab:Brechungsindex} aufgetragen.

\begin{table}
    \centering
    \caption{Literaturwerte des Brechungsindexes verschiedener Materialien.\cite{Brechungsindex} \cite{Brechungsindexplexi}  }
    \label{tab:Brechungsindex}
    \sisetup{table-format=1.6}
    \begin{tabular}{c S}
    \toprule
    {Material} &{Brechungsindex $n$}\\
    \midrule
        Luft & 1.000272 \\
        Wasser & 1.333 \\
        Kronglas & 1.510 \\
        Plexiglas & 1.489 \\
        Diamant & 2.417 \\
    \bottomrule
   
    \end{tabular}
\end{table}

Außerdem soll zu mehreren Gitterdichten die Gitterkonstante $d$ bestimmt werden.
Die Gitterkonstante $d$ ist der Kehrwert der Anzahl der Linien pro $\si{\milli\meter}$.
Somit ergibt sich
\begin{align*}
    600 \, \text{Linien} /\si{\milli\meter} & \implies  d = \frac{1}{600} \si{\milli\meter},\\
    300 \, \text{Linien} /\si{\milli\meter} & \implies  d = \frac{1}{300} \si{\milli\meter},\\
    100 \, \text{Linien} /\si{\milli\meter} & \implies  d = \frac{1}{100} \si{\milli\meter}.\\
\end{align*}