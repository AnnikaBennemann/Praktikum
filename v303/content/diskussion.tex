\section{Diskussion}
\label{sec:Diskussion}

Der Referenzwert für $U_{out}$ beträgt $\qty{32}{\volt}$.
Die aus den Messungen bestimmten Werte für $U_out$ betragen
\begin{align*}
    U_{out} &= 3.2 \pm 0.8 \si{\volt}\\
    \intertext{für die Messung ohne Noise-Generator und}
    U_{out} &= 3.4 \pm 0.6 \si{\volt}
\end{align*}
für die Messung mit zwischengeschaltetem Noise-Generator.
Die Messwert weichen um eine Größenordnung vom Referenzwert ab. Daraus lässt sich folgern, dass der Referenzwert eventuell falsch abgelesen wurde.
Eine andere Fehlerquelle stellt das Oszilloskop dar. Die Bildschirmfotos stellen nur eine Momentaufnahme der teilweise stark flackernden und schwankenden 
Amplituden dar. Die Werte der Amplituden können somit stark fehlerbehaftet sein. Normalerweise sollte die Kurve bei der Messung ohne Noise-Generator nicht
schwanken.

Da die Kurven in \autoref{fig:plot1} und \autoref{fig:plot2} sehr ähnlich verlaufen und die Werte für $U_{out}$ nur um $\qty{0.2}{\volt}$ voneinander abweichen,
lässt sich schließen, dass ein Lock-In-Verstärker gut geeignet ist um verrauschte Signale zu bereinigen und zu vermessen.


In \autoref{fig:plot3} ist zu sehen, dass die Lichtintensität der Diode mit dem Faktor $\frac{1}{x}$ abnimmt, obwohl das Eingangssignal durch Umgebungslicht verrauscht wird, was auch zu erwarten war.
Die auf dem Photodetektor detektierte Spanung wird jedoch auch in großem Abstand der Diode nicht null, da der Raum in dem die Messung stattfand nicht komplett dunkel war.
Ab einem Abstand von $\qty{130}{\centi\meter}$ ändert sich die detektierte Spannung jedoch nicht mehr, sodass dies als maximaler Abstand angenommen werden kann
bis zu dem das Licht der Diode vom Detektor erfasst werden kann.
Bis zu einem Abstand von $\qty{25}{\centi\meter}$ fallen die Werte nicht mit einem Faktor $\frac{1}{x}$. Dies kann darauf zurückzuführen sein,
dass die LED nicht exakt so ausgerichtet ist, dass das Licht senkrecht zum Detektor auftrifft. Dies scheint mit zunehmendem Abstand geringere Auswirkungen
zu haben. Der Lichtstrahl der LED scheint also nicht sehr stark gebündelt zu sein.
