\section{Diskussion}
\label{sec:Diskussion}






Bei der $\beta$-Strahlung ist der Verlauf ähnlich zu dem erwarteten Verlauf. Die Abweichungen können zum einen darin liegen, dass der Zerfall 
nicht statisch ist und gegen Ende der Nulleffekt auch hauptsächlich gemessen wurde. Welcher auch kein gleichbleibender Wert ist, sondern nur ein 
gemittelter Wert über eine bestimmte Zeit.
Die Messfehler können hauptsächlich durch Verlängerung der Messzeit verringert werden.
Es ist anzumerken, dass die Schichtdicke soweit angepasst werden soll, dass die Zählrate annähernd gleich bleibt, 
bis auf statistische Schwankungen.
