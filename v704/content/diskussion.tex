\section{Diskussion}
\label{sec:Diskussion}

Die im ersten Versuchsteil bestimmten Absorptionskoeffizienten werden nun mit theoretisch nach \autoref{eqn:sigmacom} und
\autoref{eqn:Absorptionskoeffizent} berechneten Absorptionskoeffizienten 


Um eine Aussage über die im ersten Versuchsteil ablaufenden Absorptionsmechanismen
machen zu können, werden die experimentell bestimmten Absorptionskoeffizienten
mit den theoretisch berechneten Koeffizienten verglichen, die aus Überlegungen
zur Compton-Streuung folgen. Dabei werden die Gleichungen~\eqref{eqn:sigma_theo}
und~\eqref{eqn:mu_theo} aus dem Theorieteil verwendet, wobei für die Berechnung
der Wirkungsquerschnitte $\sigma_{\mathup{com}}$ der für den
$\ce{^{137}Cs}$-Strahler charakteristische Wert $\varepsilon = \num{1.295}$
benutzt wird. Dabei bezeichnet $z$ die
Kernladungszahl, $M$ die molare Masse und $\rho$ die Dichte des jeweiligen
Stoffes.


\begin{table}[ht]
    \centering
    \caption{Theoretische Werte für die Absorptionskoeffizienten von Zink und Eisen.}
    \label{tab:theoriewerte}
    \sisetup{table-format=2.0}
	\begin{tabular}{S S S S S S S}
			\toprule
			{Absorber} & {$\sigma_{\mathup{com}}$} & {z} & {$M\left[\frac{\si{\gram}}{\si{\mol}}\right]$} &
            {$\rho\left[\frac{\si{\gram}}{\si{\centi\meter\cubed}}\right]$} &
            {$\mu_{\mathup{com}}\left[\frac{1}{\si{\meter}}\right]$} &
            {$\mu_{\mathup{berechnet}}\left[\frac{1}{\si{\meter}}\right]$} \\
			\midrule
			{Eisen} & 2.57e-29 & 26 &  55.8 &  7.874 & 41.16  & 54\\
            {Blei}  & 2.57e-29 & 82 & 207.2 & 11.340 & 97.80 &  \\
			\bottomrule
		\end{tabular}
\end{table}


Bei der $\beta$-Strahlung ist der Verlauf ähnlich zu dem erwarteten Verlauf. Die Abweichungen können zum einen darin liegen, dass der Zerfall 
nicht statisch ist und gegen Ende der Nulleffekt auch hauptsächlich gemessen wurde. Welcher auch kein gleichbleibender Wert ist, sondern nur ein 
gemittelter Wert über eine bestimmte Zeit.
Die Messfehler können hauptsächlich durch Verlängerung der Messzeit verringert werden.
Es ist anzumerken, dass die Schichtdicke soweit angepasst werden soll, dass die Zählrate annähernd gleich bleibt, 
bis auf statistische Schwankungen.
