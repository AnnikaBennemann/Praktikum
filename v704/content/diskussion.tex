\section{Diskussion}
\label{sec:Diskussion}

Die im ersten Versuchsteil bestimmten Absorptionskoeffizienten werden nun mit theoretisch nach \autoref{eqn:sigmacom} und
\autoref{eqn:Absorptionskoeffizent} berechneten Absorptionskoeffizienten 


Um eine Aussage über die im ersten Versuchsteil ablaufenden Absorptionsmechanismen
machen zu können, werden die experimentell bestimmten Absorptionskoeffizienten
mit den theoretisch berechneten Koeffizienten verglichen, die aus Überlegungen
zur Compton-Streuung folgen. Dabei werden die Gleichungen~\eqref{eqn:sigma_theo}
und~\eqref{eqn:mu_theo} aus dem Theorieteil verwendet, wobei für die Berechnung
der Wirkungsquerschnitte $\sigma_{\mathup{com}}$ der für den
$\ce{^{137}Cs}$-Strahler charakteristische Wert $\varepsilon = \num{1.295}$
benutzt wird. Tabelle~\ref{tab:theoriewerte} führt die wesentlichen Größen und
errechneten Werte für Zink und Eisen auf. Dabei bezeichnet $z$ die
Kernladungszahl, $M$ die molare Masse und $\rho$ die Dichte des jeweiligen
Stoffes.
$\ce{^{137}Cs}$-Strahler charakteristische Wert $\varepsilon = \num{1.295}$

\begin{table}[ht]
    \centering
    \caption{Theoretische Werte für die Absorptionskoeffizienten von Zink und Eisen.}
    \label{tab:theoriewerte}
    \sisetup{table-format=2.0}
	\begin{tabular}{S S S S S S S}
			\toprule
			{Absorber} & {$\sigma_{\mathup{com}}$} & {z} & {$M\left[\si{\frac{\gram}{\mol}}\right]$} &
            {$\rho\left[\si{\frac{\kilo\gram}{\metre\cubed}}\right]$} &
            {$\mu_{\mathup{com}}\left[\si{\frac{1}{\metre}}\right]$} &
            {$\mu_{\mathup{berechnet}}\left[\si{\frac{1}{\metre}}\right]$} \\
			\midrule
			Eisen & 2.57e-29 & 26 &  55.8 &  7874 &  & \\

			\bottomrule
		\end{tabular}
\end{table}

\begin{table}[H]
    \centering
    \caption{Messdaten zur Nullmessung beim $\gamma$- und $\beta$-Zerfall.}
    \label{tab:Zerfall0}
    \sisetup{table-format=3.0}
    \begin{tabular}{S S S @{${}\pm{}$}S[table-format=2.2] S[table-format=1.2] @{${}\pm{}$}S[table-format=1.2] }
      \toprule
      \multicolumn{1}{p{2cm}} {Strahlungsart} &\multicolumn{1}{p{2cm}} {Messzeit $t / \si{\second}$} & \multicolumn{2}{c}{$N_0$}& \multicolumn{2}{c}{$\frac{N_0}{t} / \frac{1}{\si{\second}}$}\\
      \midrule
      $\gamma$ & 900 & 828 & 28.77 & 0.92 & 0.03\\
      $\beta$ & 900 & 657 & 25.63 & 0.73 & 0.03\\
    \bottomrule
    \end{tabular}
  \end{table}