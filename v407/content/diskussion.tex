\section{Diskussion}
\label{sec:Diskussion}

Der Brechungsindex des Siliziumspiegels wird im Versuch auf mehrere Arten berechnet.
Der Literaturwert beträgt für eine Wellenlänge von $\qty{633}{\nano\meter}$ $n_{\text{lit}}= 3.35$ \cite{BrechungSilizium}.
In \autoref{tab:diskussion} sind die im Laufe der Messung berechneten Brechungsindizes, sowie die Abweichung
vom Literaturwert nach 
\begin{align*}
    a=\frac{|x-x_{\text{lit}}|}{x_{\text{lit}}}\cdot 100 \label{eqn:abweich}
\end{align*}
eingetragen.

\begin{table}[H]
    \centering
    \caption{Experimentell ermittelte Brechungsindizes von Silizium und Abweichung vom Literaturwert.}
    \label{tab:diskussion}
    \sisetup{table-format=1.2}
    \begin{tabular}{c c c}
        \toprule
       {Berechnungsart}& {Brechungsindex $n$} & {Abweichung $ / \si{\percent}$} \\
        \midrule
        {s-polarisiert} & $3,73 \pm 0,43 $ & $11,38 \pm 12,69$ \\
        {p-polarisiert} & $3,92 \pm 0,20$ & $17,02 \pm 5,82 $ \\
        {über Brewsterwinkel} & $3,73$ & $11,32$ \\  
        \bottomrule
    \end{tabular}
  \end{table}

Die Abweichungen der ermittelten Brechungsindizes vom Literaturwert liegen alle im Bereich zwischen $\qty{11}{\percent}$ und $\qty{18}{\percent}$.
Somit lässt sich sagen, dass die Fresnelformeln mithilfe der Messungen mit hinreichender Genauigkeit verifiziert werden können.
Die Abweichungen vom Literaturwert haben allerdings vielfältige Gründe.
Das manuelle Ablesen des Photostroms vom Messgerät stellt eine Ungenauigkeit dar, da die Werte teilweise schwankten.
Zudem musste die Apparatur zunächst einjustiert werden, auch hier liegen Fehlerquellen.
Nicht immer war es perfekt möglich den Eingangsspalt der Photozelle genau auf die Reflektion des Lasers auszurichten, wodurch die Intensität des
einfallenden Photostroms beeinflusst wird.
Kleine Veränderungen der Lichtverhältnisse im Raum haben außerdem einen Einfluss auf die Genauigkeit der Messergebnisse.

