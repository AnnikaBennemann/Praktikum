\section{Zielsetzung}
\label{sec:Zielsetzung}
Ziel dieses Versuch ist es die Elementarladung mithilfe des Millikan-Öltröpfchenversuchs zu bestimmen.
\section{Theorie}
\label{sec:Theorie}
Bei dem Millikan-Öltröpfchenversuch werden Öltröpfchen durchs Zerstäuben in das vertikale
elektrische Feld eines Plattenkondensators gebracht. Die Tröpfchen werden aufgrund der Reibung
elektrisch geladen, wobei diese Ladung ein ganzzahliges Vielfaches der Elementarladung ist.
Auf die Tröpfchen wirkt die Gravitationskraft $\vec{F_g}=m \vec{g}$, die Reibungskraft 
$\vec{F_r} = - 6 \pi r \nu_L v_0$ und die elektrische Kraft $\vec{F_{el}} = q \vec{E}$.
Die Geschwindigkeit ist abhängig von den Richtungen der Kräfte, welche in \autoref{fig:Abb_1} dargestellt wird.
\begin{figure}[H]
    \centering
    \includegraphics[width=0.5\textwidth]{Abbildungen/Abb_1.pdf}
    \caption {Schematische Darstellung der Kräfte in einem homogenen elektrischen Feld\cite[1]{V503}.}
    \label{fig:Abb_1}
\end{figure}
Liegt eine Spannung an der unteren Platte an wirkt die elektrische Kraft in Richtung der Gravitationskraft.
Die Reibungskraft wirkt entgegen der beiden Kräfte und gegen die Geschwindigkeit $\vec{v_{ab}}$.
Es gilt die Kräftegleichung
\begin{equation}
    \frac{4 \pi}{3} r^3 (\rho_{Oel}-\rho_L)g - 6 \pi \nu_L v_{ab} = - q E.
    \label{eqn:vab}
\end{equation}
Liegt eine Spannung an der oberen Platte wirkt die elektrische Kraft entgegen der Gravitationskraft.
Bei hinreichend hoher Feldstärke entsteht eine Aufwärtsbewegung, welche entgegen der Reibungskraft wirkt.
Es gilt die Kräftegleichung
%\begin{equation}
%    \frac{4 \pi}{3} r^3 (\rho_{Oel}-\rho_L)g + 6 \pi \nu_L v_{auf} = q E.
%    \label{eqn:vauf}
%\end{equation}
%Die Ladung kann mit \autoref{eqn:vab} und \autoref{eqn:vauf} zu
%\begin{equation}
%   q = 3 \pi \nu_L \sqrt{\frac{9}{4} \frac{\nu_}{g} \frac{(v_{ab}-v{auf})}{(\rho_{Oel}-\rho_L)}} \frac{(v_{ab}+v{auf})}{E}
%    \label{eqn:Ladung}
%\end{equation}
%bestimmt werden und der Radius zu
%\begin{equation}
%    r = \sqrt{\frac{9 \nu_L (v_{ab}-v{auf})}{2 g (\rho_{Oel}-\rho_L)}}.
%    \label{eqn:Radius}
%\end{equation}
%Die Viskosität von Luft muss durch 
%\begin{equation}
%    \nu_{eff} = \nu_L \BiggL(\frac{r}{r + A}\Biggr) = \nu_L \BiggL(\frac{\rho r}{\rho r + B}\Biggr)
%    \label{eqn:nu_kor}
%\end{equation}
%korregiert werden, weil das Gesetz von Stokes nur gilt, wenn die Abmessung des Tröpfchen größer als die mittlere
%freie Weglänge ist. Diese Korrektur wird auch als Cunningham-Korrektur bezeichnet , wobei $B = 6.17 \cdot 10^{-3} Torr \cdot cm$ gilt.
%Die Ladung wird mit 
%\begin{equation}
%    q^{\frac{2}{3}} = q^{\frac{2}{3}}_0 (1 + \frac{B}{(p r)})
%    \label{eqn:Ladung_kor}
%\end{equation}
%korregiert, wobei $p$ der Luftdruck ist.