\section{Diskussion}
\label{sec:Diskussion}
Die durch die Messung bestimmten Werte für die Elementarladung werden mit dem Literaturwert \cite{Elementarladung}
verglichen. 
Zur Berechnung der Abweichung wird die Formel 
\begin{align*}
    a=\frac{|x-x_{\text{lit}}|}{x_{\text{lit}}}\cdot 100 \label{eqn:abweich}
\end{align*}
herangezogen.

Die Fehler belaufen sich somit auf
\begin{align*}
    \Delta e_{0,unkorrigiert}&= \qty{6.37}{\percent},\\
    \Delta e_{0,korrigiert}&= \qty{9.49}{\percent},\\
\end{align*}

Die bestimmten Werte liegen in der Größenordnung des Literaturwertes, der Versuch weist jedoch eine hohe Anzahl an möglichen Fehlerquellen auf.
Ein Großteil der Unsicherheiten kann mit menschlichem Verhalten bzw.
dem manuellen Ablesen der Werte im Versuch erklärt werden.
So ist es relativ schwierig durch das Mikroskop einen Tropfen über längere Zeit zu beobachten und seine Position am Raster zu erkennen.
Die Zeitmessung erfolgte auch manuell, auch hier wird die Unsicherheit vergrößert.
Auch Schwankungen in der angelegten Spannung führen zu Messunsicherheiten.
Die errechneten Werte zur Elementarladung können nicht als repräsentativ angenommen werden, da die Abweichung zum Literaturwert
nach der Korrektur größer ist als davor, was nicht zu erwarten ist.
Jedoch beruht die Bestimmung der Elementarladung nicht auf einer wissenschaftlichen Berechnung sondern es wurden per Augenmaß willkürlich Linien
in die Plots hinzugefügt dessen Abstand so angepasst wurde dass möglichst viele Ladungen sich in der Nähe befinden.

