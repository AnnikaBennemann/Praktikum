\section{Diskussion}
\label{sec:Diskussion}
Die durch die Messung bestimmten Werte für die Elementarladung werden mit dem Literaturwert \cite{Elementarladung}
verglichen. 
Zur Berechnung der Abweichung wird die Formel 
\begin{align*}
    a=\frac{|x-x_{\text{lit}}|}{x_{\text{lit}}}\cdot 100 \label{eqn:abweich}
\end{align*}
herangezogen.

Die Fehler belaufen sich somit auf
\begin{align*}
    \Delta e_{0,unkorrigiert}&= \qty{}{\percent},\\
    \Delta e_{0,korrigiert}&= \qty{}{\percent},\\
    \Delta N_{A,unkorrigiert}&=\qty{}{\percent},\\
    \Delta N_{A,korrigiert}&=\qty{}{\percent}.\\
\end{align*}

Der Versuch weist eine hohe Anzahl an möglichen Fehlerquellen auf.
Ein Großteil der Unsicherheiten kann mit menschlichem 