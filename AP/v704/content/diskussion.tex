\section{Diskussion}
\label{sec:Diskussion}

In \autoref{tab:theoriewerte} sind die Ergebnisse der Messungen zur $\gamma$-Strahlung gezeigt.
Es lässt sich feststellen, dass die gemessenen Werte bei beiden Absorbermaterialen weit von den 
theoretisch bestimmten Werten abweichen. Anzumerken ist, dass der gemessene Wert für Blei über dem 
theoretischen Wert liegt, sodass angenommen werden kann, dass hier eher der Photoeffekt eine Rolle spielt.
Messfehler können durch das Absorbermaterial entstanden sein, da dieses eventuell nicht überall dieselbe Dicke 
aufweist.




Bei der $\beta$-Strahlung ist der Verlauf ähnlich zu dem erwarteten Verlauf. Die Abweichungen können zum einen darin liegen, dass der Zerfall 
zufällig ist und gegen Ende hauptsächlich der Nulleffekt gemessen wurde, welcher auch kein gleichbleibender Wert ist, sondern nur ein 
gemittelter Wert über eine bestimmte Zeit.
Die Messfehler können hauptsächlich durch Verlängerung der Messzeit verringert werden.
Es ist anzumerken, dass die Schichtdicke soweit angepasst werden sollte, dass die Zählrate annähernd gleich bleibt, 
bis auf statistische Schwankungen.
