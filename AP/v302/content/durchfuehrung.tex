\section{Durchführung}
\label{sec:Durchführung}

Die Brückenschaltungen werden nach den Abbildungen in \autoref{sec:Theorie} aufgebaut, wobei die Brückenspannung von einem Oszilloskop abgebildet wird.
Das Oszilloskop stellt somit den Nullindikator dar. Die Amplitude, sowie die Frequenz der Speisepannung werden von einem Funktionsgenerator angelegt.

\subsection{Wheatstonesche Brücke}
\label{subsec:wheatstone_durch}
Mithilfe der Wheatstoneschen Brückenschaltung (\autoref{fig:wheat}) wird der unbekannte Widerstand $Wert 10$ bestimmt.
Am Funktionsgenerator wird eine Amplitude von $\qty{1}{\volt}$ und eine konstante Frequenz von $\qty{76}{\hertz}$ eingestellt.
Das Potentiometer $R_3/R_4$ wird so eingestellt, dass die am Oszilloskop angezeigte Brückenspannung
minimal wird. Nun werden die Werte für $R_3$ und $R_4$ abgelesen und in einer Tabelle notiert. Außerdem wird $Wert 10$ berechnet.
Die Messung wird drei Mal mit verschiedenen Referenzwiderständen $R_2$ durchgeführt.

\subsection{Kapazitätsmessbrücke}
\label{subsec:kapazitäts_durch}
Als nächstes wird eine Kapazitätsmessbrücke nach \autoref{fig:kapazität} aufgebaut und es werden nacheinander zwei unbekannte Kapazitäten, $Wert 9$ (mit integriertem Widerstand) und $Wert 1$ (ohne integriertem Widerstand),
bestimmt. Zur Bestimmung der unbekannten Kapazitäten werden abwechselnd das Potentiometer $R_3/R_4$ und der verstellbare Widerstand $R_2$ so eingestellt,
dass auch hier die Brückenspannung minimal wird. Das Potentiometer dient hierbei der Grobeinstellung und der verstellbare Widerstand dient der Feineinstellung.
Die Werte der Widerstände $R_2$, $R_3$ und $R_4$ werden notiert und die unbekannten Kapazitäten mitsamt der zugehörigen Widerstände berechnet.

\subsection{Induktivitätsmessbrücke}
\label{subsec:induktivität_durch}
Nun wird die unbekannte Induktivität $L_{\text{x}}=Wert 17$ und der zugehörige Widerstand $R_{\text{x}}$ berechnet. Die Schaltung wird wie in \autoref{fig:induktivität} konstruiert.
Die Frequenz beträgt nun $\qty{2076}{\hertz}$.
Zur Messung der unbekannten Induktivität werden das Potentiometer und der verstellbare Widerstand wieder so eingestellt, dass die Brückenspannung minimal wird.
Die Werte der Widerstände $R_2$, $R_3$ und $R_4$ werden notiert und die Induktivität, sowie der zugehörige Widerstand berechnet.

\subsection{Maxwell Brücke}
\label{subsec:maxwell_durch}
Mit der Maxwell-Brücke, die nach \autoref{fig:maxwell} aufgebaut wird, wird die Induktiivität $L_{\text{x}}=Wert 17$ und der zugehörige Widerstand $R_{\text{x}}$ nochmal bestimmt.
Die Frequenz wird auf $\qty{4076}{\hertz}$ erhöht. 
$R_2$ ist nun ein fester Widerstand und $R_3$ sowie $R_4$ sind verstellbare Widerstände, die so eingestellt werden, dass die Brückenspannung minimal wird.
Analog zu \autoref{subsec:induktivität_durch} werden die Messwerte notiert und ausgewertet.

\subsection{Wien-Robinson-Brücke}
\label{subsec:wien-robinson_durch}
Die Bauteile der Wien-Robinson-Brücke werden wie in \autoref{fig:wien-robinson} verschaltet.
In dieser Schaltung sind die Stellglieder nicht die Widerstände, sondern die Frequenz.
Die Amplitude wird auf $\qty{10}{\volt}$ erhöht.
Nun wird die Grenzfrequenz berechnet und die Brückenspannung bei verschiedenen Frequenzen abgelesen.
Es werden Frequenzen zwischen $20$ und $\qty{20000}{\hertz}$ eingestellt, wobei der Abstand der Messpunkte in der Umgebung der Grenzfrequenz geringer wird.
Zuletzt wird der Klirrfaktor berechnet.