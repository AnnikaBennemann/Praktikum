\section{Diskussion}
\label{sec:Diskussion}

Das aus \autoref{fig:plot7} bestimmte Verhältnis $\frac{h}{e}$ und die ebenso bestimmte Austrittsarbeit lauten
\begin{align*}
    \frac{h}{e} &= (4.014 \pm 0.177) 10^{-15} \si{\volt\second} \\
    A_K &=  (1.474 \pm 0.111) \si{\electronvolt}.
\end{align*}

Der Literaturwert des Verhältnisses von Planck'schem Wirkungsquantum\cite{Planck} und Elementarladung\cite{Elementarladung} beträgt
\begin{align*}
    \Bigl(\frac{h}{e}\Bigr)_{lit} &= 4.136 \cdot 10^{-15} \si{\volt\second}.
\end{align*}

Die Abweichung zum Literaturwert beläuft sich nach
\begin{align*}
    a=\frac{|x-x_{\text{lit}}|}{x_{\text{lit}}}\cdot 100 \label{eqn:abweich}
\end{align*}

auf $\qty{2.95}{\percent}$. Es handelt sich somit nur um eine geringe Abweichung.

Trotz der geringen Abweichung gibt es einige Fehlerquellen, die berücksichtigt werden müssen.
Die Verbindung der Photozelle mit dem Picoamperemeter stellt eine große Unsicherheit dar.
So ist das Picoamperemeter sehr sensibel und selbst kleinste Bewegungen im Raum führten zu starken Schwankungen.
Zudem war der Raum nicht komplett dunkel, was die Messwerte weiter verfälscht.
Die Messwerte sind daher nicht sehr genau, da vielmals ein Mittelwert nach Augenmaß gebildet werden musste.
Auch bei den Spektrallinien war es teilweise nicht möglich, sie exakt zu fokussieren. Dadurch ergibt sich eine nicht optimale Intensität
an der Photokathode, was auch zu Abweichungen führen kann.