\section{Diskussion}
\label{sec:Diskussion}

Die in \autoref{fig:plot1} dargestellte Charakteristik zeigt nicht ganz das zu erwartende Bild.
Es gibt keinen Anstieg vor oder nach dem Plateau.
Dies kann am Abstand der Quelle zum Zählrohr liegen.
Die Steigung des Plateaus von $a_\text{rel}=(3.46 \pm 3.97)\si{\percent\per100\volt}$ entspricht jedoch der zu erwartenden Größenordnung.
Die Fehler der einzelnen Werte sind aufgrund der Poissonverteilung relativ hoch,
das Zählrohr scheint daher geeignet für eine Messung der Intensität von
hochenergetischer Strahlung zu sein, aber die Präzision ist nicht sehr hoch.

Die bestimmten Totzeiten
\begin{align*}
    T&= 125 \si{\micro\second}\\
    T_2 &= (78 \pm 36)\si{\micro\second}
\end{align*}

weichen um $(60.26 \pm 73.96) \si{\percent}$ voneinander ab.
Gründe dafür sind eine sehr schlechte Ablesbarkeit beim Oszilloskop und die hohen Fehler der Zählraten bei der Zwei-Quellen-Methode.
Durch eine digitale Anzeige oder bessere Skalierung könnte die Genauigkeit der mithilfe des Oszilloskop bestimmten Totzeit verbessert werden.


Die Anzahlen der freigesetzten Ladungen pro einfallendem
Teilchen sind deutlich höher als die Zählrate, was zeigt, dass ein einfallendes Teilchen tatsächlich viele Townsend-Lawinen
auslöst. Außerdem scheint der Zusammenhang zwischen der angelegten Spannung und der Anzahl der freigesetzen
Ladungen linear steigend zu sein.