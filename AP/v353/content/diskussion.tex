\section{Diskussion}
\label{sec:Diskussion}
\subsection{Bestimmung der Zeitkonstante} % (fold)
\label{sub:Bestimmung der Zeitkonstante}
Aus den drei Möglichkeiten zur Bestimmung der Zeitkonstante des Relaxationsverhalten des RC-Kreises folgen drei verschiedene Werte (siehe \autoref{sec:Auswertung}.).
\begin{align*}
    RC_1=&(1.02 ± 0.03)\si{\milli\second}\\
    RC_2=&(1.61 ± 0.06)\si{\milli\second}\\
    RC_3=&(1.65 ± 0.25)\si{\milli\second}
\end{align*}

Die Werte weichen nur wenig von einander ab. Die kleinste Abweichung, von $RC_2$ zu $RC_3$, beträgt $\qty{2.4}{\percent}$ und die größte Abweichung
liegt mit $\qty{38.2}{\percent}$ zwischen $RC_1$ und $RC_3$.
Anhand der Abbildungen in \autoref{sec:Auswertung} und den daraus berechneten Fehlern der Werte, lässt sich folgern, dass die Bestimmung der Zeitkonstante über die Entladekurve,
sowie über den Zusammenhang von der Frequenz und dem Verhältnis der Spannungen weniger fehlerbehaftet ist, als über den Zusammenhang von Phase und Frequenz der Schwingung.
Daraus lässt sich schließen, dass die reale Zeitkonstante in einer Umgebung von $RC_1$ und $RC_2$ liegt.
Die Abweichungen von den Regressionsfunktionen können zum Beispiel durch das manuelle Ablesen der Werte aus der Entladekurve in der ersten Messung entstehen.
Dies wird in der nachfolgenden Messung durch das maschinelle Auswerten der Kurven durch Python vermindert. 
Weitere Messungenauigkeiten lassen sich auf die Ungenauigkeiten der verwendeten Geräte bzw. Bauteile erklären, wie zum Beispiel den Innenwiderstand der
verwendeten Kabel.

\subsection{RC-Kreis als Integrator} % (fold)
\label{sub:RC-Kreis als Integrator}
In \autoref{fig:Integration} ist zu sehen, dass die angelegten Spannungen durch den RC-Kreis integriert werden.
So ensteht bei einer angelegten Sinusspannung am Kondensator eine Kosinusspannung. Dies bestätigt auch den in \autoref{fig:Polar_plot} zu sehenden
Zusammenhang zwischen Amplitude und Phase der Schwingung. So ist dort das Verhältnis $U_C/U_0$ gleich dem Kosinus der Phasenverschiebung.
Weiter bildet sich bei einer angelegten Rechteckspannung eine Dreieckspannung am Kondensator aus und bei einer angelegten Dreieckspannung
erscheint die Kondensatorspannung in paraboloider Kurvenform auf dem Oszilloskop, was auch zu erwarten war.
Somit lässt sich folgern, dass ein RC-Kreis sich unter den gegebenen Bedingungen optimal als Integrator eignet.
Die genauen Werte aus \autoref{fig:Integration} lassen sich nicht mit den Messungen zuvor vergleichen, da für diese ein anderes Oszilloskop verwendet wurde.

