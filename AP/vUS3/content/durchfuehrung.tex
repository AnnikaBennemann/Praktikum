\section{Aufbau}
\label{sec:Aufbau}
Der Aufbau des Versuchs besteht aus einem Ultraschallechoskop mit einer angekoppelten Ultraschallsonde. Das Ultraschallechoskop
ist zudem an einen Computer angeschlossen um die Daten auslesen und auswerten zu können.
Die Ultraschallsonde generiert Impulse mit einer Frequenz von $\qty{2}{\mega\hertz}$.
Zu untersuchen ist eine Anordnung von Röhren mit verschiedenen Durchmessern ($\qty{7}{\milli\meter}$, $\qty{10}{\milli\meter}$ und $\qty{16}{\milli\meter}$), die mit einer einstellbaren
Zentrifugalpumpe verbunden sind. So kann die Strömungsgeschwindigkeit zwischen $0$ und $8500 rpm$ variiert werden.
Das durch die Röhren strömende Flüssigkeitsgemisch besteht aus Wasser, Glycerin und Glaskugeln.
Die Viskosität $\eta$ ist so angepasst, dass sich bei mittlerer Strömungsgeschwindigkeit eine laminare Strömung ausbildet.
Zur Messung der Strömung unter verschiedenen Winkeln werden Doppelprismen mithilfe von Kontaktmittel an die Messstellen
angekoppelt.\\

\section{Durchführung}
\label{sec:Durchführung}
\subsection{Strömungsgeschwindigkeiten}
\label{sub:Strömungsgeschwindigkeiten_durch}
Zunächst wird der Dopplerwinkel für $5$ Strömungsgeschwindigkeiten gemessen.
Dafür wird die Durchflussgeschwindigkeit an der Zentrifugalpumpe eingestellt und für jeden Prismawinkel die
zugehörige Frequenzverschiebung $\Delta \nu$ gemessen. 
Dann wird die Strömungsgescheindigkeit verändert und die Messung erneut durchgeführt.
Die Messwerte werden in einer Tabelle festgehalten.
Außerdem wird aus den Dopplerverschiebungen die Strömungsgeschwindigkeit der Dopplerflüssigkeit berechnet und 
für jeweils einen Dopplerwinkel gegenüber von $\Delta \nu / \cos{\alpha}$ in ein Diagramm aufgetragen.\\

\subsection{Strömungsprofil}
\label{sub:Strömungsprofil_durch}
Im zweiten Teil des Versuchs wird das Strömungsprofil der Dopplerflüssigkeit in dem $\qty{10}{\milli\meter}$ Schlauch mit einem Dopplerwinkel
von $\qty{15}{\degree}$ bestimmt.
Am Ultraschall Doppler-Generator kann die Messtiefe manuell eingestellt werden. Um den Schlauch einmal vollständig
erfassen zu können, wird die Messung von $30$ bis $\qty{11}{\milli\meter}$ durchgeführt. Dies wird für eine
Pumpleistung von $\qty{70}{\percent}$ also $5950$ rpm durchgeführt.
Es werden die Strömungsgeschwindigkeit und der Streuintensitätswert gemessen und in einem Diagramm die
Streuintensität und die Momentangeschwindigkeit als Funktion der Messtiefe aufgetragen.

