\section{Zielsetzung}
\label{sec:Zielsetzung}

Ziel des Versuches ist es, die Reichweite von Alpha-Strahlung in Luft und dadurch die Energie der Alpha-Strahlung zu bestimmen.
Außerdem soll die Statistik des radioaktiven Zerfalls bestimmt werden.

\section{Theorie}
\label{sec:Theorie}

Wenn Alpha-Teilchen durch Luft laufen, kommt es zur Wechselwirkung mit anderen Atomen. Aufgrund von Ionisierungsprozessen, Anregung
und Dissoziation von Mplekülen, kommt es zu Energieverlusten bzw. einer Abgabe von Energien.
Dieser Energieverlust pro Wegstück ist abhängig von der Dichte, der durchlaufenen Materie und der Energie der Alpha-Strahlung.
Die Wechselwirkungswahrscheinlichkeit nimmt bei kleinen Geschwindigkeiten zu, es gibt jedoch keine universelle Formel.
Für alle Energiebereiche gelten unterschiedliche Formeln, bei hinreichend großen Energien gilt die Bete-Bloch-Gleichung
\begin{align}
    -\frac{dE_\alpha}{dx} = \frac{z^2e^4}{4 \pi \epsilon_0 m_e}\frac{nZ}{v^2}\text{ln}\left(\frac{2 m_e v^2}{I}\right).\label{eqn:Bete}
\end{align}
wobei $v$ die Geschwindigkeit der $\alpha$-Strahlung, $z$ die Ladung, $n$ die Teilchendichte, $I$ die Ionisierungsenergie des durchlaufenen Gases, $Z$ die Ordnungszahl sind.
Aufgrund von Ladungsaustauschprozessen gilt die Bete-Bloch-Gleichung nicht für kleine Energien.
Deshalb wird eine empirisch gewonnene Kurve für die Alpha-Strahlung in Luft verwendet.
Es muss
\begin{align}
   R_m=3,1 \cdot E_{\alpha}^{3/2} \label{eqn:Energiereichweite}
\end{align}
gelten, weswegen nur Energien unter 2,5 MeV berücksichtigt werden.
$R_m$ ist die mittlere Reichweite von $\alpha$-Teilchen, welche die Reichweite angibt, die die Hälfte der Teilche noch erreichen.
Die Reichweite $R$ eines Alpha-Teilchens lässt sich durch
\begin{align}
  R=  \int_0^{E_\alpha} \frac{dE_\alpha}{-dE_\alpha/dx} \label{eqn:Reichweite}
\end{align} 
beschreiben, wobei die Energie in Megaelektronenvolt angegeben in einem Bereich unter 2,5 MeV
liegen sollte und $Rm$ in Millimeter angegeben wird.
Wenn die Temperatur und das Volumen konstant bleiben, so ist die Reichweite $R$ eines Alpha-Teilchens proportional zum Druck.
Es wird eine Absorptionsmessung durchgeführt, sodass sich die effektive Länge $x$ bei festem Abstand $x_0$ zwischen Probe und Detektor
durch
\begin{align}
    x = x_0 \frac{p}{p_0} \label{eqn:x}
\end{align}
ausdrücken lässt. Der Normaldruck $p_0$ beträgt hierbei $p_0= \qty{1013}{\milli\bar}$.



