\section{Diskussion}
\label{sec:Diskussion}

Wie in \autoref{tab:Tab3} zu sehen, weicht die reale Güteziffer stark von der idealen Güteziffer der Wärmepumpe ab, was auch zu erwarten war.
Zu den vier betrachteten Zeitpunkten liegt die Abweichung in einem Bereich zwischen $\qty{79.7}{\percent}$ bei $t_2=\qty{840}{\second}$
und $\qty{85.8}{\percent}$ bei $t_4=\qty{1680}{\second}$.
Die starke Abweichung der realen von der idealen Güteziffer hat mehrere Gründe.\\
So wird bei den Berechnungen davon ausgegangen, dass es sich um einen reversiblen Prozess handelt, was bei dem Versuch in der Realität nicht umsetzbar ist.
Zum Beispiel entsteht an verschiedenen Stellen im Aufbau Reibung, welche zu Energieverlusten führen kann.
Das Gas wird im Kompressor außerdem nur nahezu adiabatisch komprimiert.
Ein weiterer Grund für Unsicherheiten besteht in der Dichtigkeit der Apparatur. So sind die Eimer nicht dicht abgeschlossen und nicht komplett wärmeisolierend.
Auch die Isolierung der Kupferrohre ist nicht optimal und kann den Wärmeaustausch mit der Umgebung nur begrenzt verhindern.
Ein weiteres Problem stellt die Temperaturmessung dar. Die Temperatur in Reservoir 2 sinkt unter den Gefrierpunkt, sodass sich Eis bildet und das Wasser
nicht mehr optimal durchgerührt werden kann, beziehungsweise keine homogene Temperatur mehr aufweist.
Das manuelle Ablesen der Werte jede Minute kann auch zu Fehlern führen. Zum Einen sind die Barometer und das Wattmeter nicht digital und relativ grob skaliert, was ein
genaues Ablesen erschwert. Außerdem ist es schwer fünf Werte gleichzeitig zu erfassen und aufzuschreiben, auch hier können menschliche Fehler unterlaufen,
die jedoch in den berechneten Messungenauigkeiten der Werte nicht erfasst werden.

