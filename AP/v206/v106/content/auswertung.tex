\section{Auswertung}
\label{sec:Auswertung}

Die Messungen werden nach \autoref{sec:Durchführung} durchgeführt und die Ergebnisse im Folgenden dargestellt und ausgewertet.
Die Länge der Pendel beträgt beim ersten Durchgang $l=\qty{0.63}{\meter}$ und wird, nachdem alle drei Messreihen durchgeführt wurden, auf 
$l=\qty{0.36}{\meter}$ verkürzt. Die Masse der Pendelkörper beträgt $m=\qty{1}{\kilo\gram}$. 
Die Auswertung erfolgt parallel für beide Pendellängen.
Als Konstante für den Ortsfaktor der Erdbeschleunigung wird in den Berechnungen $g =\qty{9.80665}{\meter\per\second\squared}$
verwendet.

\subsection{Gleichsinnige Schwingung}
\label{subsec:aus_gleich}

\begin{table}[H]
  \centering
  \caption{Messwerte der Schwingungsdauer bei gleichsinniger Schwingung.}
  \label{tab:T+}
  \sisetup{table-format=1.2}
  \begin{tabular}{l S S}
    \toprule
     & {$T_+ \mathbin{/} \si{\second}$ bei $l=\qty{0.63}{\meter}$} & {$T_+ \mathbin{/} \si{\second}$ bei $l=\qty{0.36}{\meter}$}\\
    \midrule
      &1,50 & 1,26\\
      &1,53 & 1,23\\
      &1,55 & 1,26\\
      &1,56 & 1,24\\
      &1,50 & 1,24\\
      &1,53 & 1,24\\
      &1,56 & 1,24\\
      &1,51 & 1,22\\
      &1,50 & 1,24\\
      &1,53 & 1,24\\
    \midrule
      Mittelwert $\overline{T_+}$& 1,53 & 1,24\\
      Fehler $\Delta \overline{T_+}$& 0,01 & 0,00\\
    \bottomrule
  \end{tabular}
\end{table}
In \autoref{tab:T+} sind die Messwerte der Schwingungsdauer bei gleichsinniger Schwingung dargestellt. Zusätzlich wird der Mittelwert und der zugehörige
Fehler mithilfe von Python nach den Formeln in \autoref{sec:Fehlerrechnung} bestimmt.
Aus den Messwerten lässt sich mithilfe von \autoref{eqn:omega+} $w_+$ berechnen zu 
\begin{align*}
  \omega_{+(0,63)}&=(4,11 \pm 0,02) \si{\per\second} & \text{und} && \omega_{+(0,36)}&=(5,06 \pm 0,02) \si{\per\second}.
\end{align*}

\subsection{Gegensinnige Schwingung}
\label{subsec:aus_gegen}
\begin{table}[H]
  \centering
  \caption{Messwerte der Schwingungsdauer bei gegensinniger Schwingung.}
  \label{tab:T-}
  \sisetup{table-format=1.2}
  \begin{tabular}{l S S}
    \toprule
     & {$T_- \mathbin{/} \si{\second}$ bei $l=\qty{0.63}{\meter}$} & {$T_- \mathbin{/} \si{\second}$ bei $l=\qty{0.36}{\meter}$}\\
    \midrule
      &1,39 & 1,24\\
      &1,41 & 1,22\\
      &1,39 & 1,21\\
      &1,40 & 1,21\\
      &1,36 & 1,19\\
      &1,32 & 1,20\\
      &1,41 & 1,18\\
      &1,38 & 1,22\\
      &1,37 & 1,19\\
      &1,44 & 1,18\\
    \midrule
      Mittelwert $\overline{T_-}$& 1,39 & 1,20\\
      Fehler $\Delta \overline{T_-}$& 0,01 & 0,01\\
    \bottomrule
  \end{tabular}
\end{table}
Die Messwerte der Messung der Schwingungsdauer bei gegensinniger Schwingung, sowie der Mittelwert und der zugehörige Fehler sind analog zu
\autoref{subsec:aus_gleich} in \autoref{tab:T-} aufgetragen.
Aus diesen Werten lässt sich mithilfe von \autoref{eqn:omega-} $w_-$ berechnen zu 
\begin{align*}
  \omega_{-(0,63)}&=(4,53 \pm 0,03) \si{\per\second} & \text{und} && \omega_{-(0,36)}&=(5,22 \pm 0,02) \si{\per\second}.
\end{align*}

$K$ kann nun mit \autoref{eqn:K} und den oben bestimmten Werten für $T_+$ und $T_-$ berechnet werden zu
\begin{align*}
  K_{(0,63)}&=(0,10 \pm 0,01) & \text{und} && K_{(0,36)}&=(0,03 \pm 0,01).
\end{align*} 

\subsection{Gekoppelte Schwingung}
\label{subsec:aus_gekoppelt}

\begin{table}[H]
  \centering
  \caption{Messwerte der Schwingungsdauer bei gekoppelter Schwingung.}
  \label{tab:Ts}
  \sisetup{table-format=2.2}
  \begin{tabular}{l S S}
    \toprule
     & {$T_s \mathbin{/} \si{\second}$ bei $l=\qty{0.63}{\meter}$} & {$T_s \mathbin{/} \si{\second}$ bei $l=\qty{0.36}{\meter}$}\\
    \midrule
      &15,35& 25,29\\
      &15,29& 25,27\\
      &15,19& 24,95\\
      &15,39& 25,69\\
      &15,24& 25,24\\
    \midrule
      Mittelwert $\overline{T_-}$& 15,29 & 25,29\\
      Fehler $\Delta \overline{T_-}$& 0,03 & 0,11\\
    \bottomrule
  \end{tabular}
\end{table}
Aus den Messwerten der Schwebungsdauer bei der gekoppelten Schwingung (siehe \autoref{tab:Ts}) lässt sich mithilfe
von \autoref{eqn:omegaS} $w_s$ berechnen zu 
\begin{align*}
  \omega_{s(0,63)}&=(0,42 \pm 0,04) \si{\per\second} & \text{und} && \omega_{s(0,36)}&=(0,15 \pm 0,03) \si{\per\second}.
\end{align*}






