\section{Diskussion}
\label{sec:Diskussion}

\begin{table}[H]
    \centering
    \caption{Vergleich der gemessenen Werte mit den Theoriewerten.}
    \label{tab:Vergleich}
    \sisetup{table-format=3.2}
    \begin{tabular}{c c c c S}
      \toprule
      &&{gemessener Wert}&{Theoriewert}&{Abweichung $\mathbin{/} \si{\percent}$}\\
      \midrule
        \multirow{6}{*}{$l=0,63 \si{\meter}$}
        &{$T_+ \mathbin{/} \si{\second}$} & \, 1,53 ± 0,01 & 1,59 & 4,1 \\
        &{$T_- \mathbin{/} \si{\second}$} & \, 1,39 ± 0,01 & 1,58 & 12,1 \\
        &{$T_s \mathbin{/} \si{\second}$} & 15,29 ± 0,03 & 163,00 ± 14,00 & 90,6\\ 
        &{$\omega_+ \mathbin{/} \si{\per\second}$} & \, 4,11 ± 0,02 & 3,95 & 4,3\\
        &{$\omega_- \mathbin{/} \si{\per\second}$} & \, 4,53 ± 0,03 & 3,98 & 13,7\\
        &{$\omega_s \mathbin{/} \si{\per\second}$} & \, 0,42 ± 0,04 & 0,04 & 982,0\\
        \midrule
        \multirow{6}{*}{$l=0,36 \si{\meter}$}
        &{$T_+ \mathbin{/} \si{\second}$} & \, 1,24 ± 0,00 & 1,20 & 3,0\\
        &{$T_- \mathbin{/} \si{\second}$} & \, 1,20 ± 0,01 & 1,20 & 0,3\\
        &{$T_s \mathbin{/} \si{\second}$} & 25,29 ± 0,11 & 400,00 ± 70,00 & 93,6\\ 
        &{$\omega_+ \mathbin{/} \si{\per\second}$} & \, 5,06 ± 0,02 & 5,22 & 3,0\\
        &{$\omega_- \mathbin{/} \si{\per\second}$} & \, 5,22 ± 0,02 & 5,24 & 0,3 \\
        &{$\omega_s \mathbin{/} \si{\per\second}$} & \, 0,15 ± 0,03 & 0,02 & 867,6\\
      \bottomrule
    \end{tabular}
\end{table}

In \autoref{tab:Vergleich} sind die Messwerte aus \autoref{sec:Auswertung} dargestellt.
Außerdem werden die Theoriewerte, sowie die Abweichung von den Theoriewerten berechnet und auch in der Tabelle aufgetragen.
Die Abweichungen der Messwerte der gleichsinnigen Schwingung liegen alle unter $\qty{5}{\percent}$, sodass diese Messung als Erfolg verbucht werden kann.
Größere Abweichungen bis zu $\qty{982}{\percent}$ finden sich jedoch bei den Schwingungsdauern und Schwebungsfrequenzen der gekoppelten Schwingung.
Hier muss jedoch hinzugefügt werden, dass der Wert für $K$ nach \autoref{eqn:K} mit den experimentell bestimmten Werten berechnet wird.
Somit sind die Theoriewerte für die gegensinnige und die gekoppelte Schwingung indirekt von den experimentellen Werten abhängig und können nicht 
direkt verglichen werden. Die Kopplungsfeder kann auch nicht wie in der Theorie angenommen an den Pendelkörpern befestigt werden, sondern befindet
sich in geringem Abstand darüber.


Die Messunsicherheiten haben verschiedene Gründe.
Die Pendellänge wird mit einem Maßband bestimmt und ist somit nicht sehr genau. Somit kann es auch kleine Unterschiede in der Pendellänge der beiden 
Pendel geben.
Zudem werden die Pendel manuell ausgelenkt. Es kann aufgrund fehlender Skala nur abgeschätzt werden, wie groß der Auslenkungswinkel maximal sein darf um 
im Bereich der Kleinwinkelnäherung ($\alpha < \qty{10}{\degree}$) zu bleiben. Die Kleinwinkelnäherung ist, wie der Name schon sagt, auch nur eine Näherung,
durch die kleinere Abweichungen entstehen können.
Die Pendel bewegen sich außerdem nicht nur zweidimensional sondern auch dreidimensional zur Wand hin und weg, wodurch 
etwas Pendelenergie verloren geht und das Pendel kein perfekter harmonischer Oszillator mehr ist. 
Auch Reibung wird in der Auswertung nicht berücksichtigt, was zu weiteren Unsicherheiten führen kann.
Um den Fehler der Stoppuhr und der menschlichen Reaktionszeit auszugleichen werden die Messungen mehrfach und über mehrere Perioden durchgeführt.
Eine genauere Messung wäre zum Beispiel durch eine automatische Zeitmessung über Lichtschranken oder ähnliches möglich.



