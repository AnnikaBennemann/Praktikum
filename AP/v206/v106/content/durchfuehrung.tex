\section{Durchführung und Aufbau}
\label{sec:Durchführung}
\subsection{Versuchsaufbau} % (fold)
\label{sub:Versuchsaufbau}
Es werden zwei Stabpendel verwendet, welche eine reibungsarme Spitzenlagerung besitzen.
An den Stabpendeln sind zwei Pendelkörper mit gleicher Masse befestigt.
Diese Pendelkörper können am Stabpendel verschoben werden, sodass mehrere Pendellängen eingestellt werden können.
Zusätzlich können die Pendel mit einer Feder in der Nähe der Massen verbunden werden.

% subsection Versuchsaufbau (end)
\subsection{Versuchdurchführung} % (fold)
\label{sub:Versuchdurchführung}
Zuerst wird die Pendellänge mit einem Maßband gemessen und notiert. Beide Pendel werden auf die gleiche Pendellänge eingestellt.
Nun wird die Schwingungsdauer für die gleichphasige Schwingung (\autoref{subsec:Gleich}) gemessen. Dazu werden beide Pendel in dieselbe Richtung
um möglichst den selben Auslenkungswinkel $\phi$ ausgelenkt sodass die Kleinwinkelnäherung noch gilt. Die Pendel werden gleichzeitig losgelassen und
eine Stoppuhr gestartet. Nach fünf Perioden wird die Stoppuhr gestoppt und der Wert für die Schwingungsdauer notiert. 
Die Messung wird über fünf Periodenlängen durchgeführt und zehnmal wiederholt, damit Fehler verringert werden.
Danach wird die Messung der Schwingungsdauer bei gegenphasiger Schwingung (\autoref{subsec:Gegen}) gemessen.
Dies wird analog durchgeführt, jedoch werden die Pendel nun in entgegengesetzte Richtungen ausgelenkt.
Zuletzt wird die gekoppelte Schwingung (\autoref{subsec:Gekoppelt}) erzeugt.
Hierbei wird nur ein Pendel ausgelenkt, während sich das andere in Ruhelage befindet.
Die Schwebungsdauer wird gemessen, indem die Zeit gestoppt wird, bis das Pendel welches im Anfangszustand in Ruhelage war zum fünften Mal wieder still steht. 
Auch diese Messung wird somit für fünf Schwebungsperioden durchgeführt und die Messung fünf Mal wiederholt, damit Fehler verringert werden. \\
Die drei verschiedenen Messreihen werden analog für eine weitere Pendellänge durchgeführt.


% subsection Versuchdurchführung (end)