\section{Diskussion}
\label{sec:Diskussion}

Wie schon in \autoref{subsec:aus_stat} angegeben, bestitzt Aluminium von den verglichenen Metallen die beste Wärmeleitung
und Edelstahl die schlechteste. Dies war im Bezug auf \autoref{tab:Vorbereitung} auch anzunehmen. 


\begin{table}[H]
    \centering
	\caption{Vergleich der berechneten Wärmeleitfähigkeiten mit Literaturwerten.}
	\label{tab:Vlit}
    \sisetup{table-format=3.2}
    \begin{tabular}{l
        S@{${}\pm{}$}
        S
        S[table-format=3.0]
        S}
		\toprule
		\multicolumn{1}{l}{Material}&\multicolumn{2}{p{3cm}}{berechneter Wert/$\,\si{\watt\per\meter\per\kelvin}$} &\multicolumn{1}{p{3cm}}{ Literaturwert/ $\si{\watt\per\meter\per\kelvin}$} & \multicolumn{1}{l}{Abweichung/ $\si{\percent}$}\\
		\midrule
		Messing &133.95& 63.92&113&18.54\\
        Aluminium&250.72&140.93&237&5.79\\
        Edelstahl&10.97&1.52&20&45.17\\
		\bottomrule
	\end{tabular}
\end{table}	

\noindent In \autoref{tab:Vlit} sind die aus den Messwerten berechneten Wärmeleitfähigkeiten und die zugehörigen Literaturwerte, sowie die Abweichungen davon aufgelistet.
Die Werte von Messing und Aluminium wurden der Messung nach der Angström-Methode mit einer Periode von $\qty{80}{\second}$ entnommen, bei Edelstahl wurden die Werte aus der Messung
mit einer Periode von $\qty{200}{\second}$ genommen.


\begin{table}[H]
    \centering
    \caption{Wellenlängen der Temperaturwellen.}
    \label{tab:Wellenlängevergleich}
    \sisetup{table-format=1.3}
    \begin{tabular}{l
        S@{${}\pm{}$}
        S
        S
        S[table-format=2.3]}
        \toprule
        \multicolumn{1}{l}{Material}&\multicolumn{2}{p{3cm}}{berechneter Wert/ $\, \si{\m}$} &\multicolumn{1}{p{3cm}}{ Literaturwert/ $\si{\m}$ siehe \autoref{tab:Vorbereitung}} & \multicolumn{1}{l}{Abweichung/ $\si{\percent}$}\\
        \midrule
        Messing & 0.203 & 0.048 & 0.186 & 8.88\\
        Aluminium & 0.329 & 0.093 & 0.320 & 2.85\\
        Edelstahl & 0.093 & 0.006 & 0.125 & 25.95\\
        \bottomrule
    \end{tabular}
  \end{table}

\noindent Die Literaturwerte für die Wärmeleitfähigkeiten, sowie für die Wellenlängen von Aluminium und Messing liegen im Fehlerbereich der berechneten Werte, wobei die Abweichung bei Aluminium am geringsten ist.
Die Werte von Edelstahl weichen jedoch stark von den Literaturwerten ab. Ein Grund dafür kann sein, dass es verschiedene Edelstahl-Zusammensetzungen mit unterschiedlichen
Eigenschaften gibt und nicht genau klar ist um welche Zusammensetzung es sich beim Probestab handelt.

\noindent Es gibt verschiedene weitere Umstände die zu Unsicherheiten der Messungen geführt haben können.
Systematische Unsicherheiten treten zum Beispiel dadurch auf, dass bei dem Versuch keine Laborbedingungen, wie z.B $\qty{20}{\degreeCelsius}$ Raumtemperatur, vorlagen, was auch zu größeren Abweichungen führt.
Die Stäbe waren unzureichend isoliert, wodurch der Wärmeaustausch mit der Umgebung zwar verringert, aber nicht ganz vermieden werden konnte.
Ein Grund für statische Messunsicherheiten stellen die Anfangstemperaturen der Stäbe dar, die nicht exakt gleich waren.
Schwankungen in der Periodendauer treten durch das manuelle Umschalten des Schalters und Entfernen der Isolierung bei der periodischen Messung auf.
Die Unsicherheiten in der Temperaturmessung von den Thermoelementen fließen auch in die Messung ein.
Auch die digitale Auswertung der Daten der Thermoelemente durch den GLX Datenlogger kann zu Fehlern führen.
So gibt es in \autoref{fig:mess_dyn} und \autoref{fig:alu_dyn} Ausreißer bei ca. $\qty{130}{\second}$ wo die Temperatur ohne erkennbaren Grund stark fällt und sofort wieder stark
steigt, bevor die Kurve wieder den vorherigen Verlauf annimmt.

 




