\section{Diskussion}
\label{sec:Diskussion}
In diesem Versuch wurden zwei Aluminiumstäbe mit verschiedenen Formen ausgewählt.
Die Form ist einmal eckig und einmal rund.
Beide Stäbe wurden wie in der Durchführung beschrieben untersucht.
Die experimentell bestimmten Werte für das Elastizitätsmodul sind bei einseitiger Einspannung:
\begin{align*}
    E_{\text{rund}}=&(116,16\pm3,04)\si{\giga\pascal}\\
    E_{\text{eckig}}=&(89,22\pm1,13)\si{\giga\pascal}
 \end{align*}
und bei beidseitiger Einspannung:
\begin{align*}
    E_{\text{rund-3}}=&(72,53\pm2,23)\si{\giga\pascal}\\
    E_{\text{eckig-3}}=&(128,25\pm3,96)\si{\giga\pascal} \text{.}
 \end{align*}
Die Literaturwerte für das Elastizitätsmodul \cite{Elastizität} liegen ungefähr bei 70 GPa. 
Es fällt auf, dass die experimentell bestimmten Werte, bis auf das Elastizitätmodul beim beidseitig eingespannten runden Stab, stark vom Literaturwert abweichen.
Die größte prozentuale Abweichung gab es bei der Messung $E_{\text{eckig-3}}$ mit $83,21\%$ und die niedrigste Abweichung gab es bei der $E_{\text{rund-3}}$ mit $3,61\%$.
Die geringen Fehlerwerte, lassen darauf schließen, dass es sich um statistische Messunsicherheiten handelt, da diese sich bei vielen Messungen "herausmitteln".
Der Wert mit der geringsten Messungenauigkeit ist $E_{\text{eckig}}$. 
Somit liefert die Methode des einseitigen Einspannens eines eckigen Stabes die genauesten Werte für die Bestimmung des Elastizitätsmoduls eines Materials.
Trotzdem gibt es in der Durchführung einige Umstände, die zu Messungenauigkeit führen können.
So sind beispielsweise die Messuhren sehr stoßempfindlich und verstellen sich häufig auch ohne äußere Einwirkung.
Digitale Messuhren würden daher die Messung vereinfachen und die Fehler minimieren.
Außerdem ist die Auflagefläche für die Messuhren beim runden Stab nicht ganz eben, weswegen die Uhren teilweise seitlich vom Stab rutschen können und somit die Messwerte verfälscht werden.
Ungenaues Ablesen kann eine weitere Fehlerquelle darstellen.
Diese Methode kann durch das Phänomen der elastischen Nachwirkung verfälscht werden.
Es kann angenommen, dass die Stäbe auch vor anhängen eines Gewichtes Verformungen aufgewiesen haben.
Durch das nullen der Messuhren vor jeder Messung wurde dies jedoch versucht auszugleichen.
Zuletzt könnte eine weitere Fehlerquelle sein, dass bei der beidseitigen Messung das Gewicht nicht so ausgewählt wurde, dass die maximale Auslenkung zwischen $3 \unit{mm}$ und $7 \unit{mm}$ liegt.




