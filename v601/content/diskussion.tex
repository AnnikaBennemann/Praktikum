\section{Diskussion}
\label{sec:Diskussion}

Das Verhältnis der Weglänge zum Abstand zwischen Kathode und Beschleunigungselektrode liegt für die Temperaturen $T_2$ und $T_3$
im Bereich zwischen $1000$ und $4000$, sodass hier eine ausreichende Stoßwahrscheinlichkeit gegeben ist.
Für $T_1$ ist $\bar{\omega}$ zu groß im Verhältnis zu $a$, dadurch wächst die Wahrscheinlichkeit, dass die Elektronen die Auffängerelektrode
ohne Wechselwirkung erreichen.
Bei $T_4$ übersteigt das Verhältnis einen Wert von $4000$, sodass hier die Zahl der elastischen Stöße zunimmt, welche mit 
Richtungsänderungungen verbunden sind und somit weniger Elektronen die Auffängerelektrode erreichen.\\

Die integrale und auch die daraus berechnete differentielle Energieverteilung der Elektronen zeigt den zu erwartenden Verlauf.
Bei der differentiellen Kurve von $T_1$ ist der Tiefpunkt sehr gut abzulesen und somit kann das Kontaktpotential zu 
\begin{align*}
    K_1 &= \qty{2.75}{\volt}.\\
\end{align*}
bestimmt werden.
Auch die Franck-Hertz-Kurven zeigen die zu erwartenden Verläufe unter den in \autoref{subsec:Franck-Hertz-Kurve} beschriebenen Nebeneffekten,
die die Kurve verändern.
Die Anregungsenergie und die Wellenlängen der emittierten Photonen ergeben sich zu
\begin{align*}
    E_{169} &= (4,815 \pm 0,051) \si{\electronvolt}\\
    E_{179} &= (4,714 \pm 0,057) \si{\electronvolt}\\
    \lambda_{169} &= (257,50 \pm 2,73) \si{\nano\meter}\\
    \lambda_{179} &= (263,01 \pm 3,18) \si{\nano\meter}.
\end{align*}
Die Energieverluste der Elektronen beim zentral elastischen Stoß sind
so gering, dass diese nicht berücksichtig werden müssen, solange der Richtungswechsel die 
Elektronen nicht aus dem Bereich hinter der Beschleunigungselektrode rausbewegt.
Der Theoriewert für die Anregungsenergie des Hg-Atoms liegt bei $\qty{4.9}{\electronvolt}$\cite{Anregungsenergie}.
Die Abweichungen der Energien betragen nach 
\begin{align*}
    abw=\frac{|x_\mathrm{gemessen}-x_\mathrm{theorie}|}{x_\mathrm{theorie}}\cdot 100 \label{eqn:abweich}
\end{align*}
bei $T_3$ $\qty{1.73}{\percent}$ und bei $T_4$ $\qty{3.80}{\percent}$.
Die Ergebnisse sind also zufriedenstellend genau.\\


Fehlerquellen liegen vor allem im XY-Schreiber beziehungsweise dem Digitalisieren und Ablesen der Werte.
Die Digitalisierung mithilfe des WebPlotDigitizers ist verhältnismäßig einfach, und es muss nicht jeder Wert einzelnd abgelesen werden, 
jedoch können Fehler durch ungenaue Skalierungen der Achsen passieren.
Auch die generelle Genauigkeit des XY-Schreibers ist anzuzweifeln. Das Gerät ist sehr sensibel und die dargestellten Verläufe weisen
ein Zittern des Stiftes auf, welches allerdings auch von anderen Bauteilen wie dem Picoamperemeter ausgehen kann.
Die Dicke der Stiftmine ist eine weitere Fehlerquelle die das genaue Ablesen der Werte erschwert.
Auch die Einstellung der Spannungen mithilfe von Kippschaltern ist nicht optimal und führt zur Abweichungen, zum Beispiel beim Skalieren der
X-Achse.
Außerdem ist es mit dem Heizgenerator nicht möglich eine optimale Temperatur einzustellen, sodass die Temperatur während der Messungen 
leicht schwankt und somit die Messwerte verfälscht werden.
\pagebreak



