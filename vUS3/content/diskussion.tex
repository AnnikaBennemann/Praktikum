\section{Diskussion}
\label{sec:Diskussion}
Die im ersten Veruschsteil bestimmten Strömungsgeschwindigkeiten sind in \autoref{fig:plot} aufgetragen.
Es ist klar zu erkennen, dass die Werte dem erwarteten Verlauf folgen.
So steigt bei jeder Messreihe die Frequenzverschiebung linear zur Strömungsgeschwindigkeit an.
Dass die Geschwindigkeiten exakt den realen Werten entsprechen ist jedoch nicht
gesichert, da bei der Rechnung idealisierte Annahmen wie z.B. eine vollständig laminare
Dopplerflüssigkeit gemacht werden. Außerdem sind die Werte durch ständige Schwankungen nur sehr schwer abzulesen.


%Die Momentangeschwindigkeit beschreibt näherungsweise den zu erwarteten Verlauf im
%Rohr. Sie ist in der Mitte des Rohres größer als am Rand.