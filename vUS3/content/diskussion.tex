\section{Diskussion}
\label{sec:Diskussion}
Die im ersten Versuchsteil bestimmten Strömungsgeschwindigkeiten sind in \autoref{fig:plot} aufgetragen.
Es ist klar zu erkennen, dass die Werte dem erwarteten Verlauf folgen.
So steigt bei jeder Messreihe die Frequenzverschiebung linear zur Strömungsgeschwindigkeit an.
Dass die Geschwindigkeiten exakt den realen Werten entsprechen ist jedoch nicht
gesichert, da bei der Rechnung idealisierte Annahmen wie z.B. eine vollständig laminare
Dopplerflüssigkeit gemacht werden. Außerdem sind die Werte durch ständige Schwankungen nur sehr schwer abzulesen.


Die Momentangeschwindigkeit beschreibt im Bereich bis ca. $\qty{15}{\micro\second}$ näherungsweise den zu erwarteten Verlauf im
Rohr. Zu erwarten ist ein ungefähr parabelförmiger Verlauf, in dem die Geschwindigkeit in der Mitte des Rohres größer ist als am Rand.  
Jedoch stimmt der Intensitätsverlauf absolut nicht mit dem 
erwateten Verlauf (antiproportional zur Geschwindigkeit) überein. Die Intensität steigt während der Messung stark an.
Die Gründe dafür sind vielfältig.
Zum einen kann es daran liegen, dass die Antriebspumpe nicht ordnungsgemäß funktioniert.
Zum anderen können die Abweichungen von den erwarteten Verläufen auch an der Position der Sonde liegen, die eventuell nicht mit der
gesamten Fläche zu jeder Zeit an der richtigen Position liegt um optimale Ergebnisse zu erzielen.
Zudem haben Luftblasen im Kontaktgel zwischen dem Dopplerprisma und dem Rohr die Messung erschwert.