\section{Auswertung}
\label{sec:Auswertung}

\subsection{Vorbereitungsaufgabe}
Als Vorbereitungsaufgabe für den Versuch wurden die Dopplerwinkel $\alpha$ zu den Prismenwinkeln $\theta = 15° , 30°$ und $60°$
mithilfe von \autoref{eqn:alpha} bestimmt.
Die Schallgeschwindigkeiten $c_L=\qty{1800}{\meter\per\second}$ und $c_P=\qty{2700}{\meter\per\second}$ werden hierbei der Versuchsanleitung
\cite{VUS3} entnommen.
Die Ergebnisse sind in \autoref{tab:vor} eingetragen.

\begin{table}[H]
    \centering
    \caption{Prismenwinkel zu Dopplerwinkeln}
    \begin{tabular}{S[table-format=2.0] S[table-format=2.2]}
        \toprule
        {$\theta / \si{\degree}$} & {$\alpha  / \si{\degree}$}  \\
        \midrule
        15 & 80.06  \\
        30 & 70.53  \\
        60 & 54.74  \\
        \bottomrule
    \end{tabular}
    \label{tab:vor}
\end{table}

\subsection{Strömungsgeschwindigkeiten}
\label{sub:Strömungsgeschwindigkeiten}

Die Messung wird nach \autoref{sub:Strömungsgeschwindigkeiten_durch} durchgeführt und die Messwerte der Frequenzverschiebungen
zu den verschiedenen Pumpdrehzahlen in \autoref{tab:Mess1} eingetragen.
Die Drehzahlen werden nach
\begin{align*}
  v = 2 \pi r \frac{\omega}{60} 
\end{align*}\cite{Umrechnung}
umgestellt um die Vergleichsgeschwindigkeit zu berechnen.
Außerdem wird mithilfe der nach $v$ umgestellten \autoref{eqn:Deltanu}
\begin{align}
  v= \frac{\Delta \nu \cdot c}{2 \nu_0 \cos{\alpha}} \label{eqn:v}
\end{align}
die Strömungsgeschwindigkeit des Mediums berechnet und der Tabelle hinzugefügt. 

\begin{table}[H]
  \centering
  \caption{Messdaten zur Bestimmung der Strömungsgeschwindigkeiten ($D = \SI{10}{\milli\meter}$).}
  \label{tab:Mess1}
  \begin{tabular}{ S[table-format=2.2] S[table-format=4.0]
                  S[table-format=3.0] S[table-format=1.2] 
                  S[table-format=3.0] S[table-format=1.3] 
                  S[table-format=4.0] S[table-format=1.3]}
      \toprule
      & & 
      \multicolumn{2}{c}{$\theta = \ang{15;;}$} &
      \multicolumn{2}{c}{$\theta = \ang{30;;}$} & 
      \multicolumn{2}{c}{$\theta = \ang{60;;}$} \\
      \cmidrule(lr){3-4} \cmidrule(lr){5-6} \cmidrule(lr){7-8}
      {$\text{rpm}$}& {$v_{vergl} \mathbin{/} \si{\meter\per\second}$}&
      {$\symup{\Delta} \nu \mathbin{/} \si{\hertz}$} & {$v \mathbin{/} \si{\meter\per\second}$} & 
      {$\symup{\Delta} \nu \mathbin{/} \si{\hertz}$} & {$v \mathbin{/} \si{\meter\per\second}$} &
      {$\symup{\Delta} \nu \mathbin{/} \si{\hertz}$} & {$v \mathbin{/} \si{\meter\per\second}$} \\
      \midrule
      5000 & 2.62 & 208 & 0.54 & 377 & 0.51 & 670 & 0.52\\
      5250 & 2.75 & 275 & 0.71 & 428 & 0.58 & 753 & 0.59\\
      5500 & 2.88 & 309 & 0.81 & 500 & 0.68 & 840 & 0.65\\
      5750 & 3.01 & 339 & 0.88 & 563 & 0.76 & 938 & 0.73\\
      6000 & 3.14 & 375 & 0.98 & 612 & 0.83 & 1050 & 0.82\\
  \end{tabular}
\end{table}

Die jeweiligen Frequenzverschiebungen werden gegenüber der Strömungsgeschwindigkeit in \autoref{fig:plot} aufgetragen.
  
  \begin{figure}[H]
    \centering
    \includegraphics{build/plot1.pdf}
    \caption {Darstellung der Strömungsgeschwindigkeit gegenüber der Frequenzverschiebungen.}
    \label{fig:plot}
  \end{figure}

\subsection{Strömungsprofil}
\label{sub:Strömungsprofil}

Die Messung wird nach \autoref{sub:Strömungsprofil_durch} durchgeführt und die Messwerte in \autoref{tab:Messtiefe} eingetragen. Außerdem wird
nach \autoref{eqn:v} die Strömungsgeschwindigkeit berechnet und auch in \autoref{tab:Messtiefe} eingetragen.
\begin{table}[H]
  \centering
  \caption{Messdaten zur Bestimmung des Strömungsprofils.}
  \begin{tabular}{S[table-format=2.1] S[table-format=3.0] S[table-format=3.0] S[table-format=1.2]}
      \toprule
      {Messtiefe $d / \si{\micro\second}$}  & {Frequenzverschiebung $\symup{\Delta} \nu \mathbin{/} \si{\hertz}$} & {Intensität $I  / \si{\kilo\volt\squared\per\second}$} & {Geschwindigkeit $v  / \si{\meter\per\second}$} \\
      \midrule
      12,5  &  330 & 36  & 0.86 \\
      13,0  &  373 & 48  & 0.97 \\
      13,5  &  420 & 53  & 1.09 \\
      14,0  &  455 & 61  & 1.19 \\
      14,4  &  476 & 62  & 1.24 \\
      15,0  &  461 & 70  & 1.20 \\
      15,5  &  423 & 71  & 1.20 \\
      16,0  &  368 & 77  & 0.96 \\
      16,5  &  336 & 83  & 0.88 \\
      17,0  &  373 & 75  & 0.97 \\
      17,5  &  406 & 143 & 1.06 \\
      18,0  &  410 & 344 & 1.07 \\
      \bottomrule
  \end{tabular}
  \label{tab:Messtiefe}
\end{table}
Aus den Messwerten für die Strömungsgeschwindigkeit und die Streuintensität in \autoref{tab:Messtiefe} wird ein Diagramm (\ref{fig:plot2}) erstellt, indem die Werte 
als Funktion der Messtiefe aufgetragen werden.
\begin{figure}[H]
  \centering
  \includegraphics{build/plot2.pdf}
  \caption {Streuintensität und Momentangeschwindigkeit gegenüber der Messtiefe.}
  \label{fig:plot2}
\end{figure}
