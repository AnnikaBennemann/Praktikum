\section{Auswertung}
\label{sec:Auswertung}

\subsection{Vorbereitungsaufgabe}
Als Vorbereitungsaufgabe für den Versuch wurden die Dopplerwinkel $\alpha$ zu den Prismenwinkeln $\theta = 15° , 30°$ und $60°$
mithilfe von \autoref{eqn:alpha} bestimmt.
Die Schallgeschwindigkeiten $c_L=\qty{1800}{\meter\per\second}$ und $c_P=\qty{2700}{\meter\per\second}$ werden hierbei der Versuchsanleitung
\cite{VUS3} entnommen.
Die Ergebnisse sind in \autoref{tab:vor} eingetragen.

\begin{table}
    \centering
    \caption{Prismenwinkel zu Dopplerwinkeln}
    \begin{tabular}{S[table-format=2.0] S[table-format=2.2]}
        \toprule
        {$\theta / \si{\degree}$} & {$\alpha  / \si{\degree}$}  \\
        \midrule
        15 & 80.06  \\
        30 & 70.53  \\
        60 & 54.74  \\
        \bottomrule
    \end{tabular}
    \label{tab:vor}
\end{table}

\subsection{Strömungsgeschwindigkeiten}
\label{sub:Strömungsgeschwindigkeiten}

Die Messung wird nach \autoref{sub:Strömungsgeschwindigkeiten_durch} durchgeführt und die Messwerte der Frequenzverschiebungen
zu den verschiedenen Pumpdrehzahlen in \autoref{tab:Mess1} eingetragen.
Außerdem wird mithilfe der nach $v$ umgestellten \autoref{eqn:Deltanu}
\begin{align}
  v= \frac{\Delta \nu \cdot c}{2 \nu_0 \cos{\alpha}} \label{eqn:v}
\end{align}
die Strömungsgeschwindigkeit des Mediums berechnet und der Tabelle hinzugefügt. 

\begin{table}
  \centering
  \caption{Gemessene Frequenzverschiebungen
          und die daraus errechneten Strömungsgeschwindigkeiten ($D_\text{groß} = \SI{16}{\milli\metre}$)}
  \label{tab:Mess1}
  \begin{tabular}{S[table-format=4.0]
                  S[table-format=2.0] S[table-format=1.3] 
                  S[table-format=3.0] S[table-format=2.3] 
                  S[table-format=3.0] S[table-format=2.3]}
      \toprule
      &
      \multicolumn{2}{c}{$\theta = \ang{15;;}$} &
      \multicolumn{2}{c}{$\theta = \ang{30;;}$} & 
      \multicolumn{2}{c}{$\theta = \ang{60;;}$} \\
      \cmidrule(lr){2-3} \cmidrule(lr){4-5} \cmidrule(lr){6-7}
      {$\text{rpm}$}&
      {$\symup{\Delta} \nu \mathbin{/} \si{\hertz}$} & {$v \mathbin{/} \si{\milli\meter\second\tothe{-1}}$} & 
      {$\symup{\Delta} \nu \mathbin{/} \si{\hertz}$} & {$v \mathbin{/} \si{\milli\meter\second\tothe{-1}}$} &
      {$\symup{\Delta} \nu \mathbin{/} \si{\hertz}$} & {$v \mathbin{/} \si{\milli\meter\second\tothe{-1}}$} \\
      \midrule
      5000 & 208 & 24.366 & 377 & 18.790 & 670 & 16.347\\
      5250 & 275 & 30.333 & 428 & 25.225 & 753 & 21.697\\
      5500 & 309 & 36.300 & 500 & 29.858 & 840 & 27.196\\
      5750 & 339 & 42.267 & 563 & 31.403 & 938 & 30.911\\
      6000 & 375 & 48.731 & 612 & 40.927 & 1050 & 37.152\\
  \end{tabular}
\end{table}

vf15=  [0.54224156 0.71690591 0.80554155 0.88374947 0.97759897]
vf30=  [0.50998789 0.5789783  0.67637651 0.76159996 0.82788485]
vf60=  [0.52226991 0.58696902 0.65478615 0.73117787 0.81848269]
  %Tabellen werte anpassen und andere 2 Tabellen noch erstellen!!!!!!!!!


\subsection{Strömungsprofil}
\label{sub:Strömungsprofil}


