\section{Zielsetzung}
\label{sec:Zielsetzung}
Ziel dieses Versuches ist Untersuchung der Funktionsweise sowie die charakteristichen
Parameter eines Geiger-Müller-Zählrohrs bei der Detektion von ionisierender Strahlung.
\section{Theorie}
\label{sec:Theorie}
Im Folgendem wird der Grundlegende Aufbau eines Geiger-Müller-Zählrohrs erklärt und 
anschließend die physikalischen Abläufe im Inneren, sowie die Zählrohrcharakteristik.
\subsection{Aufbau und Funktionsweise eines Geiger-Müller-Zählrohrs}
\label{AufuFunk}
Das Geiger-Müller-Zählrohr wird dazu verwendet um die Intensität der ionisierender Strahlung
zu messen. Der Aufbau besteht aus einem Hohlzylinder, dessen Außenwand mit dem Radius $r_A$ als Kathode
dient. 

