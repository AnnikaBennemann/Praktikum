\section{Fehlerrechnung}
\label{sec:Fehlerrechnung}

Der Mittelwert wird durch die Formel
\begin{align*}
    \bar{x}=\frac{1}{n} \sum_{i=1}^n x_i \label{eqn:Mittelwert}
\end{align*}
berechnet.

Die Standardabweichung ergibt sich durch
\begin{align*}
    \sigma_x=\sqrt{\frac{1}{n-1}\sum_{j=1}^n (x_j-\bar{x_i})^2},
\end{align*}
somit lautet der Fehler des Mittelwerts
\begin{align*}
    \Delta \bar{x}= \frac{\sigma_x}{n} = \frac{\sqrt{\frac{1}{n-1}\sum_{j=1}^n (x_j-\bar{x_i})^2}}{\sqrt{n}}.
\end{align*}

Wird mit fehlerbehafteten Größen weiter gerechnet, muss der Fehler des Ergebnisses mit der Fehlerfortpflanzung nach Gauß
\begin{equation*}
    \Delta f=\sqrt{\sum_{j=1}^n \left(\frac{\partial f}{\partial x_j}\Delta x_j \right)^{2} }\label{eqn:Gauß}
\end{equation*}
bestimmt werden.

Mit Hilfe der linearen Regression 
\begin{subequations}
    \label{eqn:lin}
\begin{align}
    y(x)=&A+B\cdot x\\
    \intertext{werden Ausgleichsgeraden zu den Messwerten erstellt. Die Parameter der Ausgleichsgeraden berechnen sich zu}
    A&=\overline{y}-B\cdot \overline{x}\label{eqn:lina}\\
    B&=\frac{\overline{xy}-\overline{x}\cdot \overline{y}}{\overline{x^2}-\overline{x}^2}\label{eqn:linb}.
\end{align}
\end{subequations}

Abweichungen von den Theoriewerten werden mit der Formel
\begin{align}
  a=\frac{|a_\mathrm{gemessen}-a_\mathrm{theorie}|}{a_\mathrm{theorie}} \label{eqn:abweich}
\end{align}
berechnet.

Die Berechnungen der Mittelwerte, Standardabweichungen, sowie der weiteren Fehler und die Ausgleichsgeraden der linearen Regression und weiteren Funktionen
wird im folgenden mit Hilfe von Python und den Pythonmodulen Matplotlib \cite{matplotlib}, Scipy \cite{scipy}, Uncertainties \cite{uncertainties} und Numpy \cite{numpy}
durchgeführt.