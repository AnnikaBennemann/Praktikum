\section{Durchführung}
\label{sec:Durchführung}

%3.1 Versuchsaufbau
%Das Michelson-Interferometer ist wie in Abbildung 2 aufgebaut.
%Um Interferenzen beobachten zu können, wird der Laserstrahl zunächst durch einen
%Strahlteiler, realisiert als eine semipermeable Platte, in zwei Teilbündel aufgeteilt.
%Für die spätere Messung ist einer der Strahlwege von variabler Länge. Dazu ist
%der Spiegel des oberen Strahlwegs durch einen Synchronmotor mit Zehnganggetriebe
%
%7
%
%Abbildung 2: Schematischer Aufbau der Messapparatur. [1]
%
%verschiebbar.
%Zusätzlich ist im oberen Strahlweg eine Messzelle mit Breite b = 50 mm installiert. Diese
%lässt sich sowohl evakuieren, als auch mit verschiedenen Gasen befüllen, sodass durch
%den geänderten Brechungsindex in der Messzelle ebenso ein optischer Wegunterschied
%zwischen beiden Strahlwegen entsteht.
%Im rechten Strahlweg ist zudem eine Ausgleichsplatte angebracht, da dieser Strahlweg im
%Gegensatz zum anderen Strahlweg nicht dreimal, sondern nur einmal durch den in der
%Mitte angebrachten Strahlteiler führt.
%Beide Strahlwege treffen schließlich auf das Photoelement, an welchem die ankommenden
%Lichtsignale registriert werden und über einen Verstärker und einen Impulsformer an ein
%elektronisches Zählwerk gegeben wird.
%3.2 Versuchsbeschreibung
%Zu Beginn des Versuchs muss das Michelson-Interferometer für die Messung justiert
%werden. Dazu wird der Laser eingeschaltet und an die Position des Photoelements (vgl.
%Abbildung 2) eine Mattscheibe eingebracht.
%Der justierbare Spiegel wird so ausgerichtet, dass die beiden hellsten Intensitätsmaxima
%der beiden ankommenden Strahlen möglichst genau zur Deckung gebracht werden. Das
%Photoelement wird entsprechend so ausgerichtet, dass das Intensitätsmaxima beider
%Strahlen genau auf den Eintrittsspalt des Photoelements liegt.
%Für die Messung der Wellenlänge des Lasers wird der verschiebbare Spiegel genutzt.
%Dazu wird der Synchronmotor eingeschaltet und eine Verschieberichtung ausgewählt. Zu
%
%beachten ist hierbei, dass der Motor nicht zu schnell bewegt wird, da sonst das Photo-
%8
%
%element nicht alle ankommenden konstruktiven Interferenzen, also Intensitätsmaxima
%der ankommenden Strahlen, als getrennte Intensitätsmaxima eindeutig zählen kann und
%somit das falsche Ergebnis liefert.
%Bei Bedarf muss ein kleinerer Gang ausgewählt werden. Im vorliegenden Experiment
%wurde die Messung im kleinsten Gang durchgeführt. Nachdem etwa 1000 konstruktive
%Interferenzen der Teilstrahlen am Photoelement registriert wurden, wird die exakte
%Anzahl an registrierten Interferenzen notiert sowie die Verschiebestrecke des Spiegels über
%die Verschiebung der Mikrometerschraube bestimmt. Dazu wird die Position zu Beginn
%und zum Ende der Verschiebung an der Mikrometerschraube abgelesen und mittels der
%Hebelübersetzung die Verschiebestrecke Δd berechnet.
%Die Messung wird etwa sieben- bis zehnmal durchgeführt.
%Zur Bestimmung des Brechungsindex von Luft wird der verschiebbare Spiegel nicht
%mehr bewegt.
%Die Messzelle wird mittels der Vakuumpumpe auf den Druck p evakuiert, welcher notiert
%wird. Beim langsamen Wiedereinlassen der Luft werden erneut die konstruktiven Interfe-
%renzen am Photoelement gezählt und sobald wieder der Normaldruck p0
%in der Messzelle
%herrscht, wird deren Anzahl z notiert.
%Die Messung wird sechs Mal wiederholt.
%