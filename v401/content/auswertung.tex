\section{Auswertung}
\label{sec:Auswertung}

\subsection{Bestimmung der Wellenlänge des Lasers}
\label{subsec:Wellenlänge}
Zur Bestimmung der Wellenlänge wird der Verscuh, wie in \autoref{sec:Durchführung} durchgeführt.
Die Anzahl $z$ der Maxima und die Abstandsänderung $\Delta d$ sind in \autoref{tab:Wellenlänge} eingetragen.

\begin{table}[H]
  \centering
  \caption{Messdaten zur Bestimmung der Wellenlänge des Lasers.}
  \label{tab:Wellenlänge}
  \sisetup{table-format=2.2}
  \begin{tabular}{S[table-format=1.2] S[table-format=4.0] S[table-format=3.2]}
  \toprule
  {$\Delta d / \si{\milli\meter}$} & {Anzahl $z$ der Intensitätsmaxima} & {Wellenlänge $\lambda / \si{\nano\meter}$}\\
  \midrule
    3.38  & 2005 & 668.17 \\
    3.37  & 2002 & 667.19 \\
    3.36  & 2006 & 663.88 \\
    3.39  & 2003 & 670.81 \\
    3.39  & 2007 & 669.48 \\
    3.38  & 2001 & 669.50 \\
    3.38  & 2003 & 668.83 \\
    3.37  & 1999 & 668.19 \\
    3.38  & 2003 & 668.83 \\
    3.38  & 2003 & 668.83 \\
  \bottomrule
  \end{tabular}
\end{table}

Aus den gemessenen Daten wird durch Umstellung von \autoref{eqn:d} nach
\begin{align*}
  \lambda = \frac{\Delta d \cdot 2}{ U \cdot z}
\end{align*}
die Wellenlänge des Lasers berechnet und auch in \autoref{tab:Wellenlänge} eingetragen.
$U$ steht hierbei für die Übersetzung des Motors mithilfe dessen der Spiegel verschoben wird.
Bei dem gegebenen Motor beträgt die Übersetzung $U= 1:5,046$. 
Der Mittelwert der berechneten Wellenlängen ergibt sich durch
\begin{align}
  \bar{\lambda}=\frac{1}{m} \sum_{i=1}^m \lambda_i.
  \label{eqn:Mittelwert}
\end{align}
Der Fehler des Mittelwerts wird durch 
\begin{align}
  \Delta \bar{\lambda}= \frac{\sigma_\lambda}{\sqrt{m}} = \frac{\sqrt{\frac{1}{m-1}\sum_{j=1}^m (\lambda_j-\bar{\lambda})^2}}{\sqrt{m}}
  \label{eqn:Fehler}
\end{align}
berechnet.
Somit ergibt sich als berechnete Wellenlänge des Lasers
\begin{align*}
  \lambda= (668.37 \pm 0.58 )\si{\nano\meter}.
\end{align*}

Der reale Wert der Wellenlänge des Lasers ist auf dem Etikett des Lasers mit $\lambda_{\text{real}}=\qty{635}{\nano\meter}$ angegeben.
Im Folgenden wird mit diesem Wert weitergerechnet.
Die Abweichung zu $\lambda_{\text{real}}$ wird durch
\begin{align}
  a=\frac{|\lambda-\lambda_{\text{real}}|}{\lambda_{\text{real}}}\cdot 100 \label{eqn:abweich}
\end{align}
zu
\begin{align*}
  a=(5,26 \pm 0,09) \si{\percent}
\end{align*}
berechnet.

\subsection{Bestimmung des Brechungsindexes von Luft}
\label{sub:Brechungsindex}

Im zweiten Teil des Versuchs soll der Brechungsindex von Luft bestimmt werden.
Nach \autoref{sec:Durchführung} wird der Versuch durchgeführt und die Messwerte in \autoref{tab:Brechungsindex} eingetragen.
Zur Berechnung des Brechungsindex nach \autoref{eqn:n}
werden die folgenden Werte verwendet\cite{V401}:
\begin{align*}
  T_0 &= \qty{273.15}{\kelvin}\\
  p_0 &= \qty{1.0132}{\bar}\\
  T &= \qty{293.15}{\kelvin}\\
  b &=\qty{50}{\milli\meter}\\
  \lambda &=\qty{635}{\nano\meter}
\end{align*}

\begin{table}[H]
  \centering
  \caption{Messdaten zur Bestimmung des Brechungsindexes von Luft.}
  \label{tab:Brechungsindex}
  \sisetup{table-format=2.2}
  \begin{tabular}{S[table-format=1.1] S[table-format=2.0] S[table-format=1.6]}
  \toprule
  {Luftdruck $ p / \si{\bar}$} & {Anzahl $z$ der Intensitätsmaxima} & {Brechungsindex $n$}\\
  \midrule
    0.6 & 19 & 1.000328 \\
    0.6 & 17 & 1.000293 \\
    0.6 & 21 & 1.000363 \\
    0.6 & 16 & 1.000276 \\
    0.6 & 17 & 1.000293 \\
    0.6 & 17 & 1.000293 \\
  \bottomrule
  \end{tabular}
\end{table}

Analog zu \autoref{eqn:Mittelwert} und \autoref{eqn:Fehler} berechnet sich der Mittelwert des Brechungsindexes zu
\begin{align*}
  \bar{n}= (1.000308 \pm 0.000013)
\end{align*}
Die Abweichung vom Literaturwert \cite{Brechungsindex} des Brechungsindexes von Luft $n_{\text{lit}}= 1,000272$
wird analog zu \autoref{eqn:abweich} berechnet und beträgt
\begin{align*}
  a=(0,0036 \pm 0,0013) \si{\percent}.
\end{align*}

