\section{Auswertung}
\label{sec:Auswertung}



\begin{table}[H]
  \centering
  \caption{Messdaten zur Bestimmung der Wellenlänge des Lasers.}
  \label{tab:Wellenlänge}
  \sisetup{table-format=2.2}
  \begin{tabular}{S[table-format=1.2] S[table-format=4.0] S[table-format=3.2]}
  \toprule
  {$\Delta d / \si{\milli\meter}}$} & {Anzahl $z$ der Intensitätsmaxima} & {Wellenlänge $\lambda / \si{\nano\meter}$}\\
  \midrule
    3.38  & 2005 & 668.17 \\
    3.37  & 2002 & 667.19 \\
    3.36  & 2006 & 663.88 \\
    3.39  & 2003 & 670.81 \\
    3.39  & 2007 & 669.48 \\
    3.38  & 2001 & 669.50 \\
    3.38  & 2003 & 668.83 \\
    3.37  & 1999 & 668.19 \\
    3.38  & 2003 & 668.83 \\
    3.38  & 2003 & 668.83 \\
  \bottomrule
  \end{tabular}
\end{table}

Aus den gemessenen Daten wird mithilfe von !!!! die Wellenlänge des Lasers berechnet und auch in \autoref{tab:Wellenlänge} eingetragen.
Der 
\begin{align}
  \bar{\lambda}=\frac{1}{m} \sum_{i=1}^m \lambda_i
  \label{eqn:Mittelwert}
\end{align}


Der reale Wert der Wellenlänge des Lasers ist auf dem Etikett des Lasers mit $\lamda_{\text{real}}=\qty{635}{\nano\meter}$ angegeben.
Im Folgenden wird mit diesem Wert weitergerechnet.



\begin{table}[H]
  \centering
  \caption{Messdaten zur Bestimmung des Brechungsindexes von Luft.}
  \label{tab:Brechungsindex}
  \sisetup{table-format=2.2}
  \begin{tabular}{S[table-format=1.1] S[table-format=2.0] S[table-format=1.6]}
  \toprule
  {Luftdruck $ p / \si{\bar}}$} & {Anzahl $z$ der Intensitätsmaxima} & {Brechungsindex $n$}\\
  \midrule
    0.6 & 19 & 1.000328
    0.6 & 17 & 1.000293
    0.6 & 21 & 1.000363
    0.6 & 16 & 1.000276
    0.6 & 17 & 1.000293
    0.6 & 17 & 1.000293
  \bottomrule
  \end{tabular}
\end{table}


