\section{Diskussion}
\label{sec:Diskussion}

Für die zu bestimmende Wellenlänge des Lasers wurde der Wert
\begin{align*}
    \lambda= (668.37 \pm 0.58 )\si{\nano\meter}    
\end{align*}
ermittelt. Dieser Wert weicht um
\begin{align*}
    a=(5,26 \pm 0,09) \si{\percent}
\end{align*}
vom gegebenen Wert $\lambda_{\text{real}}=\qty{635}{\nano\meter}$ ab. 
Die Messung ist also relativ genau.
Die Ursache der Abweichungen kann eine ungenaue Justierung des Lasers sein, wodurch vom Detektor nicht jedes Maximum erkannt wurde.
Außerdem wurde der Motor manuell gestartet und gestoppt, wodurch kleine Erschütterungen am Messtisch stattfanden. Der Aufbau ist jedoch
sehr sensibel, weshalb Erschütterungen am Versuch die Messung verfälschen können.
Auch menschliche Ablesefehler bei der Mikrometerschraube des Motors sind als Fehlerquelle nicht auszuschließen.\\
\\  
Der Brechungsindex von Luft wurde durch den Versuch zu
\begin{align*}
    n= (1.000308 \pm 0.000013)
\end{align*}
bestimmt.
Die relative Abweichung zum Literaturwert $n_{\text{lit}}= 1,000272$ \cite{Brechungsindex} beträgt hier 
\begin{align*}
    a=(0,0036 \pm 0,0013) \si{\percent}. 
\end{align*}
Eine Fehlerquellen im zweiten Versuchsteil ist vor allem die Vakuumpumpe, da diese per Hand bedient wurde und somit Erschütterungen
ausgelöst hat. Außerdem ist die Skala an der Pumpe nicht sehr genau, sodass es zu Ablesefehlern gekommen sein kann.
Das Wiedereinlassen der Luft erfolgt auch manuell, sodass es auch hier durch ungleichmäßiges oder zu schnelles Einlassen der Luft dazu kommen kann,
dass nicht alle Interferenzmaxima vom Detektor erkannt werden. \\
\\
Insgesamt sind die Abweichungen in beiden Versuchsteilen jedoch sehr gering ausgefallen, sodass sich daraus schließen lässt,
dass sich das Michelson-Interferometer gut für die Messung von Wellenlängen und Brechungsindizes eignet.
