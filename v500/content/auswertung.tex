\section{Auswertung}
\label{sec:Auswertung}
Der Versuch wird wie in \autoref{sec:Durchführung} beschrieben durchgeführt.

\subsection{Bestimmung der Grenzfrequenz für verschiedenen Wellenlängen} % (fold)
\label{sub:Grenzfrequenz_aus}



\begin{table}[H]
  \centering
  \caption{Messwerte für das gelbe Licht.}
  \label{tab:gelb}
  \sisetup{table-format=2.2}
  \begin{tabular}{S S[table-format=1.3] }
  \toprule
  {Spannung $U / \si{\volt}$} & {Stromstärke $ I / \si{\nano\ampere}$}\\
  \midrule
  -0.56 &  0     \\
  -0.50 &  0.004 \\
  -0.40 &  0.014 \\
  -0.30 &  0.038 \\
  -0.20 &  0.066 \\
  -0.10 &  0.090 \\
  -0.02 &  0.100 \\
   0.02 &  0.120 \\
   0.10 &  0.125 \\
   0.20 &  0.130 \\
   0.30 &  0.150 \\
   0.40 &  0.160 \\
   0.50 &  0.200 \\
   1.00 &  0.420 \\
   1.50 &  0.500 \\
   2.00 &  0.560 \\
   2.50 &  0.580 \\
   3.00 &  0.650 \\
   3.50 &  0.550 \\
   4.00 &  0.600 \\
   4.50 &  0.800 \\
   5.00 &  0.700 \\
   6.00 &  0.700 \\
   7.00 &  0.750 \\
   8.00 &  0.750 \\
   9.00 &  0.800 \\
  10.00 &  0.850 \\
  12.50 &  0.900 \\
  15.00 &  0.950 \\
  17.50 &  1.000 \\
  19.00 &  1.600 \\
  \bottomrule
  \end{tabular}
\end{table}

\begin{table}[H]
  \centering
  \caption{Messwerte für das grüne Licht.}
  \label{tab:grün}
  \sisetup{table-format=2.2}
  \begin{tabular}{S S[table-format=1.3] }
  \toprule
  {Spannung $U / \si{\volt}$} & {Stromstärke $ I / \si{\nano\ampere}$}\\
  \midrule
  -0.02 &  0.310 \\
  -0.10 &  0.280 \\
  -0.20 &  0.220 \\
  -0.30 &  0.200 \\
  -0.40 &  0.090 \\
  -0.45 &  0.050 \\
  -0.50 &  0.048 \\
  -0.55 &  0.020 \\
  -0.60 &  0.010 \\
  -0.65 &  0.004 \\
  -0.70 &  0     \\
  \bottomrule
  \end{tabular}
\end{table}

\begin{table}[H]
  \centering
  \caption{Messwerte für das türkise Licht.}
  \label{tab:türkis}
  \sisetup{table-format=2.2}
  \begin{tabular}{S S[table-format=1.3] }
  \toprule
  {Spannung $U / \si{\volt}$} & {Stromstärke $ I / \si{\nano\ampere}$}\\
  \midrule
  -0.02 &  0.030 \\
  -0.10 &  0.022 \\
  -0.20 &  0.020 \\
  -0.30 &  0.018 \\
  -0.40 &  0.016 \\
  -0.50 &  0.010 \\
  -0.60 &  0.008 \\
  -0.70 &  0.004 \\
  -0.80 &  0.001 \\
  -0.90 &  0     \\
  \bottomrule
  \end{tabular}
\end{table}

\begin{table}[H]
  \centering
  \caption{Messwerte für das blaue Licht.}
  \label{tab:blau}
  \sisetup{table-format=2.2}
  \begin{tabular}{S S[table-format=1.3] }
  \toprule
  {Spannung $U / \si{\volt}$} & {Stromstärke $ I / \si{\nano\ampere}$}\\
  \midrule
  -0.02 &  1.250 \\
  -0.10 &  0.750 \\
  -0.20 &  0.700 \\
  -0.30 &  0.620 \\
  -0.40 &  0.500 \\
  -0.50 &  0.400 \\
  -0.60 &  0.300 \\
  -0.70 &  0.280 \\
  -0.80 &  0.220 \\
  -0.90 &  0.180 \\
  -1.00 &  0.090 \\
  -1.10 &  0.020 \\
  -1.20 &  0.002 \\
  -1.22 &  0     \\
  \bottomrule
  \end{tabular}
\end{table}

\begin{table}[H]
  \centering
  \caption{Messwerte für das violette Licht.}
  \label{tab:violett}
  \sisetup{table-format=2.2}
  \begin{tabular}{S S[table-format=1.3] }
  \toprule
  {Spannung $U / \si{\volt}$} & {Stromstärke $ I / \si{\nano\ampere}$}\\
  \midrule
  -0.02 &  0.400 \\
  -0.10 &  0.320 \\
  -0.20 &  0.200 \\
  -0.30 &  0.190 \\
  -0.40 &  0.175 \\
  -0.50 &  0.150 \\
  -0.60 &  0.120 \\
  -0.70 &  0.090 \\
  -0.80 &  0.070 \\
  -0.90 &  0.050 \\
  -1.00 &  0.030 \\
  -1.10 &  0.020 \\
  -1.20 &  0.015 \\
  -1.30 &  0.005 \\
  -1.40 &  0     \\
  \bottomrule
  \end{tabular}
\end{table}
%ende von Grebzfrequenz
\subsection{Bestimmung von \frac{h}{e} und der Austrittsarbeit} % (fold)
\label{sub:Austrittsarbeit}


Aus den gegebenen Wellenlängen(\cite[80]{v500}) werden die Frequenze für verschiedene Farben berechnet mit
\begin{equation}
  \nu = \frac{c}{\lambda}
\end{equation} 
und in \autoref{tab:Frequenzen} eingetragen.
\begin{table}[H]
  \centering
  \caption{Berechnete Frequenzen für verschiedene Farben.}
  \label{tab:Frequenzen}
  \sisetup{table-format=3.2}
  \begin{tabular}{c S[table-format=3.0] S }
  \toprule
  {Farbe} & {$\lambda / \si{\nano\metre}$} & {$\nu / \si{\tera\hertz}$}\\
  \midrule
    gelb    & 578 & 518.67 \\
    grün    & 546 & 549.07 \\
    türkis  & 492 & 609.33 \\
    violett & 435 & 689.18 \\
    blau    & 408 & 734.79 \\
  \bottomrule
  \end{tabular}
\end{table}

%ende von Austrittsarbeit

\subsection{Nähere Betrachtung des Photostroms vom gelben Licht} % (fold)
\label{sub:Nähere Betrachtung des Photostroms vom gelben Licht}

\begin{figure}
  \centering
  \includegraphics[width=\textwidth]{plot1.pdf}
  \caption{Photostrom für gelbes Licht aufgetragen gegen die Spannung.}
  \label{fig:plot1}
\end{figure}

% subsection Nähere Betrachtung des Photostroms vom gelben Licht (end)
